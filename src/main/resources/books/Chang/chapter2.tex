%  c21.html 
\hypertarget{c2}{}

\chapter{Stimmen des Klaviers}\hypertarget{c2_1}{}

\section{Einleitung}

\textbf{\textit{{\normalsize [Achtung: Beim Stimmen, Intonieren und anderen Wartungsarbeiten kann ein Klavier durch Mangel an Vorsicht, unsachgemäßes Vorgehen, usw. beschädigt werden!
 Wenn jemand zum ersten Mal die Mechanik ausbaut und daran arbeitet, stehen die Chancen ungefähr 1:1, \underline{daß}  er etwas beschädigt.
 Sie sollten sich deshalb eingehend mit dem Thema beschäftigen, bevor (!) Sie Hand an Ihr eigenes oder ein fremdes Klavier legen.
 Lassen Sie sich ggf. von jemandem beraten, der sich mit der Materie auskennt, und beschaffen Sie sich gute weiterführende Literatur zu dem Thema, wie z.B. das von Chuan C. Chang angeführte Buch \enquote{Piano Servicing, Tuning, and Rebuilding} von Arthur Reblitz.]}}}

\textbf{Dieses Kapitel ist für diejenigen bestimmt, die ihr Klavier noch nie selbst gestimmt haben und sehen möchten, ob sie der Aufgabe gewachsen sind.}
 Das Buch \textit{Piano Servicing, Tuning, and Rebuilding} von Arthur Reblitz ist dabei ein sehr hilfreiches Nachschlagewerk.
 \footnote{Hinweise für deutsche Bücher gleicher Qualität nehme ich gerne hier auf.}
 Der schwierigste Teil beim Lernen des Stimmens ist das Anfangen.
 Wer in der glücklichen Lage ist, jemanden zu haben, der ihn unterrichtet, ist natürlich am besten dran.
 Unglücklicherweise sind Lehrer für das Klavierstimmen nicht ohne weiteres verfügbar.
 Probieren Sie die Vorschläge in diesem Kapitel aus, und sehen Sie, wie weit sie kommen.
 Nachdem Sie sich darüber im klaren sind, was Ihnen Probleme bereitet, können Sie mit Ihrem Stimmer über 30-minütige Lektionen mit einer vereinbarten Vergütung verhandeln oder ihn bitten zu erklären, was er tut, wenn er stimmt.
 Seien Sie darauf bedacht, Ihrem Stimmer nicht zuviel aufzubürden; Stimmen und Unterrichten kann mehr als viermal so lang dauern wie nur zu stimmen.
 Seien Sie auch vorgewarnt, daß Klavierstimmer keine ausgebildeten Lehrer sind, und einige von ihnen mögen unberechtigte Befürchtungen hegen, daß sie einen Kunden verlieren könnten.
 Diese Befürchtungen sind unbegründet, weil die tatsächliche Zahl der Menschen, die professionelle Stimmer erfolgreich ersetzt haben, vernachlässigbar klein ist.
 Am wahrscheinlichsten bekommen Sie am Ende ein besseres Verständnis dafür, was es bedeutet, ein Klavier zu stimmen.
 

Für Klavierspieler bietet das Vertrautwerden mit der Kunst des Stimmens eine Ausbildung, die für ihre Fähigkeit Musik zu erzeugen und ihre Instrumente zu warten sehr wichtig ist.
 Es versetzt sie auch in die Lage, vernünftig mit ihren Stimmern zu kommunizieren.
 So kannte z.B. die Mehrzahl der Klavierlehrer, denen ich die Frage stellte, noch nicht einmal den Unterschied zwischen \hyperlink{et1}{gleichschwebender Temperatur} und historischen Stimmungen.
 Der Hauptgrund, warum die meisten Menschen versuchen das Stimmen zu lernen, ist aus Neugier - für die meisten ist das Klavierstimmen ein rätselhaftes Geheimnis.
 Sind sie erst einmal über die Vorteile eines gestimmten (gewarteten) Klaviers unterrichtet, ist es wahrscheinlicher, daß sie ihren Stimmer regelmäßig rufen.
 Klavierstimmer können bestimmte Töne, die vom Klavier kommen, hören, die die meisten Menschen, sogar Pianisten, nicht wahrnehmen.
 Diejenigen, die das Stimmen üben, werden für die Töne von verstimmten Klavieren sensibilisiert.
 Unter der Annahme, daß Sie die Zeit haben, mindestens einmal alle ein oder zwei Monate für mehrere Stunden zu üben, wird es wahrscheinlich ungefähr ein Jahr dauern, bis Sie anfangen mit dem Stimmen zurechtzukommen.
 

Lassen Sie mich hier ein wenig abschweifen, um zu besprechen, wie wichtig es unter dem Gesichtspunkt, von dem Stimmer einen Gegenwert für Ihr Geld zu bekommen, so daß Ihr Klavier richtig gewartet werden kann, ist, die Lage des Klavierstimmers zu verstehen und richtig mit ihm zu kommunizieren.
 Diese Überlegungen haben sowohl eine direkte Auswirkung auf Ihre Fähigkeit, sich Klaviertechnik anzueignen als auch auf Ihre Entscheidungen darüber, was oder wie Sie bei einem Auftritt vorspielen, wenn Sie ein bestimmtes Klavier zur Verfügung haben.
 Eine der verbreitetsten Schwierigkeiten, die ich z.B. bei Schülern festgestellt habe, ist ihre Unfähigkeit, pianissimo zu spielen.
 Aus meinem Verständnis des Klavierstimmens heraus gibt es dafür eine einfache Erklärung - die meisten Klaviere dieser Schüler werden zu wenig gewartet.
 Die Hämmer sind zu abgenutzt bzw. verdichtet und die Mechanik so sehr verstellt, daß pianissimo zu spielen unmöglich ist.
 Diese Schüler werden nicht einmal in der Lage sein, pianissimo zu üben!
 Das gilt auch für den musikalischen Ausdruck und die Tonkontrolle.
 Diese zu wenig gewarteten Klaviere sind wahrscheinlich eine der Ursachen für die Ansicht, daß Klavierüben eine Qual für die Ohren ist, aber das sollte es nicht sein.
 

Ein weiterer Faktor ist, daß man sich im allgemeinen das Klavier nicht aussuchen kann, wenn man gebeten wird, etwas vorzuspielen.
 Man kann auf alles treffen: von einem wundervollen Konzertflügel über ein Kleinklavier bis zu (Schock!) einem billigen Stutzflügel, der völlig vernachlässigt wurde, seit er vor 40 Jahren gekauft wurde.
 Ihr Verständnis dafür, was man mit jedem dieser Klaviere tun kann bzw. nicht tun kann, sollte der erste Punkt bei der Entscheidung sein, was und wie man spielt.
 

Wenn Sie erst einmal angefangen haben, das Stimmen zu üben, werden Sie schnell verstehen, warum es einem genauen und qualitativen Stimmen nicht förderlich ist, wenn jemand dabei Staub saugt, Kinder herumspringen, der Fernseher oder die Stereoanlage plärrt oder in der Küche die Töpfe klappern, und warum ein Stimmen auf die Schnelle für 70 Euro kein Schnäppchen ist im Vergleich zu einem Stimmen für 150 Euro, bei dem der Stimmer die Hämmer neu formt und nadelt.
 Wenn man aber die Besitzer fragt, was der Stimmer an ihrem Klavier getan hat, haben sie im allgemeinen keine Vorstellung davon.
 Eine Beschwerde, die ich oft von Besitzern höre, ist, daß das Klavier nach dem Stimmen tot oder schrecklich klingen würde.
 Das geschieht oft, wenn der Besitzer keinen festen Bezugspunkt hat, von dem aus er den Klavierklang beurteilen kann - das Urteil basiert darauf, ob der Besitzer den Klang mag oder nicht.
 Solche Wahrnehmungen sind allzuoft durch die Vorgeschichte des Besitzers falsch beeinflußt.
 Der Besitzer kann sich tatsächlich an den Klang eines verstimmten Klaviers mit verdichteten Hämmern gewöhnen, so daß wenn der Stimmer den Klang wiederherstellt, der Besitzer diesen nicht mag, weil er sich nun zu sehr von dem gewohnten Klang oder Gefühl unterscheidet.
 Der Stimmer könnte sicherlich Schuld daran haben; der Besitzer braucht jedoch ein minimales Wissen über technische Details des Stimmens, um solch ein Urteil hieb- und stichfest zu machen.
 Der Nutzen eines Verständnisses für das Stimmen und der richtigen Wartung des Klaviers wird offensichtlich von der Allgemeinheit unterschätzt.
 Vielleicht ist das wichtigste Ziel dieses Kapitels, dieses Bewußtsein zu vergrößern.
 

\textbf{Klavierstimmen erfordert - im Gegensatz zum \hyperlink{c1iii12}{absoluten Gehör} - keine guten Ohren, weil das ganze Stimmen durch den Vergleich mit einer Referenz ausgeführt wird, bei dem Schwebungen benutzt werden und man mit der Bezugsfrequenz einer Stimmgabel beginnt.}
 Tatsächlich kann die Fähigkeit des absoluten Gehörs bei einigen Menschen mit dem Stimmen in Konflikt geraten.
 Deshalb ist die \enquote{einzige} notwendige Hörfertigkeit die Fähigkeit, die verschiedenen Schwebungen zu hören und zwischen ihnen zu unterscheiden, wenn zwei Saiten angeschlagen werden.
 Diese Fähigkeit entwickelt sich durch Übung und ist nicht mit dem Wissen über Musiktheorie oder mit Musikalität verknüpft.
 Größere Flügel sind leichter zu Stimmen als \enquote{\hyperlink{upright}{Aufrechte}}; die meisten Stutzflügel sind jedoch schwieriger zu stimmen als gute \enquote{Aufrechte}.
 Obwohl man seine Übungen logischerweise an einem qualitativ schlechteren Klavier beginnen sollte, wird dieses deshalb schwieriger zu stimmen sein.
 
%  c22.html 
\hypertarget{c2_2}{}

\section{Chromatische Tonleiter und Temperaturen}\hypertarget{c2_2a}{}

\subsection{Einleitung}

Die meisten von uns sind mit der chromatischen Tonleiter einigermaßen vertraut und wissen, daß sie temperiert sein muß, aber was sind die präzisen Definitionen der beiden Begriffe?
 \textbf{Warum ist die chromatische Tonleiter so besonders, und warum ist das Temperieren notwendig?}
 Wir erforschen zunächst die mathematische Grundlage der chromatischen Tonleiter und des Temperierens, weil der mathematische Ansatz die knappste, klarste und präziseste Behandlung ist.
 Wir besprechen dann die historischen und musikalischen Gesichtspunkte, damit wir die relativen Vorzüge der verschiedenen Stimmungen besser verstehen.
 Ein grundlegendes mathematisches Fundament dieser Konzepte ist entscheidend für ein gutes Verständnis dafür, wie Klaviere gestimmt werden.
 Informationen über das Stimmen finden Sie bei White, Howell, Fischer, Jorgensen oder Reblitz (s. \hyperlink{reference}{Quellenverzeichnis}).
 \hypertarget{c2_2b}{}\hypertarget{c2_2_math}{}

\subsection{Mathematische Behandlung}

In Tabelle 2.2a sind drei Oktaven aufgeführt.
 Die schwarzen Tasten des Klaviers werden mit Kreuzen dargestellt, z.B. steht das \# rechts von C für ein C\#, und ist nur bei der höchsten Oktave ausgewiesen.
 \textbf{Jede der aufeinanderfolgenden Frequenzänderungen in der chromatischen Tonleiter wird ein Halbton genannt, und eine Oktave besteht aus 12 Halbtönen.}
 Die Hauptintervalle und die Ganzzahlen, die die Frequenzverhältnisse dieser Intervalle repräsentieren, sind jeweils oberhalb und unterhalb der chromatischen Tonleiter aufgeführt.
 Das Wort Intervall wird hierbei im Sinne von zwei Noten benutzt, deren Frequenzverhältnis der Quotient kleiner ganzer Zahlen ist.
 Außer für Vielfache dieser Grundintervalle erzeugen Ganzzahlen, die größer als ungefähr 10 sind, Intervalle, die für das Ohr nicht einfach zu erkennen sind.
 Gemäß Tabelle 2.2a ist das grundlegendste Intervall die Oktave, bei der die Frequenz der höheren Note das Doppelte der Frequenz der tieferen Note ist.
 Das Intervall zwischen C und G ist eine Quinte, und die Frequenzen von C und G stehen in einem Verhältnis von 2 zu 3 zueinander.
 Die große Terz hat vier Halbtöne, und die kleine Terz hat drei.
 \textbf{Die Zahl, die jedem Intervall zugeordnet ist, z.B. vier in der Quarte, ist bei der C-Dur-Tonleiter die Zahl der weißen Tasten inkl. der beiden Tasten am Anfang und Ende des Intervalls und hat keine weitere mathematische Bedeutung.}
 Beachten Sie, daß das Wort \enquote{Tonleiter} bzw. \enquote{Skala} in \enquote{chromatische Tonleiter}, \enquote{C-Dur-Tonleiter} und \enquote{logarithmische oder Frequenz-Skala} (s.u.) eine völlig unterschiedliche Bedeutung hat; die zweite ist eine Untermenge der ersten.
 
\begin{tabular}{lllllll}
\textbf{Oktave} & \textbf{Quinte} & \textbf{Quarte} & \textbf{Gr. Terz} & \textbf{Kl. Terz} & \nolinebreak & \nolinebreak \\ 
CDEFGAH & C D E F & G A H & C \# D \# & E F \# & G \# A \# H & C \\ 
1 & 2 & 3 & 4 & 5 & 6 & 8
\end{tabular}

\subparagraph{(Tabelle 2.2a: Frequenzverhältnisse der Intervalle in der chromatischen Tonleiter)}

Wir können oben sehen, daß eine Quarte und eine Quinte sich zu einer Oktave \enquote{aufaddieren} und eine große Terz und eine kleine Terz sich zu einer Quinte \enquote{aufaddieren}.
 Beachten Sie, daß diese Addition im logarithmischen Raum erfolgt, wie unten erklärt wird.
 Die fehlende Ganzzahl 7 wird ebenfalls unten erklärt.
 \hypertarget{et1}{}

\textbf{Die \hyperlink{et}{gleichschwebend temperierte} (ET) chromatische Tonleiter besteht aus \enquote{gleichen} Halbtonschritten für jede nachfolgende Note.}
 Sie sind in dem Sinne gleich, daß das Verhältnis der Frequenzen von zwei aufeinanderfolgenden Noten immer das gleiche ist.
 Diese Eigenschaft stellt sicher, daß jede Note (außer in der Tonhöhe) mit allen anderen identisch ist.
 Diese Gleichförmigkeit der Noten gestattet es dem Komponisten oder Künstler, jede Tonhöhe und jede Tonart zu benutzen, ohne auf große Dissonanzen zu treffen, wie unten weiter erklärt wird.
 In einer Oktave einer ET-Tonleiter gibt es 12 gleiche Halbtöne und jede Oktave ist ein genauer Faktor von 2 im Frequenzverhältnis.
 Deshalb beträgt die Frequenzänderung für jeden Halbton:
 \hypertarget{gleich21}{}

\begin{equation}
 Halbtonschritt^{12}=2\;oder\;Halbtonschritt=2^{1/12} \approx 1,05946
\end{equation}

%\begin{tabular}
%&#9;Halbtonschritt$^12$&#9;= 2&#9;oder &#9;Halbtonschritt&#9;= 2$^1/12$ $\approx$ 1,05946
%\end{tabular}

\subparagraph{(Gleichung 2.1)}

Gleichung 2.1 definiert die ET chromatische Tonleiter und erlaubt die Berechnung der Frequenzverhältnisse von \enquote{Intervallen} in dieser Tonleiter.
 Wie verhalten sich die \enquote{Intervalle} bei ET zu den Frequenzverhältnissen der reinen Intervalle?
 \textbf{Der Vergleich ist in Tabelle 2.2b aufgeführt und zeigt, daß die Intervalle der ET-Tonleiter den reinen Intervallen sehr nah kommen.}
\begin{tabular}{llll}
\textbf{Intervall} & \textbf{Freq.-Verh.} & \textbf{ET-Tonleiter} & \textbf{Differenz} \\ 
Kleine Terz & 6/5 = 1,2000 & Halbtonschritt$^3$ $\approx$ 1,1892 & +0,0108 \\ 
Große Terz & 5/4 = 1,2500 & Halbtonschritt$^4$ $\approx$ 1,2599 & -0,0099 \\ 
Quarte & 4/3 $\approx$ 1,3333 & Halbtonschritt$^5$ $\approx$ 1,3348 & -0,0015 \\ 
Quinte & 3/2 = 1,5000 & Halbtonschritt$^7$ $\approx$ 1,4983 & +0,0017 \\ 
Oktave & 2/1 = 2,0000 & Halbtonschritt$^12$ = 2,0000 & 0,0000
\end{tabular}

\subparagraph{(Tabelle 2.2b: Vergleich der reinen Intervalle mit der gleichschwebend temperierten Tonleiter)}

\textbf{Die Abweichung ist bei den Terzen am größten, mehr als fünfmal so groß wie die Abweichung bei den anderen Intervallen aber trotzdem nur ungefähr 1\%.}
 Nichtsdestoweniger sind diese Abweichungen leicht zu hören, und einige Klavierliebhaber haben sie großmütig als \enquote{die rollenden Terzen} tituliert, während sie in Wahrheit inakzeptable Dissonanzen sind.
 Es ist ein Mangel, mit dem wir leben müssen, wenn wir die ET-Tonleiter akzeptieren wollen.
 Die Abweichungen bei den Quarten und Quinten erzeugen um das mittlere C Schwebungen von ungefähr 1 Hz, was bei den meisten Musikstücken kaum zu hören ist; diese Schwebungsfrequenz verdoppelt sich jedoch mit jeder höheren Oktave.
 

Wäre die Ganzzahl 7 in Tabelle 2.2a aufgenommen worden, hätte sie ein Intervall mit dem Verhältnis 7/6 repräsentiert und würde dem Quadrat eines Halbtonschritts entsprechen.
 Die Abweichung zwischen diesen beiden Zahlen beträgt mehr als 4\% und ist zu groß, um ein musikalisch akzeptables Intervall zu bilden; sie wurde deshalb nicht in Tabelle 2.2a aufgeführt.
 Es ist nur ein mathematischer Zufall, daß die aus 12 Tönen bestehende chromatische Tonleiter so viele Verhältnisse nahe an den reinen Intervallen erzeugt.
 \textbf{Von den 8 kleinsten Ganzzahlen führt nur die Zahl 7 zu einem völlig inakzeptablen Intervall.
 Die chromatische Tonleiter basiert auf einem glücklichen mathematischen Zufall der Natur!
 Sie wird durch die kleinste Anzahl von Noten gebildet, die die maximale Anzahl von Intervallen ergeben.}
 Kein Wunder, daß frühe Zivilisationen glaubten, es läge etwas mystisches in dieser Tonleiter.
 Die Zahl der Noten in einer Oktave zu erhöhen, führt zu keiner großen Verbesserung der Intervalle, bis die Zahlen sehr groß werden, was diesen Ansatz für die meisten Musikinstrumente undurchführbar macht.
 

Beachten Sie, daß die Frequenzverhältnisse der Quarten und Quinten sich nicht zu dem der Oktave aufaddieren (1,5000 + 1,3333\nolinebreak=\nolinebreak2,8333 statt 2,0000).
 Sie addieren sich allerdings im logarithmischen Maßstab, weil (3/2)x(4/3)\nolinebreak=\nolinebreak2.
 Im logarithmischen Raum wird die Multiplikation zur Addition.
 Warum ist das so wichtig?
 Weil die Geometrie der Cochlea (Ohrschnecke) anscheinend eine logarithmische Komponente hat.
 Akustische Frequenzen auf einer logarithmischen Skala wahrzunehmen erreicht zwei Dinge: man kann bei gegebener Größe der Cochlea einen breiteren Frequenzbereich hören, und das Analysieren der Frequenzverhältnisse wird einfach, weil man, anstatt die zwei Frequenzen zu dividieren oder multiplizieren, nur ihre Logarithmen subtrahieren oder addieren muß.
 Wenn z.B. das C3 von der Cochlea an einer Stelle erkannt wird und das C4 an einer anderen Stelle, die 2 mm weiter aufwärts liegt, dann wird das C5 an einer Stelle erkannt, die 4 mm aufwärts liegt, genau wie bei einem Rechenschieber.
 Um zu zeigen, wie nützlich das ist: bei einem gegebenen F5 weiß das Gehirn, daß das F4 2 mm weiter unten zu finden ist!
 Deshalb sind Intervalle (erinnern Sie sich daran, daß Intervalle Divisionen von Frequenzen sind) von einer logarithmisch aufgebauten Cochlea besonders einfach zu analysieren.
 Wenn wir Intervalle spielen, üben wir mathematische Operationen im logarithmischen Raum auf einem mechanischen Computer genannt Klavier aus, ähnlich wie es früher mit dem Rechenschieber getan wurde.
 \textbf{Deshalb hat die logarithmische Natur der chromatischen Tonleiter viel mehr Konsequenzen, als nur einen größeren hörbaren Frequenzbereich zur Verfügung zu stellen.}
 Die logarithmische Skala stellt sicher, daß die beiden Noten jedes Intervalls, unabhängig davon wo man sich auf dem Klavier befindet, immer denselben Abstand voneinander haben.
 Durch die Übernahme einer logarithmischen Skala wird die Tastatur mechanisch auf das menschliche Ohr abgebildet!
 Das ist wahrscheinlich der Grund, warum Harmonien für das Ohr so angenehm sind - Harmonien werden durch das menschliche Gehör am leichtesten entschlüsselt und erinnert.
 

Angenommen, wir würden \hyperlink{gleich21}{Gleichung 2.1} nicht kennen; können wir die ET chromatische Tonleiter aus den Beziehungen der Intervalle erzeugen?
 Wenn die Antwort ja ist, kann ein Klavierstimmer ein Klavier stimmen, ohne Berechnungen durchführen zu müssen.
 Diese Intervallbeziehungen, so stellt sich heraus, bestimmen die Frequenzen aller Noten der zwölfnotigen chromatischen Tonleiter.
 Eine Temperatur ist eine Gruppe von Intervallbeziehungen, die diese Bestimmung ermöglicht.
 Von einem musikalischen Standpunkt aus gibt es keine chromatische Tonleiter, die besser wäre als alle anderen, obwohl ET die einmalige Eigenschaft hat, daß sie ein freies Transponieren erlaubt.
 Unnötig zu sagen, daß \textbf{ET nicht die einzige musikalisch nützliche Temperatur ist, und wir werden unten weitere Temperaturen besprechen.}
 Die Temperatur ist keine Option, sondern eine Notwendigkeit; wir \textit{müssen} eine Temperatur wählen, um diese mathematischen Schwierigkeiten zu überwinden.
 \textbf{Kein musikalisches Instrument, das auf der chromatischen Tonleiter basiert, ist völlig frei von Temperatur.}
 So müssen z.B. die Löcher eines Blasinstruments und die Bünde der Gitarre für eine bestimmte temperierte Tonleiter in einem entsprechenden Abstand angeordnet sein.
 Die Geige ist ein ziemlich cleveres Instrument, weil es alle Probleme mit der Temperatur dadurch vermeidet, daß die leeren Saiten zueinander einen Abstand von einer Quinte haben.
 Wenn man die A(440)-Saite richtig stimmt und alle anderen Saiten dazu in Quinten, dann sind die anderen rein und nicht temperiert.
 Man kann Probleme mit der Temperatur auch vermeiden, indem man alle Noten mit Ausnahme des A(440) greift.
 Außerdem ist das Vibrato größer als die Korrekturen der Temperatur, was die Differenzen der Temperatur unhörbar werden läßt.
 

\textbf{Die Erfordernis für das Temperieren entsteht, weil eine chromatische Tonleiter, die auf eine Tonart gestimmt ist (z.B. C-Dur mit reinen Intervallen), in anderen Tonarten keine akzeptablen Intervalle erzeugt.}
 Wenn man eine Komposition in C-Dur, die viele reine Intervalle enthält, transponiert, kann das zu schrecklichen Dissonanzen führen.
 Es gibt ein noch grundlegenderes Problem.
 Reine Intervalle in einer Tonart erzeugen auch Dissonanzen in anderen Tonarten, die im selben Musikstück benutzt werden.
 Die Temperierschemata wurden deshalb dafür entwickelt, diese Dissonanzen zu minimieren, indem man die Verstimmung der reinen Intervalle bei den wichtigsten Intervallen minimierte und die größten Dissonanzen in die weniger benutzten Intervalle verschob.
 Die zum schlimmsten Intervall gehörende Dissonanz wurde \enquote{Wolfsquinte} genannt.
 

Das Hauptproblem ist natürlich die Intervallreinheit; die obige Diskussion macht klar, daß egal was man tut, irgendwo eine Dissonanz auftreten wird.
 \textbf{Es mag für manche ein Schock sein, daß das Klavier im Grunde ein unvollkommenes Instrument ist!}
 Wir werden deshalb in jeder Tonleiter immer mit einigen Kompromissen bei den Intervallen leben müssen.
 

Der Name \enquote{chromatische Tonleiter} wird im allgemeinen auf jede zwölfnotige Tonleiter mit beliebiger Temperatur angewandt.
 Natürlich erlaubt die chromatische Tonleiter des Klaviers nicht die Benutzung von Frequenzen zwischen den Noten (wie man das bei der Geige tun kann), so daß es eine unendliche Zahl fehlender Noten gibt.
 In diesem Sinne ist die chromatische Tonleiter unvollständig.
 Nichtsdestoweniger ist die zwölfnotige Tonleiter für die meisten musikalischen Anwendungen genügend vollständig.
 Die Situation ist einer digitalen Fotografie analog.
 Wenn die Auflösung ausreichend ist, kann man den Unterschied zwischen einem digitalen Foto und einem analogen Foto mit viel höherer Informationsdichte nicht sehen.
 Ähnlich \textbf{hat die zwölfnotige Tonleiter offenbar für eine genügend große Anzahl musikalischer Anwendungen eine ausreichende Auflösung in der Tonhöhe.}
 Diese zwölfnotige Tonleiter ist für ein bestimmtes Instrument oder musikalisches Notationssystem mit begrenzter Zahl zur Verfügung stehender Noten ein guter Kompromiß zwischen \enquote{mehr Noten je Oktave für eine größere Vollständigkeit haben} und \enquote{genug Frequenzbereich haben, um den Bereich des menschlichen Gehörs abzudecken}.
 

Es gibt eine fruchtbare Debatte darüber, welche Temperatur musikalisch gesehen am besten ist.
 ET war von der frühesten Geschichte des Temperierens an bekannt.
 Es hat definitiv Vorteile, auf eine Temperatur zu standardisieren, aber das ist hinsichtlich der Unterschiedlichkeit der Meinungen über Musik und der Tatsache, daß viel der zur Zeit existierenden Musik mit dem Gedanken an eine bestimmte Temperatur geschrieben wurde, wahrscheinlich nicht möglich oder sogar nicht wünschenswert.
 Deshalb werden wir nun die verschiedenen Temperaturen erforschen.
 \hypertarget{c2_2c}{}\hypertarget{c2_2_temp}{}

\subsection{Temperatur und Musik}

Der obige \hyperlink{c2_2b}{mathematische Ansatz} ist nicht die Art und Weise, in der die chromatische Tonleiter entwickelt wurde.
 Musiker begannen zunächst mit Intervallen und versuchten, eine Tonleiter mit einer minimalen Anzahl Noten zu finden, die diese Intervalle erzeugen würde.
 Die Erfordernis einer minimalen Anzahl von Noten ist offensichtlich wünschenswert, weil diese die Anzahl der Tasten, Saiten, Löcher, usw. bestimmt, die für die Konstruktion eines Musikinstruments notwendig sind.
 Intervalle sind notwendig, denn wenn man mehr als eine Note gleichzeitig spielen möchte, werden diese Noten für das Ohr unangenehme Dissonanzen erzeugen, außer wenn sie harmonische Intervalle bilden.
 Der Grund, warum Dissonanzen so unangenehm für das Ohr sind, hat eventuell etwas mit der Schwierigkeit zu tun, mit dem Gehirn dissonante Informationen zu verarbeiten.
 Es ist sicherlich hinsichtlich des Gedächtnisses und Verständnisses leichter, sich mit harmonischen Intervallen als mit Dissonanzen zu befassen, wobei es bei einigen davon für die meisten Menschen fast unmöglich ist, herauszufinden, ob zwei dissonante Noten gleichzeitig gespielt werden.
 Deshalb wird es, wenn das Gehirn mit der Aufgabe komplexe Dissonanzen zu erkennen überlastet ist, unmöglich zu entspannen und die Musik zu genießen oder die musikalische Idee zu verfolgen.
 Sicherlich muß jede Tonleiter gute Intervalle erzeugen, wenn wir fortgeschrittene, komplexe Musik komponieren sollen, die mehr als eine Note gleichzeitig erfordert.
 

\textbf{Die optimale Anzahl Noten in einer Tonleiter stellte sich als 12 heraus.
 Leider gibt es keine zwölfnotige Tonleiter, die überall reine Intervalle erzeugt.
 Musik würde besser klingen, wenn eine Tonleiter, die nur aus reinen Intervallen besteht, gefunden werden könnte.}
 Viele solcher Versuche wurden bereits unternommen, hauptsächlich durch das Erhöhen der Notenanzahl je Oktave und besonders bei Gitarren und Orgeln, aber keine dieser Tonleitern hat eine Akzeptanz erreicht
 \footnote{zumindest in der \enquote{westlichen} Musik, in der Musik anderer Kulturen gibt es durchaus Tonleitern mit mehr als 20 Tönen je Oktave}.
 Es ist relativ leicht, die Zahl der Noten mit einem gitarrenähnlichen Instrument zu erhöhen, weil man nur Saiten und Bünde hinzufügen muß.
 Die neuesten Verfahren, die heute entwickelt werden, beziehen computergenerierte Tonleitern mit ein, bei denen der Computer die Frequenzen bei jeder Transposition justiert; dieses Verfahren wird adaptives Stimmen (Sethares) genannt.
 

\textbf{Das grundlegendste Konzept, das benötigt wird, um Temperaturen zu verstehen, ist das Konzept des Quintenzirkels.}
 Nehmen Sie, um einen Quintenzirkel zu beschreiben, eine beliebige Oktave.
 Beginnen Sie mit der tiefsten Note, und gehen Sie in Quinten aufwärts.
 Nach zwei Quinten kommen Sie über diese Oktave hinaus.
 Wenn das geschieht, gehen Sie eine Oktave nach unten, so daß Sie weiter in Quinten aufwärts gehen können und immer noch in der ursprünglichen Oktave bleiben.
 Machen Sie das für zwölf Quinten, und Sie werden bei der höchsten Note der Oktave ankommen!
 D.h. wenn Sie mit C4 anfangen, kommen Sie am Ende zu C5, und deshalb wird es ein Zirkel \footnote{lat. circulus = Kreis} genannt.
 Nicht nur das, sondern jede Note, auf die Sie treffen, wenn Sie die Quinten spielen, ist eine andere Note.
 Das bedeutet, daß der Quintenzirkel jede Note trifft, und das nur einmal, was eine nützliche Schlüsseleigenschaft für das Stimmen der Tonleiter ist und dafür, sie mathematisch zu untersuchen.
 \hypertarget{c2_2_hist}{}

\textbf{Historische Entwicklungen sind ein zentrales Thema der Diskussionen über Temperaturen, weil die Musik aus einer Zeit mit der Temperatur aus dieser Zeit verbunden ist.
 Pythagoras wird zugeschrieben, daß er ungefähr 550 v. Chr. unter Benutzung des Quintenzirkels die \enquote{pythagoreische Stimmung} erfunden hat, bei der die chromatische Tonleiter durch das Stimmen mit reinen Quinten erzeugt wird.}
 Die zwölf reinen Quinten im Quintenzirkel bilden keinen exakten Faktor 2.
 Deshalb ist die letzte Note, die man bekommt, nicht genau die Oktavnote, sondern ist in der Frequenz um den Wert zu hoch, den man \enquote{pythagoreisches Komma} nennt, d.h. ungefähr 23 Cent (ein Cent ist ein Hundertstel eines Halbtonschritts).
 Da eine Quarte und eine Quinte eine Oktave bilden, resultiert die pythagoreische Stimmung in einer Tonleiter mit reinen Quarten und Quinten, wobei man allerdings am Ende eine sehr schlechte Dissonanz bekommt.
 Es stellt sich heraus, daß mit reinen Quinten zu stimmen, zu unreinen Terzen führt.
 Das ist ein weiterer Nachteil der pythagoreischen Stimmung.
 Wenn nun jemand stimmen sollte, indem er jede Quinte um 23/12 Cent zusammenzieht, dann hätte er am Ende genau eine Oktave, und das ist eine Möglichkeit, eine \hyperlink{et1}{ET-Tonleiter} zu stimmen.
 Wir werden im Abschnitt über das Stimmen tatsächlich \hyperlink{c2_6_et}{solch eine Methode benutzen}.
 Die ET-Tonleiter war schon ca. 100 Jahre nach der Erfindung der pythagoreischen Stimmung bekannt.
 Deshalb ist die ET keine \enquote{moderne Temperatur}.
 

Alle neueren Temperaturen, die auf die Einführung der pythagoreischen Stimmung folgten, waren Bemühungen, diese zu verbessern.
 Die erste Methode war, das pythagoreische Komma zu halbieren und auf die letzten beiden Quinten zu verteilen.
 \textbf{Eine wichtige Entwicklung war die mitteltönige Stimmung, bei der die Terzen statt der Quinten rein gemacht wurden.}
 Musikalisch spielen Terzen eine bedeutendere Rolle als die Quinten, so daß die mitteltönige Stimmung sinnvoll war, besonders in einer Zeit, in der die Musik mehr Gebrauch von den Terzen machte.
 Unglücklicherweise hat die mitteltönige Stimmung eine Wolfsquinte, die schlimmer als die der pythagoreischen Stimmung ist.
 \hypertarget{c2_2_wtk2}{}

Der nächste Meilenstein wird von Bachs \enquote{Das Wohltemperirte Clavier} markiert, in dem er Musik für verschiedene Wohltemperierte Stimmungen (WT) geschrieben hat.
 Das waren Temperaturen, die einen Kompromiß zwischen mitteltöniger und pythagoreischer Stimmung darstellten.
 Dieses Konzept funktionierte, weil die pythagoreische Stimmung zu Noten führt, die zu \textit{hoch} sind, während die mitteltönige zu Noten führt, die zu \textit{tief} sind.
 Außerdem boten die WT nicht nur die Möglichkeit guter Terzen, sondern auch von guten Quinten.
 \textbf{Die einfachste WT wurde von Kirnberger, einem Schüler Bachs, entworfen.
 Der größte Vorteil der Temperatur von Kirnberger ist ihre Einfachheit.
 Bessere WTs wurden von Werckmeister und von Young entwickelt.
 Wenn wir die Stimmungen allgemein in mitteltönig, WT und pythagoreisch einteilen, dann ist ET eine WT, weil ET weder erhöht noch erniedrigt ist.}
 Es gibt keine Aufzeichnungen darüber, welche Temperatur(en) Bach benutzte.
 Wir können die Temperatur(en) nur anhand der Harmonien in seinen Kompositionen vermuten, insbesondere seines \enquote{Wohltemperierten Klaviers}, und diese Studien zeigen, daß im Grunde alle Details des Temperierens bereits zu Bachs Zeiten (vor 1700) bekannt waren, und daß Bach eine Temperatur benutzte, die sich von der von Werckmeister nicht sehr unterschied.
 

Die Violine scheint einen Vorteil aus ihrem einzigartigen Aufbau zu ziehen, um diese Temperaturprobleme zu umgehen.
 Die leeren Saiten bilden miteinander Quintintervalle, so daß sie von Natur aus pythagoreisch gestimmt ist.
 Da die Terzen immer rein gespielt werden können, hat sie alle Vorteile der pythagoreischen, mitteltönigen und WT-Stimmung, und weit und breit ist keine Wolfsquinte in Sicht!
 

In den letzten ca. 100 Jahren wurde ET fast überall akzeptiert.
 Deshalb werden die anderen Temperaturen im allgemeinen als \enquote{historische Temperaturen} eingestuft, was klar ein falsche Bezeichnung ist.
 Der historische Gebrauch der WT führte zu dem Konzept der Tonartfarbe, bei dem jede Tonart in Abhängigkeit von der Stimmung der Musik besondere Farbe verlieh, und zwar hauptsächlich durch die kleinen Verstimmungen, die \enquote{Spannung} und andere Effekte erzeugen.
 Das komplizierte die Lage sehr, weil die Musiker sich nun nicht nur mit reinen Intervallen und Wolfsquinten befassen mußten, sondern auch mit Farben, die nicht so leicht zu definieren waren.
 Das Ausmaß, in dem die Farben herausgebracht werden können, hängt vom Klavier, dem Pianisten und dem Zuhörer genauso ab wie vom Stimmer.
 Beachten Sie, daß der Stimmer die Streckung (s. \enquote{\hyperlink{c2_5_stre}{Was ist Streckung?}} am Ende von Abschnitt 5) mit der Temperatur verbinden kann, um die Farbe zu kontrollieren.
 Nachdem man Musik gehört hat, die auf einem Klavier gespielt wird, das WT gestimmt ist, klingt ET eher trüb und farblos.
 Deshalb ist die Farbe der Tonart wichtig.
 Wichtiger sind die wundervollen Klänge von reinen (gestreckten) Intervallen bei WT.
 Auf der anderen Seite gibt es in den WTs immer eine Art von Wolfsquinte, die bei der ET reduziert ist.
 

Zum Spielen der meisten Musik, die um die Zeit von Bach, Mozart und Beethoven komponiert wurde, ist WT am besten geeignet.
 So hat Beethoven z.B. für die dissonanten Nonen im ersten Satz seiner Mondschein-Sonate Akkorde gewählt, die in WT am wenigsten dissonant und in ET viel schlechter sind.
 Diese großen Komponisten waren sich der Temperatur genauestens bewußt.
 Die meisten Werke aus der Zeit von Chopin oder Liszt wurden im Hinblick auf ET komponiert, so daß die Tonartfarbe kein Thema ist.
 Obwohl diese Kompositionen für das geschulte Ohr in ET und WT unterschiedlich klingen, ist nicht klar, daß gegen WT etwas einzuwenden ist, weil reine Intervalle immer besser klingen als verstimmte.
 

Meine persönliche Ansicht hinsichtlich des Klaviers ist, daß wir von ET abkommen sollten, weil sie uns eines der angenehmsten Aspekte der Musik beraubt - reinen Intervallen.
 Sie werden eine dramatische Demonstration davon erleben, wenn Sie den letzten Satz von Beethovens Waldstein-Sonate in ET und WT hören.
 Mitteltönige Stimmung kann ziemlich extrem sein, es sei denn Sie spielen Musik dieser Periode (vor Bach), so daß uns die WTs bleiben.
 Hinsichtlich der Einfachheit und der leichten Stimmbarkeit ist Kirnberger nicht zu schlagen.
 Ich glaube, daß wenn Sie sich an WT gewöhnt haben, ET nicht genauso gut klingen wird.
 Deshalb sollte die Welt die WTs zum Standard erheben.
 Welche man auswählt, macht für die meisten Menschen keinen großen Unterschied, weil diejenigen, die nicht in den Temperaturen ausgebildet sind, im allgemeinen keinen großen Unterschied zwischen den hauptsächlichen Temperaturen bemerken, geschweige denn zwischen den unterschiedlichen WTs.
 Das soll nicht heißen, daß wir alle Kirnberger benutzen sollten, sondern daß wir in den Temperaturen ausgebildet werden und eine Wahl haben sollten, anstatt in die Zwangsjacke der farblosen ET gesteckt zu werden.
 Das ist nicht nur eine Frage des Geschmacks oder die Frage, ob die Musik besser klingt.
 Wir sprechen darüber, unsere musikalische Sensibilität zu entwickeln und zu wissen, wie man diese wirklich reinen Intervalle benutzt.
 Ein Nachteil von WT ist, daß es hörbar wird, wenn das Klavier auch nur ein wenig verstimmt ist.
 Es würde mich jedoch freuen, wenn alle Klavierschüler ihre Sensibilität bis zu dem Punkt entwickeln würden, an dem sie bereits erkennen können, wenn das Klavier auch nur ein wenig verstimmt ist.
 
%  c23.html 
\hypertarget{c2_3}{}

\section{Werkzeuge zum Stimmen}

\textbf{Sie werden einen Stimmhammer, mehrere Gummikeile, einen Filzstreifen zum Dämpfen, eine oder zwei Stimmgabeln und Ohrstöpsel oder Ohrenschützer benötigen.}
 Professionelle Stimmer benutzen heutzutage auch eine elektronische Stimmhilfe; wir werden diese aber hier nicht berücksichtigen, weil sie für den Amateur nicht rentabel ist und ihre richtige Anwendung ein fortgeschrittenes Wissen über die Feinheiten des Stimmens erfordert.
 Die Stimmethode, die wir hier behandeln, wird aurales Stimmen genannt - Stimmen nach Gehör.
 Alle guten professionellen Stimmer müssen gute aurale Stimmer sein, auch wenn sie oft elektronische Stimmhilfen verwenden.
 

Für Flügel brauchen Sie größere Gummikeile zum Dämpfen, während \enquote{\hyperlink{upright}{Aufrechte}} kleinere mit Metallgriffen erfordern.
 Vier Keile jeden Typs werden ausreichen.
 Sie können diese per Versand bestellen oder Ihren Stimmer bitten, den ganzen Satz Werkzeuge, den Sie benötigen, für Sie zu kaufen.
 

Die verbreitetsten Dämpfungsstreifen sind aus Filz, ungefähr 4ft lang und 5/8" breit \textit{[ca. 1,22m x 1,6cm]}.
 Sie werden benutzt, um die 2 Nebensaiten der 3-saitigen Noten in der Oktave zu dämpfen, die für das \enquote{\hyperlink{c2_4}{Einstellen des Bezugspunkts}} benutzt wird (s.u.).
 Es gibt die Streifen auch als verbundene Gummikeile, aber diese funktionieren nicht genauso gut.
 Die Streifen gibt es auch in Gummi, aber Gummi dämpft nicht so gut und ist nicht so stabil wie Filz (die Streifen können sich während des Stimmens verschieben oder herausspringen).
 Der Nachteil der Filzstreifen ist, daß sie auf dem Resonanzboden eine Filzfaserschicht hinterlassen, die abgesaugt werden muß.
 

Ein Stimmhammer hoher Qualität besteht aus einem verlängerbaren Griff, einem an der Spitze des Griffs befestigten Kopf und einem auswechselbaren, in den Kopf geschraubten Einsatz.
 Es ist eine gute Idee, einen Stimmwirbel zu haben, den Sie in den Einsatz stecken können, so daß Sie den Einsatz mit Hilfe eines Schraubstocks fest in den Kopf schrauben können.
 Ansonsten könnten Sie den Einsatz verkratzen, wenn Sie ihn mit dem Schraubstock fassen.
 Wenn der Einsatz nicht fest im Kopf sitzt, wird er sich während des Stimmens lösen.
 Die meisten Klaviere erfordern einen Einsatz \#2, es sei denn, das Klavier wurde mit größeren Stimmwirbeln neu besaitet.
 Der Standardkopf ist ein 5-Grad-Kopf.
 Diese \enquote{5 Grad} sind der Winkel zwischen der Einsatzachse und dem Griff.
 Sowohl die Köpfe als auch die Einsätze gibt es in verschiedenen Längen, aber \enquote{Standard-} oder \enquote{mittlere} Länge wird genügen.
 \hypertarget{c2_3_gabel}{}

Besorgen Sie sich zwei Stimmgabeln - A440 und C523,3 - von guter Qualität.
 Entwickeln Sie die gute Angewohnheit, sie am schmalen Hals des Griffs zu halten, so daß Ihre Finger nicht die Schwingungen der Stimmgabeln stören.
 Klopfen Sie die Spitze der Gabel fest gegen einen muskulösen Teil Ihres Knies und testen Sie die Aushaltezeit (Sustain).
 Sie sollte für 10 bis 20 Sekunden deutlich zu hören sein, wenn Sie sie nahe an Ihr Ohr halten.
 Die beste Art, die Gabel zu hören, ist, die Spitze des Griffs auf den dreieckigen Knorpel (Tragus) zu setzen, der zur Mitte des Ohrlochs hin herausragt.
 Sie können die Lautstärke der Gabel anpassen, indem Sie den Tragus mit dem Ende der Gabel ein- oder auswärts drücken.
 Benutzen Sie keine Pfeifen; diese sind zu ungenau.
 

Ohrenschützer sind eine notwendige Schutzvorrichtung, da Gehörschäden das Berufsrisiko eines Stimmers sind.
 Wie weiter unten erklärt wird, ist es notwendig \hyperlink{c2_5_infi}{die Tasten hart anzuschlagen} (auf die Tasten zu hämmern - um den Jargon der Stimmer zu benutzen), um richtig zu stimmen, und die Klangintensität eines solchen Hämmerns kann das Ohr sehr leicht schädigen, was zu Gehörverlust und Tinnitus führt.
 \hypertarget{c2_4}{}

\section{Vorbereitung}

Bereiten Sie das Stimmen vor, indem Sie den Notenständer entfernen, so daß sie an die Stimmwirbel herankommen (Flügel).
 Für den folgenden Abschnitt brauchen Sie keine weiteren Vorbereitungen.
 Um \enquote{die Bezugspunkte einzustellen} müssen Sie alle Nebensaiten der 3-fachen Saiten in der \enquote{Bezugsoktave} mit den Dämpfungsstreifen dämpfen, so daß, wenn Sie eine Note in dieser Oktave spielen, nur die mittlere Saite vibriert.
 Sie werden wahrscheinlich je nach Stimmalgorithmus fast zwei Oktaven dämpfen müssen.
 Probieren Sie zunächst den ganzen Stimmalgorithmus aus, um die höchste und die tiefste Note zu bestimmen, die Sie dämpfen müssen.
 Dämpfen Sie dann alle Noten dazwischen.
 Benutzen Sie das gerundete Ende des Drahtgriffs eines Dämpfungskeils für \enquote{Aufrechte}, um den Filz in den Raum zwischen den äußeren Saiten zweier nebeneinander liegenden Noten zu pressen.
 
%  c25.html 
\hypertarget{c2_5}{}

\section{Wie man anfängt}\hypertarget{c2_5a}{}

\subsection{Einleitung}

\textbf{Ohne einen Lehrer können Sie nicht einfach mit dem Stimmen anfangen.}
 Sie werden schnell Ihren Bezugspunkt verlieren und keine Ahnung haben, wie Sie wieder zurückkommen.
 {\normalsize \textbf{Deshalb müssen Sie zunächst bestimmte Stimmverfahren lernen und üben, damit Sie am Ende nicht mit einem unspielbaren Klavier dastehen, das Sie nicht wiederherstellen können.}}
 Dieser Abschnitt ist ein Versuch, Sie auf die Stufe zu bringen, bei der Sie ein richtiges Stimmen versuchen können, ohne auf Schwierigkeiten dieser Art zu stoßen.
 

{\normalsize \textbf{Als erstes müssen Sie lernen, was man tun kann und was nicht, um zu vermeiden, daß Sie das Klavier zerstören, was leicht geschehen kann.
 Wenn man eine Saite zu stark spannt, dann bricht sie.
 \textit{[Verletzungsgefahr!]}
 Die anfänglichen Anweisungen sind dafür gedacht, Saitenbrüche aufgrund von amateurhaftem Vorgehen zu minimieren, lesen Sie sie deshalb sorgfältig.}}
 Sie müssen im voraus planen, was Sie tun, wenn eine Saite bricht.
 Eine gebrochene Saite ist, auch wenn Sie über längere Zeit nicht ersetzt wird, für sich genommen keine Katastrophe für das Klavier.
 Es ist jedoch wahrscheinlich klug, die ersten Übungen zu machen, kurz bevor man die Absicht hat, seinen Stimmer zu sich zu bitten.
 Wenn Sie erst wissen wie man stimmt, ist ein Saitenbruch - außer bei sehr alten oder mißhandelten Klavieren -  ein seltenes Problem.
 Die Stimmwirbel werden während des Stimmens um solch kleine Beträge gedreht, daß die Saiten fast nie brechen.
 Ein verbreiteter Fehler, der von Anfängern begangen wird, ist, den Stimmhammer am falschen Stimmwirbel anzusetzen.
 Da das Drehen des Wirbels keine hörbare Veränderung bewirkt, wird dann weitergedreht, bis die Saite bricht.
 Eine Möglichkeit, das zu vermeiden, ist, {\normalsize \textbf{immer damit zu beginnen, \textit{tiefer} zu stimmen, wie es unten empfohlen wird, und niemals den Wirbel zu drehen, ohne den Ton anzuhören.}}

{\normalsize \textbf{Die wichtigste Aufgabe für einen beginnenden Stimmer ist, den Zustand des Stimmstocks zu bewahren.}}
 Der Druck des Stimmstocks auf die Wirbel ist enorm.
 Sie dürfen das natürlich niemals tun, aber angenommen, Sie würden den Wirbel sehr schnell um 180 Grad drehen, wäre die dabei an der Fläche zwischen Wirbel und Stimmstock erzeugte Hitze ausreichend, um das Holz zu verbrennen und seine molekulare Struktur zu verändern.
 Es ist klar, daß alle Drehungen des Wirbels in langsamen, kleinen Schritten ausgeführt werden müssen.
 Wenn Sie den Wirbel durch Drehen entfernen müssen, drehen Sie ihn nur eine viertel Drehung (gegen den Uhrzeigersinn), warten Sie einen Moment, bis sich die Hitze von der Grenzfläche weg verteilt hat, wiederholen Sie dann den Vorgang, usw., um eine Beschädigung des Stimmstock zu vermeiden.
 

{\normalsize \textbf{Ich werde alles am Beispiel des Flügels erklären, die entsprechenden Bewegungen für \enquote{\hyperlink{upright}{Aufrechte}} sollten aber offensichtlich sein.}}\textbf{Es gibt beim Stimmen zwei grundlegende Bewegungen.
 Die erste ist die Drehung des Wirbels, so daß die Saite entweder angezogen oder entspannt wird.
 \footnote{Mit \enquote{wiegen} ist im folgenden kein wildes Hin- und Herschaukeln des Stimmwirbels gemeint, sondern jeweils das behutsame Ziehen von der Saite weg bzw. behutsame Nachgeben zur Saite hin!}
 Die zweite ist, den Wirbel rückwärts zu Ihnen hin zu wiegen (um an der Saite zu ziehen) oder ihn vorwärts zu wiegen, in Richtung der Saite, um sie nachzulassen.}
 Wenn die wiegende Bewegung extrem ausgeführt wird, vergrößert sie das Loch und beschädigt den Stimmstock.
 Beachten Sie, daß das Loch an der Oberseite des Stimmstocks ein wenig elliptisch ist, weil die Saite den Wirbel in Richtung der Hauptachse der Ellipse zieht.
 Darum vergrößert ein kleines Wiegen rückwärts die Ellipse nicht, weil der Wirbel durch die Saite immer in das vordere Ende der Ellipse gezogen wird.
 Auch ist der Wirbel nicht gerade, sondern wird durch den Zug der Saite elastisch zur Saite hin gebogen.
 Deshalb kann die wiegende Bewegung für das Bewegen der Saite sehr effektiv sein.
 Sogar ein geringes Maß an Vorwärtswiegen, innerhalb der Elastizität des Holzes, ist unschädlich.
 Anhand dieser Überlegungen wird deutlich, daß \textbf{Sie die Drehung benutzen müssen, wann immer sie möglich ist, und die wiegende Bewegung nur, wenn sie absolut notwendig ist}.
 Nur sehr kleine wiegende Bewegungen sollten angewandt werden.
 Bei den höchsten Noten (die zwei obersten Oktaven), ist die für das Stimmen der Saite notwendige Bewegung so gering, daß Sie eventuell nicht in der Lage sind, sie angemessen durch das Drehen des Wirbels zu kontrollieren.
 Das Wiegen bietet eine viel feinere Kontrolle und kann für diese \hyperlink{c2_5_infi}{abschließende, winzige Bewegung} benutzt werden, um die Saite in perfekte Stimmung zu bringen.
 

Was ist nun der einfachste Weg, mit dem Üben zu beginnen?
 Lassen Sie uns zunächst die am einfachsten zu stimmenden Noten auswählen.
 Diese liegen in der \hyperlink{Noten}{C3-C4}-Oktave.
 Tiefere Noten sind wegen ihres hohen harmonischen Gehalts schwieriger zu stimmen, und die höheren Noten sind schwierig, weil das für das Stimmen notwendige Maß der Wirbeldrehung mit steigender Tonhöhe abnimmt.
 Beachten Sie, daß C4 für das mittlere C steht; das H direkt darunter ist H3, und das D direkt über dem mittleren C ist D4.
 Die Oktavnummer 1, 2, 3, . . . ändert sich somit beim C, nicht beim A.
 Wählen wir das G3 als unsere Übungsnote, und fangen wir mit dem Numerieren der Saiten an.
 Jede Note in diesem Bereich hat 3 Saiten.
 Von der linken Seite beginnend, numerieren wir die Saiten 123 (für G3), 456 (für G\#3), 789 (für A3), usw.
 Fügen Sie zwischen den Saiten 3 und 4 einen Keil ein, um die Saite 3 zu dämpfen, so daß nur 1 und 2 schwingen können, wenn Sie G3 spielen.
 Plazieren Sie den Keil ungefähr in der Mitte zwischen Steg und Agraffe.
 

\textbf{Es gibt zwei grundlegende Arten zu stimmen: unisono und harmonisch.}
 Beim Unisono werden die beiden Saiten identisch gestimmt.
 Beim harmonischen Stimmen wird eine Saite harmonisch zur anderen gestimmt, z.B. im Abstand einer Terz, Quarte, Quinte oder Oktave.
 Die drei Saiten jeder Note unisono zu stimmen, ist einfacher, als harmonisch zu stimmen; lassen Sie uns also damit beginnen.
 \hypertarget{c2_5b}{}\hypertarget{c2_5_hamm}{}

\subsection{Einsetzen und Bewegen des Stimmhammers}

Wenn Ihr Stimmhammer eine justierbare Länge hat, ziehen Sie ihn ungefähr 3 Zoll \textit{[7,5cm]} heraus, und stellen Sie ihn fest.
 Halten Sie den Griff des Stimmhammers in Ihrer RH und den Einsatz mit Ihrer LH, und setzen Sie den Einsatz oben am Wirbel an.
 Richten Sie den Griff so aus, daß er ungefähr senkrecht zu den Saiten steht und nach rechts zeigt.
 Wackeln Sie mit Ihrer RH leicht mit dem Griff um den Stimmwirbel, und schieben Sie den Einsatz mit Ihrer LH über den Wirbel nach unten, so daß der Einsatz sicher so weit eingeschoben ist wie es geht.
 \textbf{Entwickeln Sie vom ersten Tag an die Angewohnheit, mit dem Einsatz zu wackeln, so daß er sicher eingeschoben ist.}
 An diesem Punkt ist der Griff wahrscheinlich nicht perfekt senkrecht zu den Saiten; wählen Sie einfach die Position des Einsatzes so, daß der Griff so gut wie möglich senkrecht steht.
 Finden Sie nun eine Position, in der Sie Ihre RH so abstützen, daß Sie einen festen Druck auf den Hammer ausüben können.
 Sie können z.B. die Spitze des Griffs mit dem Daumen und einem oder zwei Fingern greifen und den Arm auf dem hölzernen Klavierrahmen abstützen oder den kleinen Finger auf den Stimmwirbeln direkt unter dem Griff abstützen.
 Wenn der Griff näher an der Platte (dem Metallrahmen) über den Saiten ist, könnten Sie Ihre Hand an der Platte abstützen.
 Sie sollten den Griff nicht so greifen, wie Sie einen Tennisschläger halten, und ziehen bzw. drücken, um den Wirbel zu drehen - das gibt Ihnen nicht genügend Kontrolle.
 Sie werden vielleicht nach mehreren Jahren Übung dazu in der Lage sein, aber am Anfang ist eine exakte Kontrolle zu schwierig, wenn man den Griff packt und drückt, ohne sich an etwas abzustützen.
 \textbf{Entwickeln Sie deshalb die Angewohnheit, je nach der Griffposition gute Stellen zum Abstützen zu finden.}
 Üben Sie diese Positionen und stellen Sie sicher, daß Sie einen kontrollierten, konstanten und kräftigen Druck auf den Griff ausüben können, aber drehen Sie noch keine Wirbel.
 

Der Hammergriff muß nach rechts zeigen, so daß Sie, wenn Sie ihn zu sich hin drehen (die Saite wird gespannt), gegen die Kraft der Saite arbeiten und den Wirbel aus der Vorderseite des Lochs (zur Saite hin) befreien.
 Das gestattet - wegen der reduzierten Reibung - dem Wirbel, sich freier zu drehen.
 Wenn Sie \textit{tiefer} stimmen, versuchen sowohl Sie als auch die Saite, den Wirbel in die gleiche Richtung zu drehen.
 Der Wirbel würde dabei zu leicht drehen, wenn nicht sowohl Ihr Druck als auch der Zug der Saite den Wirbel gegen die Vorderseite des Lochs drücken, somit den Druck (Reibung) erhöhen und es verhindern würden.
 Würden Sie den Griff nach links stellen, bekämen Sie sowohl bei der Bewegung zum Höher- als auch zum Tieferstimmen Probleme.
 Beim Höherstimmen drücken sowohl Sie als auch die Saite den Wirbel gegen die Vorderseite des Lochs, was es um so schwieriger macht, den Wirbel zu drehen, und das Loch beschädigt.
 Beim Tieferstimmen neigt der Hammer dazu, den Wirbel von der Vorderkante des Lochs abzuheben und reduziert die Reibung.
 Außerdem drehen sowohl der Hammer als auch die Saite den Wirbel in die gleiche Richtung.
 Jetzt dreht sich der Wirbel zu leicht.
 Der Hammergriff muß bei \enquote{Aufrechten} nach links zeigen.
 Wenn man von oben auf den Stimmwirbel schaut, sollte der Hammer bei Flügeln nach 3 Uhr und bei \enquote{Aufrechten} nach 9 Uhr zeigen.
 In beiden Fällen befindet sich der Hammer auf der Seite der letzten Windung der Saite.
 

Professionelle Stimmer benutzen diese Hammerpositionen nicht.
 Die meisten benutzen 1-2 Uhr für Flügel und 10-11 Uhr für \enquote{Aufrechte}, und Reblitz empfiehlt 6 Uhr für Flügel und 12 Uhr für \enquote{Aufrechte}.
 Um zu verstehen warum, betrachten wir zunächst das Einsetzen des Hammers bei einem Flügel bei 12 Uhr (bei 6 Uhr ist es ähnlich).
 Nun ist die Reibung des Wirbels mit dem Stimmstock beim Höher- und Tieferstimmen die gleiche.
 Beim Höherstimmen arbeiten Sie jedoch gegen die Saitenspannung und beim Tieferstimmen hilft Ihnen die Saite.
 Deshalb ist die Differenz der benötigten Kraft zwischen Höher- und Tieferstimmen viel größer als die Differenz ist, wenn der Hammer auf 3 Uhr steht, was ein Nachteil ist.
 Anders als bei der 3-Uhr-Position, wiegt der Wirbel während des Stimmens nicht vor und zurück, so daß, wenn Sie den Druck auf den Stimmhammer nachlassen, der Wirbel nicht zurückspringt - er ist stabiler - und Sie können eine höhere Genauigkeit erreichen.
 

Die 1-2-Uhr-Position ist ein guter Kompromiß, der sowohl die Vorteile der 3-Uhr-Position als auch der 12-Uhr-Position ausnutzt.
 Anfänger haben nicht die Genauigkeit, um den vollen Vorteil aus der 1-2-Uhr-Position zu ziehen; mein Vorschlag ist deshalb, mit der 3-Uhr-Position anzufangen, was zunächst einfacher sein sollte, und zur 1-2-Uhr-Position überzugehen, wenn Ihre Genauigkeit steigt.
 Wenn Sie gut werden, kann die höhere Genauigkeit der 1-2-Uhr-Position Ihr Stimmen beschleunigen, so daß Sie jede Saite in wenigen Sekunden stimmen können.
 Bei der 3-Uhr-Position werden Sie raten müssen wieviel der Wirbel zurückspringt und um diesen Betrag überstimmen müssen, was mehr Zeit benötigt.
 Klar wird es wichtiger, wo Sie den Hammer plazieren, sobald Sie besser werden.
 \hypertarget{c2_5c}{}\hypertarget{c2_5_wirb}{}

\subsection{Den Wirbel einstellen}

\textbf{Es ist wichtig, den Stimmwirbel richtig \enquote{einzustellen}, damit die Stimmung hält.}
 Wenn man von oben auf den Wirbel schaut, kommt die Saite von der rechten Seite des Wirbels (bei Flügeln - sie ist bei \enquote{Aufrechten} auf der linken Seite) und ist um ihn herumgewickelt.
 Deshalb stimmen Sie \textit{höher}, wenn Sie den Wirbel im Uhrzeigersinn drehen, und \textit{tiefer}, wenn Sie den Wirbel gegen den Uhrzeigersinn drehen.
 Die Saitenspannung versucht immer, den Wirbel gegen den Uhrzeigersinn zu drehen (oder \textit{tiefer}).
 Normalerweise verstimmt sich ein Klavier \textit{tiefer}, wenn man es spielt.
 Da der Stimmstock den Wirbel so stark umklammert, ist der Wirbel jedoch niemals gerade sondern verdreht.
 

Wenn man ihn im Uhrzeigersinn dreht und anhält, wird die Oberseite des Wirbels in bezug auf den Boden im Uhrzeigersinn verdreht.
 In dieser Position möchte die Oberseite des Wirbels gegen den Uhrzeigersinn drehen (der Wirbel möchte sich zurückdrehen), aber er kann nicht, weil er vom Stimmstock gehalten wird.
 Erinnern Sie sich daran, daß die Saite ebenfalls versucht, ihn gegen den Uhrzeigersinn zu drehen.
 Die beiden Kräfte zusammen können genügen, um das Klavier schnell \textit{tiefer} zu verstimmen, wenn man etwas laut spielt.
 

Wenn der Wirbel gegen den Uhrzeigersinn gedreht wird, geschieht das Gegenteil - der Wirbel will sich im Uhrzeigersinn zurückdrehen, was der Saitenkraft entgegenwirkt.
 Das reduziert das Nettodrehmoment am Wirbel, was die Stimmung stabiler macht.
 Tatsächlich kann man den Wirbel so weit gegen den Uhrzeigersinn verdrehen, daß die zurückdrehende Kraft viel größer als die Saitenkraft ist, und das Klavier kann sich dann beim Spielen selbst \textit{höher} verstimmen.
 Klar muß man den Wirbel richtig einstellen, damit man eine stabile Stimmung erzeugt.
 Diese Erfordernis wird bei den folgenden Stimmanweisungen berücksichtigt.
 \hypertarget{c2_5d}{}\hypertarget{c2_5_unis}{}

\subsection{Unisono stimmen}

Stecken Sie nun den Stimmhammer auf den Wirbel für Saite 1.
 Wir werden die Saite 1 nach Saite 2 stimmen.
 \textbf{Die Stimmbewegung, die Sie üben werden, ist:}
\begin{enumerate}
	\item \textbf{\textit{tiefer}}
	\item \textbf{\textit{höher}}
	\item \textbf{\textit{tiefer}}
	\item \textbf{\textit{höher}}
	\item \textbf{\textit{tiefer}}
\end{enumerate}

Außer bei (1) muß jede Bewegung kleiner als die vorhergehende sein.
 Wenn Sie besser werden, werden Sie Schritte passend hinzufügen oder weglassen.
 Wir nehmen an, daß die beiden Saiten fast gestimmt sind.
 Während Sie stimmen, müssen Sie zwei Regeln beachten:
 
\begin{itemize}
	\item \textbf{Drehen Sie nie einen Wirbel, wenn Sie nicht gleichzeitig auf den Ton hören.}
	\item \textbf{Lassen Sie nie den Druck auf den Griff des Stimmhammers nach, bis diese Bewegung komplett ist.}
\end{itemize}

Fangen wir z.B. mit Bewegung (1) \textit{tiefer} an: spielen Sie die Note alle ein oder zwei Sekunden mit der LH, so daß es einen dauernden Ton gibt, während Sie das Ende des Hammergriffs mit dem Daumen und dem Zeigefinger von sich weg drücken.
 Spielen Sie die Note so, daß Sie einen fortwährenden Ton aufrecht erhalten.
 Heben Sie die Taste nicht an, egal wie lang, da dies den Ton stoppt.
 Halten Sie die Taste unten, und spielen Sie mit einer schnellen Auf- und Abbewegung, so daß der Ton nicht unterbrochen wird.
 Der kleine Finger und der Rest Ihrer RH sollten gegen das Klavier abgestützt werden.
 Die erforderliche Bewegung des Hammers beträgt nur ein paar Millimeter.
 Zunächst werden Sie einen steigenden Widerstand spüren, und dann wird der Wirbel anfangen sich zu drehen.
 Bevor der Wirbel anfängt sich zu drehen, sollten Sie eine Veränderung im Ton hören.
 Hören Sie beim Drehen des Wirbels darauf, wie die Saite 1 \textit{tiefer} wird und eine Schwebung mit der mittleren Saite erzeugt; die Schwebungsfrequenz nimmt während Sie drehen zu.
 Hören Sie bei einer Schwebungsfrequenz von 2 bis 3 je Sekunde auf.
 Das äußere Ende des Hammergriffs sollte sich weniger als einen cm bewegen.
 Erinnern Sie sich daran, daß Sie nie den Wirbel drehen, wenn kein Ton zu hören ist, weil Sie sonst in bezug auf die Änderung der Schwebungen sofort die Orientierung verlieren.
 Halten Sie aus demselben Grund immer einen konstanten Druck auf den Hammer aufrecht, bis die Bewegung abgeschlossen ist.
 

Was ist die rationale Erklärung für die o.a. 5 Bewegungen?
 Angenommen, die beiden Saiten sind vernünftig gestimmt, dann stimmen Sie bei Schritt (1) die Saite 1 \textit{tiefer}, um sicherzustellen, daß Sie in Schritt (2) den Stimmpunkt passieren \footnote{d.h. den Punkt, an dem die Saite genau richtig gestimmt ist}.
 Das schützt auch gegen die Möglichkeit, daß Sie den Hammer auf den falschen Stimmwirbel gesetzt haben; solange Sie \textit{tiefer} stimmen, werden Sie niemals eine Saite zerbrechen.
 

Nach (1) sind Sie mit Sicherheit \textit{tiefer}, so daß Sie in Schritt (2) auf den Stimmpunkt hören können, während Sie durch ihn hindurchkommen.
 Gehen Sie darüber hinaus, bis Sie eine Schwebungsfrequenz von ungefähr 2 bis 3 je Sekunde auf der \textit{höher}en Seite hören und stoppen Sie.
 Sie wissen nun, wo der Stimmpunkt ist und wie er klingt.
 Der Grund dafür, so weit über den Stimmpunkt hinaus zu gehen, ist, daß Sie den Wirbel wie oben erklärt einstellen möchten.
 

Kehren Sie nun zu \textit{tiefer} zurück, Schritt (3), aber stoppen Sie dieses Mal direkt hinter dem Stimmpunkt, sobald Sie irgendwelche einsetzenden Schwebungen hören können.
 Der Grund, warum man nicht zu weit hinter den Stimmpunkt kommen möchte, ist, daß man nicht das \enquote{Einstellen des Wirbels} aus Schritt (2) rückgängig machen möchte.
 Achten Sie wieder genau darauf, wie der Stimmpunkt klingt.
 Er sollte perfekt sauber und rein klingen.
 Dieser Schritt stellt sicher, daß Sie den Wirbel nicht zu weit eingestellt haben.
 

Führen Sie nun das endgültige Stimmen durch, indem Sie in Richtung \textit{höher} gehen (Schritt 4), dabei so wenig wie möglich über die perfekte Stimmung hinausgehen und die Saite dann durch Drehen nach \textit{tiefer} (Schritt 5) in die endgültige Stimmung bringen.
 Beachten Sie, daß Ihre letzte Bewegung immer \textit{tiefer} sein muß, um den Wirbel einzustellen.
 Wenn Sie gut darin werden, könnten Sie in der Lage sein, das Ganze in drei Bewegungen (\textit{tiefer, höher, tiefer}) durchzuführen.
 

Idealerweise sollten Sie von Schritt (1) bis zur endgültigen Stimmung den Ton ohne Unterbrechung aufrechterhalten, immer Druck auf den Griff ausüben und niemals den Hammer loslassen.
 Am Anfang werden Sie das wahrscheinlich Bewegung für Bewegung ausführen müssen.
 Wenn Sie es beherrschen, wird die ganze Durchführung nur ein paar Sekunden dauern.
 Aber zunächst wird es \textit{viel} länger dauern.
 Bis Sie Ihre \enquote{Stimmuskeln} entwickelt haben, werden Sie schnell ermüden und von Zeit zu Zeit aufhören müssen, um sich zu erholen.
 Das gilt nicht nur für die Hand- und Armmuskeln, auch die erforderliche Konzentration des Geistes und des Gehörs auf die Schwebungen kann eine große Anstrengung sein und schnell Ermüdung verursachen.
 Sie müssen schrittweise eine \enquote{Stimmausdauer} entwickeln.
 Die meisten kommen besser zurecht, wenn Sie nur mit einem statt mit beiden Ohren hören; drehen Sie deshalb Ihren Kopf, um festzustellen, welches Ohr besser ist.
 

\textbf{Der häufigste Fehler, den Anfänger in diesem Stadium begehen, ist, bei dem Versuch, die Schwebungen zu hören, die Stimmbewegung zu unterbrechen.}
 Schwebungen sind schwer zu hören, wenn sich nichts ändert.
 Wenn der Wirbel nicht gedreht wird, ist schwer zu entscheiden, welche der vielen Dinge, die man hört, die Schwebung ist, auf die man sich konzentrieren muß.
 \textbf{Stimmer bewegen den Hammer weiter und hören dann auf \underline{die Veränderungen der Schwebungen}.}
 Wenn die Schwebungen sich ändern, ist es einfacher, die einzelne Schwebung zu identifizieren, die man für das Stimmen dieser Saite benutzt.
 Deshalb wird es nicht einfacher, wenn man die Stimmbewegung verlangsamt.
 Somit bewegt sich der Anfänger auf einem schmalen Grat.
 Wenn man den Wirbel zu schnell dreht, bricht die Hölle los und man verliert die Orientierung.
 Wenn man auf der anderen Seite zu langsam dreht, wird es schwierig, die Schwebungen zu identifizieren.
 Arbeiten Sie deshalb daran, den Bereich der Bewegung zu bestimmen, den Sie benötigen, um die Schwebungen zu erkennen und die richtige Geschwindigkeit, mit der Sie den Wirbel beständig drehen können, um die Schwebungen entstehen und verschwinden zu lassen.
 Falls Sie sich hoffnungslos verirrt haben, dämpfen Sie die Saiten 2 und 3, indem Sie einen Keil zwischen sie setzen, spielen Sie die Note, und hören Sie, ob Sie eine andere Note auf dem Klavier finden, die der Note nahe kommt.
 Wenn die andere Note tiefer ist als G3, dann müssen Sie \textit{höher} stimmen, um zurückzukommen, und umgekehrt.
 

Wenn Sie nun die Saite 1 mit Saite 2 gleich gestimmt haben, bringen Sie den Keil in eine neue Position, so daß Saite 1 gedämpft wird und die Saiten 2 und 3 frei schwingen können.
 Stimmen Sie Saite 3 nach Saite 2.
 Wenn Sie zufrieden sind, entfernen Sie den Keil und hören Sie, ob das G nun frei von Schwebungen ist.
 Sie haben eine Note gestimmt!
 Wenn das G ziemlich gut gestimmt war, bevor Sie angefangen haben, haben Sie nicht viel erreicht; finden Sie eine Note in der Nähe, die aus der Stimmung ist, um zu sehen, ob Sie sie \enquote{reinigen} können.
 Beachten Sie, daß Sie bei diesem Schema immer eine einzelne Saite nach einer anderen einzelnen Saite stimmen.
 Im Prinzip sind, wenn Sie wirklich gut sind, die Saiten 1 und 2 perfekt gestimmt, nachdem Sie mit dem Stimmen von 1 fertig sind, so daß Sie den Keil nicht mehr brauchen.
 Sie sollten in der Lage sein, Saite 3 nach den zusammen schwingenden 1 und 2 zu stimmen.
 In der Praxis funktioniert das nicht, bis Sie es wirklich beherrschen.
 Das kommt von einem Phänomen, das man \hyperlink{mitschwingung}{Mitschwingung} nennt.
 
%  c25e.html 
\hypertarget{c2_5e}{}\hypertarget{c2_5_mits}{}

\subsection{Mitschwingung}

Die Genauigkeit, die erforderlich ist, um zwei Saiten in perfekte Stimmung zu bringen, ist so hoch, daß es eine fast unmögliche Aufgabe ist.
 Es stellt sich heraus, daß es in der Praxis einfacher ist: \textbf{Wenn die Frequenzen sich in einem Bereich einander annähern, der \enquote{Mitschwingungsbereich} genannt wird, dann ändern die beiden Saiten ihre Frequenzen aufeinander zu, so daß Sie mit der gleichen Frequenz schwingen.}
 Das geschieht, weil die beiden Saiten nicht unabhängig sind, sondern am Steg miteinander gekoppelt.
 Wenn sie gekoppelt sind, dann bringt die Saite, die mit einer höheren Frequenz schwingt, die langsamere Saite dazu, mit einer etwas höheren Frequenz zu schwingen und umgekehrt.
 Der Nettoeffekt ist, daß beide Frequenzen zur Durchschnittsfrequenz der beiden hin getrieben werden.
 Somit wissen Sie, wenn Sie die Saiten 1 und 2 unisono stimmen, überhaupt nicht, ob sie perfekt gestimmt sind oder nur im Mitschwingungsbereich (außer wenn Sie ein erfahrener Stimmer sind).
 Am Anfang werden sie wahrscheinlich nicht perfekt gestimmt sein.
 

Wenn Sie nun versuchen müßten, die dritte Saite nach den beiden Saiten zu stimmen, die in Mitschwingung sind, würde die dritte Saite die Saite, die ihr in der Frequenz am nächsten ist, in Mitschwingung versetzen.
 Die andere Saite kann aber in bezug auf die Frequenz zu weit entfernt sein.
 Sie wird aus der Mitschwingung ausbrechen und dissonant klingen.
 Das Resultat ist, daß Sie, egal wo Sie sind, immer Schwebungen hören werden - der Stimmpunkt verschwindet!
 Man könnte meinen, daß wenn die dritte Saite in der Durchschnittsfrequenz der beiden Saiten, die in Mitschwingung sind, gestimmt wäre, alle drei zur Mitschwingung übergehen sollten.
 Es stellt sich heraus, daß das nicht geschieht, außer wenn alle drei Frequenzen perfekt gestimmt sind.
 Wenn die ersten beiden Saiten genügend weit auseinander sind, erfolgt ein komplexer Energietransfer zwischen allen drei Saiten.
 Sogar wenn die ersten beiden nah beieinander sind, gibt es höhere harmonische Schwingungen, die verhindern, daß alle Schwebungen verschwinden, wenn eine dritte Saite hinzukommt.
 Zusätzlich gibt es häufig Fälle, in denen man nicht alle Schwebungen völlig eliminieren kann, weil die beiden Saiten nicht identisch sind.
 Deshalb würde sich ein Anfänger völlig verirren, wenn er eine dritte Saite nach einem Paar Saiten stimmen sollte.
 \textbf{Bis Sie es beherrschen, den Mitschwingungsbereich herauszufinden, stimmen Sie immer eine Saite nach einer, niemals eine nach zwei.}
 Außerdem bedeutet, daß Sie 1 nach 2 und 3 nach 2 gestimmt haben, nicht, daß die drei Saiten \enquote{sauber} zusammen klingen werden.
 Prüfen Sie es immer; wenn die Saiten nicht völlig \enquote{sauber} sind, müssen Sie die störende Saite finden und es erneut versuchen.
 

Beachten Sie den Gebrauch des Ausdrucks \enquote{sauber}.
 Mit genügender Übung werden Sie bald aufhören, auf die Schwebungen zu hören; statt dessen werden Sie nach einem reinen Klang suchen, der sich irgendwo innerhalb des Mitschwingungsbereichs ergibt.
 Dieser Punkt hängt davon ab, welche Arten von Obertönen jede Saite erzeugt.
 Im Prinzip versuchen wir, wenn wir unisono stimmen, die Grundschwingungen zur Deckung zu bringen.
 In der Praxis ist ein kleiner Fehler in den Grundschwingungen verglichen mit demselben Fehler in einer hohen Oberschwingung unhörbar.
 Leider sind diese hohen Obertöne im allgemeinen keine exakten harmonischen Obertöne, sondern sind von Saite zu Saite unterschiedlich.
 Wenn die Grundtöne übereinstimmen, erzeugen deshalb diese hohen Obertöne hochfrequente Schwebungen, die die Note \enquote{schmutzig} oder \enquote{blechern} machen.
 Wenn die Grundtöne gerade so verstimmt sind, daß die Obertöne keine Schwebungen erzeugen, \enquote{versäubert} sich die Note.
 \textbf{Die Realität ist sogar noch komplizierter, weil einige Saiten, besonders bei Klavieren niedrigerer Qualität, eine zusätzliche Eigenresonanz haben, was es unmöglich macht, bestimmte Schwebungen völlig zu eliminieren.}
 Diese Schwebungen werden sehr ärgerlich, wenn man diese Note benutzen muß, um eine andere zu stimmen.
 \hypertarget{c2_5f}{}\hypertarget{c2_5_infi}{}

\subsection{Diese letze infinitesimale Bewegung ausführen}

Wir kommen nun zur nächsten Schwierigkeitsstufe.
 Finden Sie eine Note nahe G5, die leicht außerhalb der Stimmung ist, und wiederholen Sie das oben für G3 angegebene Verfahren.
 Die Stimmbewegungen für diese höheren Noten sind viel kleiner, was sie schwieriger macht.
 Sie werden vielleicht nicht in der Lage sein, durch das Drehen des Wirbels eine ausreichende Genauigkeit zu erreichen.
 Wir müssen eine neue Fertigkeit erlernen.
 \textbf{Diese Fertigkeit erfordert, daß Sie auf die Tasten \enquote{hämmern}, benutzen Sie deshalb Ihre Ohrenschützer oder Ohrstöpsel.}

Typischerweise werden Sie bei Bewegung (4) erfolgreich sein, aber bei Bewegung (5) wird sich der Wirbel entweder nicht bewegen oder über den Stimmpunkt hinwegspringen.
 \textbf{Damit die Saite sich in kleineren Schritten vorwärts bewegt, müssen Sie einen Druck auf den Stimmhammer ausüben, der knapp unter dem Punkt liegt, an dem der Wirbel springt.
 Schlagen Sie nun die Note fest an, während Sie den gleichen Druck auf den Stimmhammer aufrechterhalten.}
 Die zusätzliche Saitenspannung durch den harten Hammerschlag \textit{[Hammer der Klaviermechanik, nicht der Stimmhammer!]} läßt die Saite ein kleines Stück vorwärtsgehen.
 Wiederholen Sie das, bis sie perfekt gestimmt ist.
 Es ist wichtig, niemals den Druck auf den Stimmhammer nachzulassen und den Druck während dieser wiederholten Vorwärtssprünge konstant zu halten, oder Sie werden \textit{[in bezug auf die Saitenfrequenz]} schnell die Orientierung verlieren.
 Wenn die Saite perfekt gestimmt ist und Sie den Hammer loslassen, könnte der Wirbel zurückspringen und die Saite leicht \textit{tiefer} werden lassen. Sie werden aus der Erfahrung heraus lernen müssen, wie weit er zurückspringt, und es während des Stimmvorgangs entsprechend kompensieren müssen.
 

Die Notwendigkeit, auf die Saite zu hämmern, damit sie sich vorwärts bewegt, ist ein Grund, warum man Stimmer oft auf die Tasten hämmern hört.
 Es ist eine gute Idee, sich anzugewöhnen, die meisten Noten zu hämmern, weil das die Stimmung stabilisiert.
 Der daraus resultierende Ton kann so laut sein, daß das Ohr geschädigt wird, und eines der Berufsrisiken von Stimmern ist ein Gehörschaden wegen des Hämmerns.
 Die Lösung ist die Benutzung von Ohrenstöpseln.
 Beim Hämmern werden Sie auch mit Ohrstöpseln die Schwebungen problemlos hören.
 Das verbreitetste anfängliche Symptom eines Gehörschadens ist der Tinnitus (Klingeln im Ohr).
 Sie können die zum Hämmern notwendige Kraft minimieren, indem Sie den Druck auf den Stimmhammer erhöhen.
 Ein geringeres Hämmern ist auch erforderlich, wenn der Stimmhammer parallel zu den Saiten statt rechtwinklig dazu steht, und ein noch geringeres, wenn Sie ihn nach links zeigen lassen.
 Das ist ein weiterer Grund, warum viele Stimmer ihren Stimmhammer eher parallel zu den Saiten benutzen als rechtwinklig dazu.
 Beachten Sie, daß es zwei Möglichkeiten gibt, ihn parallel zu halten: zu den Saiten hin (12 Uhr) und von den Saiten weg (6 Uhr).
 Experimentieren Sie mit unterschiedlichen Hammerpositionen, wenn Sie an Erfahrung gewonnen haben, da Ihnen das viele Möglichkeiten für das Lösen verschiedener Probleme eröffnet.
 Mit dem beliebten 5-Grad-Kopf auf dem Hammer sind Sie z.B. nicht in der Lage, bei der höchsten Oktave den Griff nach rechts zeigen zu lassen, weil er auf den hölzernen Klavierrahmen treffen kann.
 \hypertarget{c2_5g}{}\hypertarget{c2_5_span}{}

\subsection{Ausgleich der Saitenspannung}

\textbf{Das Hämmern hilft auch dabei, die Saitenspannung gleichmäßiger auf die ganzen nicht klingenden Abschnitte der Saite zu verteilen, wie z.B. den Bereich im Duplex, aber besonders den Abschnitt zwischen dem Capotasto (Druckstab) und der Agraffe.}
 Es gibt eine Kontroverse darüber, ob der Ausgleich der Saitenspannung den Klang verbessert.
 Es steht außer Frage, daß eine gleichmäßige Spannung die Stimmung stabiler macht.
 Es kann jedoch fraglich sein, ob sie einen \textit{wesentlichen} Unterschied in der Stabilität ausmacht, besonders wenn die Wirbel während des Stimmens korrekt eingestellt wurden.
 Bei vielen Klavieren sind die Duplex-Abschnitte fast völlig mit Filz gedämpft, weil sie unerwünschte Schwingungen erzeugen könnten.
 Tatsächlich sind die \enquote{nicht klingenden} Abschnitte bei fast jedem Klavier gedämpft.
 Anfänger müssen sich über die Spannung in diesen Abschnitten der Saiten keine Gedanken machen.
 Deshalb ist das schwere Hämmern für den Anfänger nicht notwendig, obwohl es nützlich ist, diese Fertigkeit zu erlernen.
 

\textbf{\textit{Meiner persönlichen Meinung nach trägt der Klang des Duplex-Abschnitts nichts zum Klavierklang bei.}}
 In Wahrheit ist dieser Klang unhörbar und wird im Baß, wo er hörbar würde, völlig abgedämpft.
 Deshalb ist die \enquote{Kunst des Stimmens des Duplex-Abschnitts} ein Mythos, obwohl den meisten Klavierstimmern (einschließlich Reblitz!) von den Herstellern beigebracht wurde, daran zu glauben, weil es ein gutes Verkaufsargument abgibt.
 Der einzige Grund, warum man den Duplex-Abschnitt stimmen sollte, ist, daß der Steg sowohl im Knoten des klingenden als auch des nicht klingenden Bereichs sein sollte; ansonsten wird das Stimmen schwierig, der Sustain wird eventuell verkürzt, und man verliert die Gleichmäßigkeit.
 Wenn man Begriffe der Mechanik benutzt, kann man sagen, daß den Duplex-Abschnitt zu stimmen die Schwingungsimpedanz des Stegs optimiert.
 Mit anderen Worten: Der Mythos ändert nichts an der Fähigkeit der Stimmer, ihren Job zu machen.
 Nichtsdestoweniger ist ein gutes Verständnis sicher förderlich.
 Der Duplex-Abschnitt wird benötigt, damit der Steg sich freier bewegen kann, nicht für die Tonerzeugung.
 Offensichtlich verbessert der Duplex-Abschnitt die Klangqualität (des klingenden Bereichs), weil er die Impedanz des Stegs optimiert, aber nicht, weil er einen Ton erzeugt.
 Die Tatsache, daß der Duplex-Abschnitt im Baß gedämpft und im Diskant völlig unhörbar ist, beweist, daß der Klang des Duplex-Abschnitts nicht benötigt wird.
 Sogar im unhörbaren Diskant ist der Duplex-Abschnitt - um die Impedanz zu optimieren - in gewissem Sinne \enquote{gestimmt}, d.h. die Aliquotleiste ist so angebracht, daß die Länge des Duplex-Abschnitts der Saite eine harmonische Länge des klingenden Abschnitts der Saite ist (\enquote{aliquot} bedeutet \enquote{ohne Rest teilend}).
 Wenn der Ton des Duplex-Abschnitts hörbar wäre, dann müßte der Duplex-Abschnitt genauso sorgfältig gestimmt werden wie der klingende Abschnitt der Saite.
 Für das Anpassen der Impedanz muß das Stimmen jedoch nur annähernd genau sein, was in der Praxis auch der Fall ist.
 Manche Hersteller haben diesen Mythos des Duplex-Abschnitts ins Lächerliche gesteigert, indem sie auf der Seite des Stimmwirbels einen zweiten Duplex-Abschnitt vorsehen.
 Da der Hammer auf diesen Bereich (wegen des festen Capotasto) nur Zugkräfte übertragen kann, kann dieser Bereich der Saite nicht schwingen, um einen Klang zu erzeugen.
 Folglich gibt praktisch kein Hersteller ausdrücklich an, daß der nicht klingende Abschnitt auf der Seite der Stimmwirbel gestimmt werden soll.
 \hypertarget{c2_5h}{}\hypertarget{c2_5_disk}{}

\subsection{Wiegen im Diskant}

\textbf{Die am schwierigsten zu stimmenden Noten sind die höchsten.}
 Hier brauchen Sie beim Bewegen der Saiten eine unglaubliche Genauigkeit, und die Schwebungen sind schwer zu hören.
 Anfänger können leicht den Bezugspunkt verlieren und es schwer haben, den Weg zurück zu finden.
 Ein Vorteil der Notwendigkeit für solch kleine Bewegungen ist, daß Sie nun die wiegende Bewegung des Wirbels für das Stimmen benutzen können.
 Da die Bewegung so klein ist, schädigt das Wiegen des Wirbels nicht den Stimmstock.
 \textbf{Um den Wirbel zu wiegen, plazieren Sie den Stimmhammer parallel zu den Saiten, und lassen Sie ihn auf die Saiten zeigen (weg von Ihnen selbst).
 Um \textit{höher} zu stimmen, ziehen Sie am Hammer nach oben, und um \textit{tiefer} zu stimmen, drücken Sie nach unten.}
 Stellen Sie zuerst sicher, daß der Stimmpunkt nahe dem Mittelpunkt der wiegenden Bewegung ist.
 Wenn er es nicht ist, dann drehen Sie den Wirbel so, daß er es ist.
 Da diese Drehung viel größer ist als jene, die für das endgültige Stimmen benötigt wird, ist es nicht schwierig, aber denken Sie daran, den Wirbel richtig einzustellen.
 Es ist besser, wenn der Stimmpunkt vor der Mitte ist (nach der Saite zu), aber ihn zu weit nach vorne zu bringen, würde das Risiko bedeuten, den Stimmstock zu beschädigen, wenn man versucht \textit{tiefer} zu stimmen.
 Beachten Sie, daß \textit{höher} zu stimmen für den Stimmstock nicht so schädlich ist wie \textit{tiefer} zu stimmen, weil der Wirbel bereits gegen die Vorderseite des Lochs gedrückt ist.
 \hypertarget{c2_5i}{}\hypertarget{c2_5_bass}{}

\subsection{Grollen im Baß}

\textbf{Die tiefsten Baßsaiten sind (nach den höchsten Noten) jene, die am zweitschwierigsten zu stimmen sind.}
 Diese Saiten erzeugen einen Ton, der zum größten Teil aus höheren Obertönen besteht.
 Nahe dem Stimmpunkt sind die Schwebungen so langsam und leise, daß sie nur schwer zu hören sind.
 Manchmal kann man sie besser \enquote{hören}, indem man sein Knie gegen das Klavier drückt, um die Vibrationen zu fühlen, als zu versuchen, sie mit den Ohren zu hören, besonders im einsaitigen Abschnitt.
 Sie können das Unisono-Stimmen nur bis zum letzten zweisaitigen Abschnitt hinunter üben.
 \textbf{Stellen Sie fest, ob sie die hochtönenden, metallischen, klingelnden Schwebungen erkennen können, die in diesem Abschnitt vorherrschend sind.}
 Versuchen Sie, diese zu eliminieren, und sehen Sie, ob Sie ein wenig verstimmen müssen, um sie zu eliminieren.
 Wenn Sie diese hohen, klingelnden Schwebungen hören können, bedeutet das, daß Sie auf dem richtigen Weg sind.
 Machen Sie sich keine Gedanken, wenn Sie sie zunächst nicht einmal erkennen können - von Anfängern wird das nicht erwartet.
 \hypertarget{c2_5j}{}\hypertarget{c2_5_harm}{}

\subsection{Harmonisches Stimmen}

Wenn Sie mit Ihrer Fähigkeit unisono zu stimmen zufrieden sind, fangen Sie an, das Stimmen von Oktaven zu üben.
 Nehmen Sie eine Oktave nahe des mittleren C und dämpfen Sie die beiden oberen Saiten jeder Note, indem Sie einen Keil zwischen ihnen einfügen.
 Stimmen Sie die obere Note nach der Note eine Oktave unterhalb davon und umgekehrt.
 Beginnen Sie wie beim Unisono nahe dem mittleren C, arbeiten Sie sich dann bis zu den höchsten Noten im Diskant vor, und üben Sie dann im Baß.
 Wiederholen Sie die gleiche Übung mit den Quinten, Quarten und den großen Terzen.
 

\textbf{Nachdem Sie perfekte Harmonien stimmen können, versuchen Sie sie zu verstimmen, um festzustellen, ob Sie die zunehmenden Schwebungsfrequenzen hören können, wenn Sie ganz leicht von der perfekten Stimmung abweichen.}
 Versuchen Sie, verschiedene Schwebungsfrequenzen zu identifizieren, insbesondere 1 bps (beats per second = Schwebungen je Sekunde) und 10 bps, indem Sie Quinten benutzen.
 Diese Fertigkeiten werden sich später als nützlich erweisen.
 \hypertarget{c2_5k}{}\hypertarget{c2_5_stre}{}

\subsection{Was ist Streckung?}

Harmonisches Stimmen ist immer mit einem Phänomen verbunden, das Streckung genannt wird.
 Harmonische Obertöne in Klaviersaiten sind niemals exakt, weil reale Saiten, die an realen Enden befestigt sind, sich nicht wie ideale mathematische Saiten verhalten.
 Diese Eigenschaft der nicht exakten Obertöne nennt man Inharmonizität.
 Die Differenz zwischen den tatsächlichen und den theoretischen harmonischen Frequenzen nennt man Streckung.
 Experimentell findet man, daß die meisten harmonischen Obertöne im Vergleich zu ihren idealen theoretischen Werten \textit{höher} sind, obwohl es ein paar geben kann, die \textit{tiefer} sind.
 

Gemäß eines Untersuchungsergebnisses (Young, 1952) wird Streckung durch Inharmonizität verursacht, die aus der Steifheit der Saiten resultiert.
 Ideale mathematische Saiten haben eine Steifheit von Null.
 Steifheit ist eine extrinsische Eigenschaft - sie hängt von den Abmessungen des Drahtes ab.
 Wenn diese Erklärung richtig ist, dann muß Streckung ebenfalls extrinsisch sein.
 Wenn eine bestimmte Art Stahl vorgegeben ist, dann ist der Draht steifer, wenn er dicker oder kürzer ist.
 Eine Konsequenz aus dieser Abhängigkeit von der Steifheit ist eine Steigerung der Frequenz mit der Zahl der \hyperlink{moden}{Schwingungsmoden}; d.h. der Draht erscheint bei harmonischen Obertönen mit kürzeren Wellenlängen steifer.
 Steifere Drähte vibrieren schneller, weil sie zusätzlich zur Saitenspannung eine weitere Rückstellkraft haben.
 Diese Inharmonizität wurde mit einer Genauigkeit von einigen Prozent berechnet, so daß die Theorie richtig erscheint, und dieser einzelne Mechanismus scheint für den größten Teil der beobachteten Streckung verantwortlich zu sein.
 

Diese Berechnungen zeigen, daß die Streckung für die zweite Schwingungsmode bei C4 ungefähr 1,2 Cent beträgt und sich ungefähr alle 8 Halbtöne bei höheren Frequenzen verdoppelt (C4 = mittleres C, die erste Mode ist die tiefste oder Grundfrequenz, ein Cent ist ein hundertstel Halbton, und es gibt 12 Halbtöne in einer Oktave).
 Die Streckung wird für tiefere Noten kleiner, besonders unterhalb von C3, weil die drahtumwickelten Saiten ziemlich flexibel sind.
 Die Streckung nimmt schnell mit steigender Modenzahl zu und nimmt mit steigender Saitenlänge noch schneller ab.
 Prinzipiell ist die Streckung bei größeren Klavieren kleiner und bei Klavieren mit geringerer Spannung größer, wenn Saiten mit dem gleichen Durchmesser benutzt werden.
 Streckung führt zu Problemen beim Entwerfen von Tonleitern, weil abrupte Veränderungen des Saitentyps, Saitendurchmessers, der Länge, usw. eine diskontinuierliche Veränderung in der Streckung erzeugen.
 Obertöne sehr hoher Moden bereiten, wenn Sie ungewöhnlich laut sind, wegen ihrer großen Streckung Probleme beim Stimmen - ihre Schwebungen herauszustimmen könnte die unteren, wichtigeren Obertöne hörbar aus der Stimmung bringen.
 

Da größere Klaviere oft eine geringere Streckung haben, aber auch dazu neigen, besser zu klingen, könnte man daraus schließen, daß eine kleinere Streckung besser ist.
 Die Differenz der Streckung ist jedoch im allgemeinen gering, und die Klangqualität eines Klaviers wird zu einem großen Teil von anderen Eigenschaften als der Streckung kontrolliert.
 

Beim harmonischen Stimmen stimmt man z.B. die Grundfrequenz oder einen Oberton der oberen Note nach einem höheren Oberton der tieferen Note.
 Die resultierende neue Note ist kein genaues Vielfaches der tieferen Note, sondern ist um den Betrag der Streckung \textit{höher}.
 Das interessante an der Streckung ist, daß eine Tonleiter mit Streckung \enquote{lebhaftere} Musik erzeugt als eine ohne!
 Das hat einige Stimmer veranlaßt, mit doppelten Oktaven statt mit einzelnen Oktaven zu stimmen, was die Streckung vergrößert.
 

Der Betrag der Streckung ist für jedes Klavier einzigartig und, in Wahrheit, einzigartig für jede Note des Klaviers.
 Moderne elektronische Stimmhilfen sind genügend mächtig, um die Streckung für alle gewünschten Noten eines bestimmten Klaviers aufzuzeichnen.
 Stimmer mit elektronischen Stimmhilfen können auch die durchschnittliche Streckung oder die Streckungsfunktion für jedes Klavier berechnen und das Klavier entsprechend stimmen.
 Tatsächlich gibt es anekdotenhafte Berichte über Pianisten, die eine Streckung weit über der natürlichen Streckung des Klaviers wünschen.
 Beim auralen Stimmen wird die Streckung natürlich und genau berücksichtigt.
 Deshalb muß der Stimmer, obwohl die Streckung ein wichtiger Aspekt des Stimmens ist, nichts besonderes tun, um die Streckung einzubeziehen, wenn man nur die natürliche Streckung des Klaviers möchte.
 \hypertarget{c2_5l}{}\hypertarget{c2_5_prae}{}

\subsection{Präzision, Präzision, Präzision}

\textbf{Das, worum es beim Stimmen geht, ist Präzision.}
 Alle Stimmverfahren sind so angeordnet, daß man nacheinander die erste Note nach einer Stimmgabel stimmt, die zweite nach der ersten, usw.
 Deshalb werden sich eventuelle Fehler schnell aufaddieren.
 Tatsächlich wird ein Fehler an einem Punkt oft einige nachfolgende Schritte unmöglich machen.
 Das geschieht, weil man auf den kleinsten Hinweis auf eine Schwebung hört, und wenn die Schwebungen in einer Note nicht vollständig eliminiert wurden, kann man sie nicht benutzen, um eine andere zu stimmen, weil diese Schwebungen klar zu hören sein werden.
 Das wird bei Anfängern, bevor sie gelernt haben, wie präzise man sein muß, tatsächlich oft geschehen.
 Wenn das geschieht, hört man Schwebungen, die man nicht eliminieren kann.
 Gehen Sie in diesem Fall zu Ihrer Referenznote zurück, und stellen Sie fest, ob sie die gleiche Schwebung hören; wenn das so ist, ist dort der Ursprung Ihres Problems - beseitigen Sie es.
 

\textbf{Der beste Weg, die Präzision sicherzustellen, ist, die Stimmung zu prüfen.}
 Fehler treten auf, weil jede Saite anders ist und Sie nie sicher sind, daß die Schwebung, die Sie hören, jene ist, nach der Sie suchen; das gilt besonders für Anfänger.
 Ein weiterer Faktor ist, daß Sie die Schwebungen pro Sekunde (bps) zählen müssen, und Ihre Vorstellung von sagen wir 2 bps wird an verschiedenen Tagen oder zu verschiedenen Zeiten desselben Tags unterschiedlich sein, bis Sie sich diese \enquote{Schwebungsgeschwindigkeiten} gut gemerkt haben.
 Wegen der entscheidenden Wichtigkeit der Präzision zahlt es sich aus, jede gestimmte Note zu prüfen.
 Das gilt besonders, wenn Sie \enquote{\hyperlink{c2_6}{die Bezugsnoten einstellen}}, was unten erklärt wird.
 Unglücklicherweise ist die Note genauso schwierig zu prüfen, wie sie richtig zu stimmen ist; d.h. ein Person, die nicht hinreichend genau stimmen kann, ist üblicherweise unfähig, eine sinnvolle Prüfung durchzuführen.
 Außerdem funktioniert das Prüfen nicht, wenn die Stimmung weit genug daneben ist.
 Deshalb \textbf{habe ich Methoden gewählt, die ein Minimum an Prüfungen benutzen.}
 Die resultierende Stimmung wird für die Gleichschwebende Temperatur zunächst nicht sehr gut sein.
 Die \hyperlink{c2_6_kirn}{Kirnberger-Stimmung} (s.u.) ist einfacher akkurat zu stimmen.
 Auf der anderen Seite können Anfänger ohnehin keine guten Stimmungen erzeugen, unabhängig davon, welche Methoden Sie benutzen.
 Zumindest werden die Verfahren, die unten vorgestellt werden, eine Stimmung bieten, die keine Katastrophe sein sollte und die besser wird, sobald sich Ihre Fertigkeiten verbessern.
 \textbf{Tatsächlich ist das wahrscheinlich der schnellste Weg zum Lernen.}
 Nachdem Sie sich genug verbessert haben, können Sie die Prüfungsverfahren untersuchen, wie jene, die bei Reblitz oder in \enquote{Tuning} von Jorgensen angegeben sind.
 
%  c26.html 
\hypertarget{c2_6}{}

\section{Stimmverfahren}\hypertarget{c2_6a}{}

\subsection{Einleitung}

Stimmen besteht aus dem \enquote{Einstellen der Bezugsnoten} in einer Oktave in der Nähe des mittleren C und daraus, diese Oktave in geeigneter Weise auf alle anderen Tasten zu \enquote{kopieren}.
 Sie werden verschiedene harmonische Stimmungen benötigen, um die Bezugsnoten einzustellen, und zunächst wird nur die mittlere Saite jeder Note der \enquote{Bezugsoktave} gestimmt.
 Das \enquote{Kopieren} wird durch das Stimmen in Oktaven durchgeführt.
 Wenn eine Saite jeder Note auf diese Art gestimmt ist, werden die restlichen Saiten jeder Note unisono gestimmt.
 

Beim Einstellen der Bezugsnoten müssen wir uns entscheiden, welche Temperatur wir benutzen möchten.
 Wie oben in \hyperlink{c2_2}{Abschnitt 2} erklärt wurde, sind die meisten Klaviere heutzutage auf \hyperlink{et1}{gleichschwebende Temperatur (ET)} gestimmt, aber die historischen Temperaturen, insbesondere die \hyperlink{c2_2_wtk2}{Wohltemperierten Stimmungen (WT)} erfreuen sich zunehmender Beliebtheit.
 Deshalb habe ich ET und eine WT, \hyperlink{c2_6_kirn}{Kirnberger II (K-II)}, für dieses Kapitel ausgewählt.
 K-II ist eine der am leichtesten zu stimmenden Temperaturen; deshalb werden wir diese zuerst ansehen.
 Die meisten, die nicht mit den verschiedenen Temperaturen vertraut sind, werden zunächst keinen Unterschied zwischen ET und K-II bemerken; sie werden beide im Vergleich zu einem verstimmten Klavier hervorragend klingen.
 Auf der anderen Seite sollten die meisten Klavierspieler einen deutlichen Unterschied hören und in der Lage sein, eine Meinung oder eine Vorliebe zu entwickeln, wenn man ihnen bestimmte Musikstücke vorspielt und die Unterschiede aufzeigt.
 Der einfachste Weg für Außenstehende, sich die Unterschiede anzuhören, ist, ein modernes elektronisches Klavier zu benutzen, das alle diese Temperaturen eingebaut hat, und dasselbe Stück mit jeder der Temperaturen zu spielen.
 Benutzen Sie als ein leichtes Teststück z.B. den ersten Satz von Beethovens Mondschein-Sonate; als ein schwierigeres Stück können Sie den dritten Satz seiner Waldstein-Sonate benutzen.
 Probieren Sie auch ein paar Ihrer Lieblingsstücke von Chopin aus.
 Mein Vorschlag für einen Anfänger ist, zuerst K-II zu lernen, so daß man ohne zu viele Schwierigkeiten anfangen kann, und dann ET zu lernen, wenn man schwierigere Aufgaben in Angriff nehmen kann.
 Ein Nachteil dieses Plans ist, daß man eventuell K-II so sehr gegenüber ET bevorzugt, daß man sich nie dazu entschließt, ET zu lernen.
 Wenn man sich an K-II gewöhnt hat, wird ET ein wenig ungenügend oder \enquote{schmutzig} klingen.
 Man kann jedoch nicht wirklich als Stimmer angesehen werden, bevor man nicht ET stimmen kann.
 Auch gibt es viele WTs, auf die Sie vielleicht einen Blick werfen möchten, die in verschiedener Hinsicht K-II überlegen sind.
 

WT-Stimmungen sind wünschenswert, weil sie perfekte Harmonien haben, die der Kern der Musik sind.
 Sie haben jedoch einen großen Nachteil.
 Weil die perfekten Harmonien so schön sind, treten die Dissonanzen in den \enquote{Wolfs}-Tonleitern hervor und sind sehr unangenehm.
 Nicht nur das, sondern jede Saite, die ein wenig aus der Stimmung ist, ist sofort zu erkennen.
 Deshalb erfordern WT-Stimmungen ein viel häufigeres Stimmen als ET.
 Man könnte meinen, daß ein leichtes Verstimmen der Unisono-Saiten bei ET genauso unangenehm wäre, aber offenbar sind, wenn die Intervalle wie bei der ET aus der Stimmung sind, die geringfügigen Abweichungen in der Stimmung der Unisono-Saiten bei ET weniger wahrnehmbar.
 Deshalb kann für Klavierspieler, die ein sensibles Gehör für das Stimmen haben, WT ziemlich unangenehm sein, solange sie ihr Klavier nicht selbst stimmen können.
 Das ist ein wichtiger Punkt, weil die meisten Klavierspieler, die die Vorteile der WT hören können, empfindlich auf die Stimmung reagieren.
 Die Erfindung des selbststimmenden Klaviers kann vielleicht der Retter der WT sein, weil das Klavier immer richtig gestimmt sein wird.
 Deshalb wird WT eventuell nur durch elektronische und selbststimmende Klaviere (wenn sie verfügbar werden - s. \hyperlink{c1iv6h}{Abschnitt IV.6h \enquote{Die Zukunft des Klaviers}}) eine breite Zustimmung finden.
 

Sie können das Stimmen in ET überall beginnen, aber die meisten Stimmer benutzen die Stimmgabel A440 um anzufangen, weil Orchester im allgemeinen nach A440 stimmen.
 Das Ziel bei K-II ist, C-Dur und so viele Tonarten \enquote{in der Nähe} wie möglich rein zu haben (mit reinen Intervallen), weshalb das Stimmen mit dem mittleren C (C4\nolinebreak=\nolinebreak261,6 - die meisten Stimmer benutzen die C523,3-Stimmgabel um das mittlere C zu stimmen) begonnen wird.
 Nun ist das aus K-II resultierende A, wenn man vom richtigen C aus stimmt, nicht das A440.
 Deshalb benötigen Sie zwei Stimmgabeln (A und C), um sowohl ET als auch K-II stimmen zu können.
 Alternativ können Sie nur mit einer C-Gabel beginnen und fangen das Stimmen in ET bei C an.
 Zwei Stimmgabeln zu haben ist ein Vorteil, denn egal ob Sie von C oder von A aus starten, können Sie sich selbst überprüfen, wenn Sie bei ET bei der anderen Note ankommen.
 \hypertarget{c2_6b}{}\hypertarget{c2_6_gabe}{}

\subsection{Das Klavier nach der Stimmgabel stimmen}

Einer der schwierigsten Schritte beim Stimmvorgang ist das Stimmen des Klaviers nach der Stimmgabel.
 Diese Schwierigkeit hat zwei Ursachen:
 
\begin{enumerate}
	\item Die Stimmgabel hat eine andere - üblicherweise kürzere - Aushaltezeit (Sustain) für den Ton als das
  Klavier, so daß die Gabel ausklingt, bevor Sie einen genauen Vergleich machen können.
	\item Die Gabel erzeugt eine reine Sinuswelle ohne die lauten Obertöne der Klaviersaiten.
\end{enumerate}

Deshalb kann man keine Schwebungen mit höheren Obertönen benutzen, um die Genauigkeit des Stimmens zu erhöhen, wie man es mit zwei Klaviersaiten tun kann.
 Ein Vorteil von elektronischen Stimmgeräten ist, daß sie so programmiert werden können, daß sie Referenztöne mit Rechteckschwingungen liefern, die eine große Anzahl von höheren harmonischen Obertönen beinhalten.
 Diese hohen harmonischen Obertöne (die notwendig sind, um die scharfen Ecken der Rechteckschwingungen zu erzeugen) sind für eine höhere Stimmgenauigkeit nützlich.
 Wir müssen deshalb diese beiden Probleme lösen, damit wir das Klavier genau nach der Stimmgabel stimmen können.
 

Beide Schwierigkeiten können beseitigt werden, wenn wir das Klavier als Stimmgabel benutzen können und diesen Übergang von der Stimmgabel zum Klavier durchführen, indem wir einige hohe harmonische Obertöne des Klaviers benutzen.
 Finden Sie, um diesen Übergang zu erreichen, eine Note innerhalb der gedämpften Noten, die laute Schwebungen mit der Gabel erzeugt.
 Wenn Sie keine Note finden können, benutzen Sie die Note einen Halbton höher oder tiefer; benutzen Sie z.B. für eine Stimmgabel A das Ab oder A\# auf dem Klavier.
 Wenn diese Schwebungsfrequenzen etwas zu hoch sind, versuchen Sie die gleichen Noten eine Oktave tiefer.
 Stimmen Sie nun das A auf dem Klavier so, daß es Schwebungen der gleichen Frequenz mit diesen Bezugsnoten erzeugt (Ab, A\#, oder jede andere Note, die Sie gewählt haben).
 Die beste Möglichkeit, die Stimmgabel zu hören, ist, sie \hyperlink{c2_3_gabel}{wie oben beschrieben} gegen den Tragus zu drücken oder sie auf irgendeine große, harte, flache Oberfläche zu drücken.
 \hypertarget{c2_6c}{}\hypertarget{c2_6_kirn}{}

\subsection{Kirnberger II}
\begin{itemize}
	\item Dämpfen Sie alle Nebensaiten von F3 bis F4.
	\item Stimmen Sie C4 (das mittlere C) nach der Gabel.
	\item Benutzen Sie dieses C4, um G3 (Quarte), E4 (Terz), F3 (Quinte), und F4 (Quarte) zu stimmen.
	\item Benutzen Sie G3, um D4 (Quinte) und H3 (Terz) zu stimmen.
	\item Benutzen Sie H3, um F\#3 (Quarte) zu stimmen.
	\item Benutzen Sie F\#3, um Db4 (Quinte) zu stimmen.
	\item Benutzen Sie F3, um B3 (Quarte) zu stimmen.
	\item Benutzen Sie B3, um Eb4 (Quarte) zu stimmen.
	\item Benutzen Sie Eb4, um Ab3 (Quinte) zu stimmen.
	\item Alle Stimmungen bis hierhin sind \textit{rein}.
\\
 Stimmen Sie nun A3 so, daß die Schwebungsfrequenzen von F3-A3 und A3-D4 die gleichen sind.
\end{itemize}


 Sie sind fertig mit dem Einstellen der Bezugsnoten!
 \hypertarget{c2_6_kirn2}{}

Stimmen Sie nun in \textit{reinen} Oktaven aufwärts, bis zu den höchsten Noten.
 Stimmen Sie dann abwärts, bis zu den tiefsten Noten.
 Beginnen Sie dabei mit der Bezugsoktave als Referenz.
 Stimmen Sie bei all diesen Stimmungen nur eine neue Oktavsaite, während Sie die anderen dämpfen.
 Stimmen Sie dann die eine bzw. zwei verbleibenden Saiten mit der neu gestimmten Saite \hyperlink{c2_5_unis}{unisono}.
 

Das ist ein Moment, in dem Sie die Regel \enquote{Stimmen Sie eine Saite nach einer anderen.} brechen sollten.
 Wenn z.B. Ihre Referenznote eine (gestimmte) 3-saitige Note ist, benutzen Sie sie wie sie ist, ohne eine Saite davon zu dämpfen.
 Das dient als ein Test der Qualität Ihres Stimmens.
 Wenn es Ihnen schwer fällt, sie zu benutzen, um eine neue einzelne Saite zu stimmen, dann war u.U. Ihr Unisono-Stimmen der Referenznote nicht genügend genau, und Sie sollten zurückgehen und sie bereinigen.
 Wenn Sie auch nach erheblicher Mühe nicht 3 gegen 1 stimmen können, haben Sie selbstverständlich keine andere Chance, als zwei der drei Saiten zu dämpfen, damit Sie vorwärtskommen.
 Sie gefährden jedoch die Qualität des Stimmens.
 Wenn alle Noten im Diskant und Baß fertig sind, dann sind die einzigen ungestimmten Noten jene, die Sie für das Einstellen der Bezugsnoten gedämpft haben.
 Stimmen Sie diese - mit der tiefsten Note beginnend - unisono mit ihrer mittleren Saite, indem Sie vom Filz jeweils eine Schleife wegziehen.
 \hypertarget{c2_6d}{}\hypertarget{c2_6_et}{}

\subsection{Gleichschwebende Temperatur \protect\hyperlink{et}{(gleichstufige Temperatur, gleichmäßige Temperatur)}}

Ich zeige hier das leichteste, angenäherte Verfahren für die gleichschwebende Temperatur.
 Genauere Algorithmen kann man in der Literatur finden (Reblitz, Jorgensen).
 Kein professioneller Stimmer, der etwas auf sich hält, würde dieses Schema benutzen; wenn man jedoch gut darin wird, kann man eine annehmbare gleichschwebende Temperatur erzeugen.
 Bei einem Anfänger werden die vollständigeren und präziseren Schemata nicht notwendigerweise zu besseren Ergebnissen führen.
 Mit den komplexeren Methoden kann ein Anfänger schnell durcheinander kommen, ohne eine Vorstellung davon zu haben, was er falsch gemacht hat.
 Mit der hier gezeigten Methode kann man schnell die Fähigkeit entwickeln, herauszufinden, was man falsch gemacht hat:
 
\begin{itemize}
	\item Dämpfen Sie die Nebensaiten von G3 bis C\#5.
	\item Stimmen Sie A4 nach der A440 Gabel.
	\item Stimmen Sie A3 nach A4.
	\item Stimmen Sie dann mit verkürzten Quinten von A3 aus aufwärts, bis Sie nicht mehr weiter aufwärts gehen können, ohne den gedämpften Bereich zu verlassen, dann eine Oktave tiefer, und wiederholen Sie dieses \enquote{aufwärts in Quinten und eine Oktave abwärts}-Verfahren bis Sie zu A4 kommen.
 Sie beginnen z.B. mit einer verkürzten A3-E4, dann einer verkürzten E4-H4.
 Die nächste Quinte würde Sie über die höchste gedämpfte Note, C\#5, hinausführen; stimmen Sie deshalb eine Oktave abwärts, H4-H3.
\end{itemize}


 Alle Oktaven sind selbstverständlich \textit{rein}.
 Die verkürzten Quinten sollten am unteren Ende des gedämpften Bereichs mit etwas weniger als 1 Hz schweben und ungefähr 1,5 Hz am oberen Ende.
 Die Schwebungsfrequenzen der Quinten zwischen dieser oberen und unteren Grenze sollten langsam mit zunehmender Tonhöhe steigen.
 

Wenn Sie in Quinten aufwärts gehen, stimmen Sie von \textit{rein} zu \textit{tiefer}, um eine verkürzte Quinte zu erzeugen.
 Deshalb können Sie mit \textit{rein} beginnen und \textit{tiefer} stimmen, um gleichzeitig die Schwebungsfrequenz auf den gewünschten Wert zu steigern und \hyperlink{c2_5_wirb}{den Wirbel richtig einzustellen}.
 Wenn Sie alles perfekt getan haben, sollte das letzte D4-A4 ohne neu zu stimmen eine verkürzte Quinte mit einer Schwebungsfrequenz von 1 Hz sein.
 Dann sind Sie fertig.
 Sie haben gerade einen \enquote{Quintenzirkel} beendet.
 Das Wunder des Quintenzirkels ist, daß er jede Note einmal stimmt, ohne irgendeine in der A3-A4-Oktave zu überspringen!
 

Wenn die abschließende D4-A4 nicht richtig ist, haben Sie irgendwo einen Fehler begangen.
 Kehren Sie in diesem Fall die Prozedur um; beginnen Sie bei A4, gehen Sie in verkürzten Quinten abwärts und in Oktaven aufwärts, bis Sie A3 erreichen, wobei die abschließende A3-E4 eine verkürzte Quinte mit einer Schwebungsfrequenz von etwas weniger als 1 Hz sein sollte.
 Um in Quinten abwärts zu gehen, erzeugen Sie eine verkürzte Quinte, indem Sie von \textit{rein} nach \textit{höher} stimmen.
 Dieser Schritt des Stimmens wird jedoch den Wirbel nicht einstellen.
 Um den Wirbel korrekt einzustellen, müssen Sie deshalb zunächst \textit{zu hoch} stimmen und dann die Schwebungsfrequenz auf den gewünschten Wert vermindern.
 Deshalb ist in Quinten abwärts zu gehen eine schwierigere Operation als in Quinten aufwärts zu gehen.
 

Eine alternative Methode ist, mit A anzufangen, mit Quinten aufwärts bis zum C zu stimmen und dieses C mit einer Stimmgabel zu prüfen.
 Wenn Ihr C \textit{zu hoch} ist, waren Ihre Quinten nicht ausreichend verkürzt und umgekehrt.
 Eine weitere Variation ist, in Quinten von A3 aus etwas mehr als die Hälfte der Strecke aufwärts zu stimmen und dann von A4 bis zur letzten Note, die Sie beim Aufwärtsgehen gestimmt haben, abwärts zu stimmen.
 

Wenn die Bezugsnoten eingestellt sind, fahren Sie wie oben im Abschnitt über \hyperlink{c2_6_kirn2}{Kirnberger II} beschrieben fort.
 
%  c27.html 
\hypertarget{c2_7}{}

\section{Kleinere Reparaturen durchführen}

Wenn man mit dem Stimmen angefangen hat, muß man zwangsläufig kleinere Reparaturen und ein paar Wartungsarbeiten durchführen.
 \hypertarget{c2_7a}{}\hypertarget{c2_7_hamm}{}

\subsection{Intonieren der Hämmer}

\textbf{Ein verbreitetes Problem, das man bei vielen Klavieren findet, sind verdichtete Hämmer.
 Ich bringe diesen Punkt zur Sprache, weil der Zustand der Hämmer für die richtige Entwicklung der Klaviertechnik und der Fertigkeiten für das Auftreten viel wichtiger ist, als vielen Menschen bewußt ist.}
 Zahlreiche Stellen in diesem Buch weisen auf die Wichtigkeit des musikalischen Übens für das Erwerben der Technik hin.
 Man kann aber nicht musikalisch spielen, wenn die Hämmer ihre Aufgabe nicht erfüllen können - ein entscheidender Punkt, der sogar von vielen Stimmern übersehen wird (oftmals weil sie fürchten, daß die zusätzlichen Kosten die Kunden vergraulen würden).
 Bei einem Flügel ist, daß man es für notwendig hält, den Deckel zumindest teilweise zu schließen, um leise Passagen zu spielen, ein sicheres Zeichen für verdichtete Hämmer.
 Ein weiteres sicheres Zeichen ist, daß man dazu neigt, das Dämpferpedal zu Hilfe zu nehmen, um leise zu spielen.
 Verdichtete Hämmer erzeugen entweder einen lauten Ton oder überhaupt keinen.
 Jede Note neigt dazu, mit einem lästigen, perkussiven Schlag zu beginnen, der zu stark ist, und der Klang ist übermäßig hell.
 Es sind diese perkussiven Schläge, die für das Gehör des Stimmers so schädlich sind.
 Ein richtig intoniertes Klavier erlaubt die Kontrolle über den ganzen Dynamikbereich und erzeugt einen gefälligeren Klang.
 

Lassen Sie uns zunächst sehen, wie ein verdichteter Hammer zu so extremen Resultaten führen kann.
 Wie können kleine, leichte Hämmer laute Töne erzeugen, wenn sie mit relativ geringer Kraft auf eine Saite treffen, die unter einer solch hohen Spannung steht?
 Wenn man versuchen würde, die Saite herunterzudrücken oder zu zupfen, müßte man eine ziemlich große Kraft aufwenden, um nur einen kleinen Ton zu erzeugen.
 Die Antwort liegt in einem unglaublichen Phänomen, das auftritt, wenn straff gespannte Saiten im rechten Winkel angeschlagen werden.
 \textbf{Es stellt sich heraus, daß die vom Hammer erzeugte Kraft im Moment des Aufpralls theoretisch unendlich ist!}
 Diese fast unendliche Kraft ist es, was den leichten Hammer in die Lage versetzt, praktisch jede erreichbare Spannung der Saite zu überwinden und sie zum Schwingen zu bringen.
 

Hier ist die Berechnung dieser Kraft.
 Stellen Sie sich vor, daß der Hammer an seinem höchsten Punkt ist, nachdem er die Saite angeschlagen hat (Flügel).
 Die Saite bildet zu diesem Zeitpunkt mit ihrer ursprünglichen horizontalen Position ein Dreieck (das ist nur eine idealisierte Näherung, s.u.).
 Die kürzeste Seite dieses Dreiecks ist der Abstand zwischen der Agraffe und dem Aufschlagspunkt des Hammers.
 Die zweitkürzeste Seite ist die vom Hammer bis zum Steg.
 Die längste ist die ursprüngliche horizontale Lage der Saite, eine gerade Linie vom Steg zur Agraffe.
 Wenn wir nun eine vertikale Linie vom Aufschlagspunkt des Hammers nach unten zur ursprünglichen Saitenposition ziehen, erhalten wir zwei aneinanderliegende rechtwinklige Dreiecke.
 Das sind zwei extrem spitze rechtwinklige Dreiecke, die sehr kleine Winkel an der Agraffe und dem Steg haben; wir werden diese kleinen Winkel \enquote{theta} nennen.
 

Das einzige, das wir zu dieser Zeit kennen, ist die Kraft des Hammers, aber das ist nicht die Kraft, die die Saite bewegt, weil der Hammer die Saitenspannung überwinden muß, bevor die Saite nachgibt.
 D.h. die Saite kann sich nicht aufwärts bewegen, solange sie nicht länger werden kann.
 Das ist verständlich, wenn man sich die beiden oben beschriebenen rechtwinkligen Dreiecke ansieht.
 Die Saite hatte, bevor der Hammer auftraf, die Länge der langen Katheten der rechtwinkligen Dreiecke, aber nach dem Auftreffen bildet die Saite die Hypotenusen, welche länger sind.
 D.h., wenn die Saite absolut unelastisch wäre und die Enden der Saiten wären fest fixiert, könnte keine noch so große Hammerkraft die Saite dazu bringen, sich zu bewegen.
 

Es ist eine einfache Angelegenheit, mit Vektordiagrammen zu zeigen, daß die \textit{zusätzliche} Spannungskraft F (zusätzlich zu der ursprünglichen Saitenspannung), die vom Hammeraufschlag erzeugt wird, durch $f=F*sin(theta)$ gegeben ist, wobei f die Kraft des Hammers ist.
 Es ist egal, welches rechtwinklige Dreieck wir für diese Berechnung verwenden (das auf der Seite des Stegs oder das auf der Seite der Agraffe).
 Deshalb ist die Saitenspannung $F=f/sin(theta)$.
 Im ersten Moment des Auftreffens ist theta\nolinebreak=\nolinebreak0, und deshalb F\nolinebreak=\nolinebreak unendlich!
 Das geschieht, weil sin(0)\nolinebreak=\nolinebreak0.
 Selbstverständlich kann F nur unendlich werden, wenn die Saite sich nicht strecken kann und sich nichts anderes bewegt.
 In der Realität geschieht folgendes: F steigt in Richtung unendlich an, irgend etwas gibt nach (die Saite streckt sich, der Steg bewegt sich, usw.), so daß der Hammer anfängt, die Saite zu bewegen, und theta größer als Null wird, was F endlich werden läßt.
 

Diese Vervielfachung der Kraft erklärt, warum ein kleines Kind auf einem Klavier trotz der mehreren hundert Pfund Spannung auf den Saiten einen ziemlich lauten Ton erzeugen kann.
 Es erklärt auch, warum eine normale Person eine Saite beim Klavierspielen zerbrechen kann, besonders wenn die Saite alt ist und ihre Elastizität verloren hat.
 Der Mangel an Elastizität führt dazu, daß F weitaus mehr ansteigt, als wenn die Saite elastischer ist, die Saite kann sich nicht strecken, und theta bleibt nahe Null.
 Diese Situation wird außerordentlich verschärft, wenn der Hammer ebenfalls verdichtet ist, so daß er eine große, flache, harte Kerbe hat, die die Saite berührt.
 In diesem Fall gibt die Oberfläche des Hammers nicht nach, und die anfängliche Kraft \enquote{f} in der obigen Gleichung wird sehr groß.
 Da das bei einem verdichteten Hammer alles nahe theta\nolinebreak=\nolinebreak0 geschieht, wird der Vervielfachungsfaktor der Kraft ebenfalls vergrößert.
 Das Resultat ist eine gebrochene Saite.
 

Die obige Berechnung ist eine starke Vereinfachung und nur qualitativ richtig.
 In Wirklichkeit sendet ein Hammerschlag zunächst eine wandernde Welle in Richtung des Stegs, ähnlich dem was geschieht, wenn man das Ende eines Seils nimmt und es schnalzen läßt.
 Um solche Wellenformen zu berechnen, muß man bestimmte wohlbekannte Differentialgleichungen lösen.
 Der Computer hat die Lösung solcher Differentialgleichungen zu einer einfachen Angelegenheit werden lassen, und realistische Berechnungen dieser Wellenformen können nun routinemäßig erfolgen.
 Deshalb führen die obigen Ergebnisse, obwohl sie nicht genau sind, zu einem qualitativen Verständnis dafür, was geschieht und was die wichtigen Mechanismen und kontrollierenden Faktoren sind.
 

Zum Beispiel zeigt die obige Berechnung, daß es nicht die Energie der Transversalschwingung der Saite ist, sondern die Zugspannung der Saite, die für den Klang des Klaviers verantwortlich ist.
 Die Energie, die durch den Hammer abgegeben wird, wird im gesamten Klavier gespeichert, nicht nur in den Saiten.
 Das ist weitgehend analog zu Pfeil und Bogen: Wenn die Sehne gezogen wird, dann wird die gesamte Energie im Bogen gespeichert, nicht in der Sehne.
 Und die gesamte Energie wird durch die Spannung in den Saiten übertragen.
 In diesem Beispiel ist der mechanische Vorteil und die oben berechnete Vervielfachung der Kraft (nahe theta\nolinebreak=\nolinebreak0) leicht zu sehen.
 Es ist das gleiche Prinzip, auf dem die Harfe basiert.
 

Warum verdichtete Hämmer höhere harmonische Obertöne erzeugen, ist am einfachsten zu verstehen, wenn man erkennt, daß das Auftreffen in kürzerer Zeit stattfindet.
 Wenn es schneller geschieht, generiert die Saite als Antwort auf das schnellere Ereignis Komponenten mit höherer Frequenz.
 

Die obigen Abschnitte machen deutlich, daß ein verdichteter Hammer zunächst einen großen Aufschlag auf den Saiten erzeugt, während ein richtig intonierter Hammer sanfter auf die Saite trifft und somit mehr seiner Energie an die niedrigeren Frequenzen als an die harmonischen Obertöne abgibt.
 Da die gleiche Menge an Energie bei einem verdichteten Hammer in einem kürzeren Zeitraum verteilt wird, kann der anfängliche Lautstärkegrad viel höher als bei einem richtig intonierten Hammer sein, besonders bei den höheren Frequenzen.
 Solche kurzen Tonspitzen können das Gehör schädigen, ohne Schmerzen zu verursachen.
 Verbreitete Symptome solcher Schäden sind Tinnitus (Klingeln im Ohr) und Hörverlust bei hohen Frequenzen.
 Klavierstimmer, die ein Klavier mit solchen abgenutzten Hämmern stimmen müssen, tun gut daran, Ohrenstöpsel zu tragen.
 Es ist klar, daß das Intonieren der Hämmer mindestens genauso wichtig ist wie das Stimmen des Klaviers, besonders weil wir über potentielle Gehörschäden sprechen.
 Ein verstimmtes Klavier mit guten Hämmern schädigt das Ohr nicht.
 Trotzdem lassen viele Klavierbesitzer ihr Klavier zwar stimmen, vernachlässigen aber das Intonieren.
 

\textbf{Die beiden wichtigsten Prozeduren beim Intonieren sind das Wiederherstellen der Form und das Nadeln.}

Wenn der verflachte Auftreffpunkt des Hammers größer als ungefähr 1 cm ist, ist es Zeit, die Form des Hammers wieder herzustellen.
 Beachten Sie, daß Sie zwischen der Länge der Saitenkerbe und dem flachgedrückten Bereich unterscheiden müssen; sogar bei gut intonierten Hämmern können die Kerben mehr als 5 mm lang sein.
 Bei der endgültigen Beurteilung werden sie anhand des Klangs entscheiden müssen.
 Das Formen wird durch das Schleifen der \enquote{Schultern} des Hammers erreicht, so daß er seine ursprüngliche, gerundete Form am Auftreffpunkt wiedergewinnt.
 Das wird üblicherweise mit 1 Zoll \textit{[ca. 2,5 cm]} breiten Streifen Sandpapier ausgeführt, die mit Leim oder doppelseitigem Klebeband auf Holz- oder Metallstreifen befestigt sind.
 Sie könnten mit Papier der Körnung 80 beginnen und zum Schluß Papier der Körnung 150 verwenden.
 Die Schleifbewegung muß in der Ebene des Hammers ausgeführt werden; schleifen Sie niemals quer zur Ebene.
 Es besteht fast nie die Notwendigkeit den Auftreffpunkt zu schleifen.
 Lassen Sie deshalb ungefähr 2 mm vom Zentrum des Auftreffpunkts unberührt.
 

\textit{[Eine detaillierte Beschreibung findet man z.B. in der amerikanischen Ausgabe von Reblitz auf den Seiten 137 bis 140:}
\begin{itemize}
	\item \textit{Schleifen Sie nur an der schmalen umlaufenden Fläche, die durch den Auftreffpunkt (und die Saitenkerben) hindurchgeht.}
	\item \textit{Führen Sie dabei das Schleifpapier immer mit einer bogenförmigen Bewegung vom Stiel zum Auftreffpunkt hin.}
	\item \textit{Der Filz muß auf beiden Seiten des Auftreffpunkts symmetrisch geformt sein, damit die Spitze des Hammers beim Auftreffen auf die Saiten nicht schrittweise in die Richtung der geringeren Unterstützung hin verformt wird und sich der Auftreffpunkt verschiebt.}
	\item \textit{Die Fläche darf nicht nach der Seite abgerundet werden und muß auch rechtwinklig zu den beiden großen Seitenflächen sein, damit die Saiten einer Note gleichzeitig angeschlagen werden.}
	\item \textit{Es muß genügend Filz stehenbleiben, so daß die Saite beim Anschlag nicht den Filz durchschlägt und auf das Holz des Hammers trifft.
 Deshalb soll der Filz der schmalen Hämmer für die hohen Töne überhaupt nicht geschliffen werden.}
\end{itemize}

\textit{\textbf{Also alles andere als einfach und somit nur jemandem mit wirklicher handwerklicher Begabung zu empfehlen, der stets größte Sorgfalt walten läßt!}]}

Nadeln ist nicht einfach, weil die richtige Stelle zum Nadeln und die richtige Tiefe vom jeweiligen Hammer bzw. Hammerhersteller abhängen und davon, wie der Hammer ursprünglich intoniert war.
 Besonders im Diskant werden beim Intonieren der Hämmer in der Fabrik oft Härter wie Lack, usw. benutzt.
 Fehler beim Nadeln sind im allgemeinen nicht rückgängig zu machen.
 Tiefes Nadeln ist üblicherweise an den Schultern unmittelbar außerhalb des Auftreffpunkt erforderlich.
 Sehr sorgfältiges und flaches Nadeln kann im Bereich des Auftreffpunkts nötig werden.
 Der Klang des Klaviers reagiert auf das flache Nadeln am Auftreffpunkt sehr empfindlich, so daß man sehr genau wissen muß, was man tut.
 Wenn er richtig genadelt ist, sollte der Hammer Ihnen erlauben, sowohl sehr leise Töne zu kontrollieren als auch laute Töne zu produzieren, die nicht schrill sind.
 Sie bekommen das Gefühl der totalen klanglichen Kontrolle.
 Sie können nun Ihren Flügel ganz öffnen und ohne das Dämpferpedal sehr leise spielen!
 Sie können auch diese lauten, reichen, respekteinflößenden Töne erzeugen.
 \hypertarget{c2_7b}{}\hypertarget{c2_7_pilo}{}

\subsection{Polieren der Piloten}

\textbf{\textit{{\normalsize [Die Beschreibung des  Aus- und Einbaus der Mechanik und der Tastatur ist z.Zt. (14.2.2005) im Originaltext relativ knapp gehalten.
 Ich erinnere deshalb an dieser Stelle noch einmal an das \enquote{\hyperlink{c2_1}{Achtung: ...}} am Anfang dieses Kapitels!]}}}

Das Polieren der Piloten kann eine lohnende Pflegearbeit sein.
 Sie müssen eventuell poliert werden, wenn Sie mehr als 10 Jahre nicht gereinigt wurden, manchmal auch früher.
 Drücken Sie die Tasten langsam herunter und stellen Sie fest, ob Sie eine Reibung in der Mechanik fühlen können.
 Eine reibungslose Mechanik wird sich anfühlen, als ob man mit einem geölten Finger über ein glattes Glas fährt.
 Wenn Reibung vorhanden ist, fühlt es sich wie die Bewegung eines sauberen Fingers über quietschendes sauberes Glas an.
 Um an die Piloten zu kommen, muß man die Mechanik von den Tasten abheben, indem man bei einem Flügel die Schrauben löst, die die Mechanik unten halten.
 Bei \enquote{\hyperlink{upright}{Aufrechten}} muß man im allgemeinen die Knöpfe losschrauben, die die Mechanik an ihrem Ort halten; stellen Sie sicher, daß die Pedalstangen usw. losgelöst sind.
 

Wenn die Mechanik entfernt wurde, können die Tasten herausgehoben werden, nachdem man die Tastendeckleiste entfernt hat.
 Stellen Sie zuerst sicher, daß alle Tasten numeriert sind, so daß Sie sie wieder in der richtigen Reihenfolge einsetzen können.
 Das ist ein guter Zeitpunkt, um alle Tasten zu entfernen und alle vorher unzugänglichen Bereiche sowie die Seiten der Tasten zu reinigen.
 Sie können ein mildes Reinigungsmittel wie ein mit Xxxxxx befeuchtetes Tuch für das Reinigen der Seiten der Tasten benutzen.
 

Stellen Sie fest, ob die oberen, kugelförmigen Kontaktflächen der Piloten stumpf sind.
 Wenn sie keine glänzende Politur haben, sind sie stumpf.
 Benutzen Sie eine gute Messing-, Bronze- bzw. Kupferpolitur (wie z.B. Xxxxx), um die Kontaktflächen zu polieren und blank zu putzen.
 Bauen Sie alles wieder zusammen, und die Mechanik sollte nun viel leichtgängiger sein.
 