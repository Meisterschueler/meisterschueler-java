%  c31.html 
\hypertarget{c3_1}{}

\chapter{Wissenschaftliche Methode, Theorie des Lernens und das Gehirn}

\section{Einleitung}

Der erste Teil dieses Kapitels beschreibt meine Vorstellung davon, was eine wissenschaftliche Methode ist, und wie ich sie benutzt habe, um dieses Buch zu schreiben.
 Dieser wissenschaftliche Ansatz ist der Hauptgrund, warum sich dieses Buch von allen anderen Büchern über das Thema des Klavierspielenlernens unterscheidet.
 

Die anderen Abschnitte behandeln Themen des Lernens im allgemeinen, und die Gleichung für die Lernrate wird hergeleitet.
 Das ist die Gleichung, die benutzt wurde, um die Lernraten in \hyperlink{c1iv5}{Kapitel 1, Abschnitt IV.5} zu berechnen.
 Ich bespreche auch Themen, die das Gehirn betreffen, weil das Gehirn offensichtlich ein integraler Bestandteil des Spielmechanismus ist.
 Mit Ausnahme der anfänglichen Diskussion darüber, wie sich das Gehirn im Laufe des Älterwerdens entwickelt und wie diese Entwicklung das Lernen des Klavierspielens beeinflußt, haben die Themen über das Gehirn jedoch nur eine geringe direkte Verbindung zum Klavier.
 Die Rolle des Gehirns beim Lernen des Klavierspielens muß natürlich viel mehr erforscht werden.
 Ich habe auch eine Diskussion über die Interpretation von Träumen eingefügt, die mehr Licht in die Frage bringt, wie das Gehirn arbeitet.
 Zum Schluß beschreibe ich meine Erfahrungen mit meinem Unterbewußtsein, welches mir in zahlreichen Fällen gute Dienste geleistet hat.
 \hypertarget{c3_2}{}

\section{Der wissenschaftliche Ansatz}\hypertarget{c3_2a}{}

\subsection{Einleitung}

Dieses Buch wurde mit dem besten mir möglichen wissenschaftlichen Ansatz geschrieben, wobei ich benutzt habe, was ich während meiner 31-jährigen Karriere als Wissenschaftler lernte.
 Ich befaßte mich nicht nur mit Grundlagenforschung (es wurden mir sechs Patente erteilt), sondern auch mit Materialwissenschaft (Mathematik, Physik, Chemie, Biologie, Maschinenbau, Elektronik, Optik, Akustik, Metallen, Halbleitern, Isolatoren), industrieller Problemlösung (Fehlermechanismen, Ausfallsicherheit, Fertigung) und wissenschaftlichen Veröffentlichungen (ich habe über 100 gegengeprüfte Artikel in den meisten großen Wissenschaftsmagazinen veröffentlicht).
 Sogar nachdem ich meinen Doktortitel in Physik von der Cornell University verliehen bekam, investierten meine Arbeitgeber im Laufe meiner Karriere über eine Million Dollar, um meine Ausbildung zu fördern.
 Zurückblickend war diese ganze wissenschaftliche Ausbildung für das Schreiben dieses Buchs unentbehrlich.
 Diese Notwendigkeit, die wissenschaftliche Methode zu verstehen, läßt darauf schließen, daß es den meisten Klavierspielern schwerfallen würde, das gleiche Ergebnis zu erzielen.
 Ich erkläre unten genauer, daß die Ergebnisse wissenschaftlicher Anstrengungen für jeden nützlich sind, nicht nur für Wissenschaftler.
 \textbf{Deshalb bedeutet die Tatsache, daß dieses Buch von einem Wissenschaftler geschrieben wurde, daß jeder in der Lage sein sollte, es leichter zu verstehen, als wenn es nicht von einem Wissenschaftler geschrieben wäre.}
 Ein Ziel dieses Abschnitts ist es, diese Botschaft zu erläutern.
 \hypertarget{c3_2b}{}

\subsection{Lernen}

Klavier, Algebra, Bildhauerei, Golf, Physik, Biologie, Quantenmechanik, Tischlerei, Kosmologie, Medizin, Politik, Wirtschaftswissenschaft usw. - was haben diese gemeinsam?
 Sie sind alle wissenschaftliche Disziplinen und haben deshalb eine große Zahl grundlegender Prinzipien gemeinsam.
 In den folgenden Abschnitten \textbf{ werde ich viele der wichtigen Prinzipien der wissenschaftlichen Methode erklären und zeigen, wie sie für das Erzeugen eines nützlichen Produkts benötigt werden}, z.B. für ein Handbuch zum Lernen des Klavierspielens.
 Diese Erfordernisse für ein Klavierbuch unterscheiden sich nicht von den Erfordernissen für das Schreiben eines fortgeschrittenen Lehrbuchs über Quantenmechanik; die Erfordernisse sind ähnlich, obwohl die Inhalte Welten voneinander entfernt sind.
 Ich beginne mit der Definition der wissenschaftlichen Methode, weil sie von der Öffentlichkeit so oft mißverstanden wird.
 Danach beschreibe ich den Beitrag der wissenschaftlichen Methode zum Schreiben dieses Buchs.
 Bei dieser Gelegenheit stelle ich heraus, wann die Klavierlehre in der Vergangenheit wissenschaftlich oder unwissenschaftlich war.
 Während der letzten Jahrhunderte gab es durch das Anwenden der wissenschaftlichen Methode auf fast alle wichtigen Disziplinen enorme Fortschritte; ist es nicht an der Zeit, daß wir dasselbe mit dem Lernen und Unterrichten des Klavierspielens tun?
 

Dieser Abschnitt wurde hauptsächlich geschrieben, um die wissenschaftliche Methode zu skizzieren, in der Hoffnung, anderen dabei zu helfen, sie auf den Klavierunterricht anzuwenden.
 Ein weiteres Ziel ist, zu erklären, warum es einen Wissenschaftler wie ich einer bin erforderte, um zu einem solchen Buch zu kommen.
 Warum konnten Musiker ohne wissenschaftliche Ausbildung nicht bessere Bücher über das Klavierlernen schreiben?
 Schließlich sind sie die führenden Experten auf diesem Gebiet!
 Ich werde unten ein paar der Antworten darauf geben.
 

Ich vermute, Sie werden beim Lesen der folgenden Abschnitte Konzepte finden, die sich von Ihren Vorstellungen von der Wissenschaft unterscheiden.
 \textbf{Wissenschaft an sich besteht nicht aus Mathematik, Physik oder Gleichungen.
 Sie befaßt sich mit menschlichen Interaktionen, die andere Menschen zu etwas befähigen} (s.u.).
 Ich habe viele \enquote{Wissenschaftler} gesehen, die nicht verstehen was Wissenschaft ist, und deshalb in ihrer eigenen Berufung versagten (d.h. entlassen wurden).
 So wie täglich 8 Stunden zu üben Sie nicht notwendigerweise zu einem vollendeten Pianisten werden läßt, macht Sie das Bestehen aller Physik- und Chemieexamen nicht zu einem Wissenschaftler; Sie müssen etwas mit diesem Wissen vollbringen.
 Ich war von vielen Klaviertechnikern besonders beeindruckt, die ein praktisches Verständnis der Physik haben, obwohl sie kein Wissenschaftsdiplom haben.
 Diese Techniker müssen wissenschaftlich sein, weil das Klavier so tief in der Physik verwurzelt ist.
 So definieren Mathematik, Physik usw. nicht die Wissenschaft (ein verbreitetes Mißverständnis); diese Gebiete haben sich lediglich als nützlich für Wissenschaftler erwiesen, weil sie in einer absolut vorhersagbaren Weise befähigen.
 \textbf{Ich habe vor, Ihnen im folgenden die Ansicht eines Insiders darüber zu zeigen, wie Wissenschaft ausgeführt wird.}

Kann jemand, der keinerlei wissenschaftliche Ausbildung besitzt, das folgende lesen und sofort damit beginnen, den wissenschaftlichen Ansatz zu benutzen?
 Wahrscheinlich nicht.
 Es gibt keinen anderen Weg, als Wissenschaft zu studieren.
 Sie werden sehen, daß die Erfordernisse und Komplexitäten der wissenschaftlichen Methode die meisten Menschen vor unüberwindbare Schwierigkeiten stellen.
 Das ist natürlich eine Erklärung dafür, daß dieses Buch so einmalig ist.
 Sie werden aber zumindest eine Vorstellung davon bekommen, was einige der nützlichen Vorschläge sind, wenn Sie den wissenschaftlichen Ansatz verfolgen möchten.
 

Lassen Sie uns, bevor wir die Definition der Wissenschaft in Angriff nehmen, ein verbreitetes Beispiel dafür untersuchen, wie Menschen die Wissenschaft mißverstehen, weil uns das ermitteln hilft, warum wir eine Definition benötigen.
 Sie können einen Klavier- oder Tanzlehrer sagen hören, daß er ein Gefühl beschreibt, den Flug eines Vogels oder die Bewegung einer Katze, und seine Schüler bekommen sofort auf eine Art eine Vorstellung davon, wie sie spielen oder tanzen müssen, die der Lehrer unmöglich erreicht hätte, wenn er die Bewegung der Knochen, Muskeln, Arme usw. beschrieben hätte.
 Der Lehrer behauptet dann, daß die Vorgehensweise des Künstlers besser ist als die wissenschaftliche.
 Dieser Lehrer bemerkt nicht, daß er wahrscheinlich eine sehr gute wissenschaftliche Methode benutzt hat.
 Indem man eine Analogie herstellt oder das Endprodukt der Musik beschreibt, kann man oft viel mehr Informationen übermitteln als durch das detaillierte Beschreiben jeder Komponente der Bewegung.
 Es ist so, als ob man von Schmalband- zu Breitbandkommunikation übergeht und ist ein gültiges wissenschaftliches Vorgehen; es hat wenig mit der Unterscheidung zwischen Wissenschaft und Kunst zu tun.
 Diese Art von Mißverständnis entsteht oft, weil die Menschen glauben, daß Wissenschaft schwarz oder weiß ist - daß etwas entweder wissenschaftlich ist oder nicht; die meisten Dinge im richtigen Leben sind mehr oder weniger wissenschaftlich, es ist nur eine Frage des Ausmaßes.
 Was diese Lehrmethoden wissenschaftlicher macht oder nicht, hängt davon ab, wie gut sie die notwendigen Informationen übermitteln.
 In dieser Hinsicht sind viele berühmte Künstler, die gute Lehrer sind, Meister dieser Art von Wissenschaft.
 Ein weiteres häufiges Mißverständnis ist, daß Wissenschaft zu schwierig für Künstler sei.
 Das verwundert doch sehr.
 Die geistigen Prozesse, die Künstler beim Erzeugen der höchsten Stufen von Musik oder anderen Künsten durchlaufen, sind mindestens so komplex wie jene von Wissenschaftlern, die über den Ursprung des Universums nachdenken.
 Das Argument, daß die Menschen mit unterschiedlichen Talenten für Kunst oder Wissenschaft geboren werden, mag teilweise gültig sein; ich stimme dieser Ansicht jedoch nicht zu - für den größten Teil der Menschen gilt, daß sie Künstler oder Wissenschaftler sein können, je nachdem in welchem Ausmaß sie, besonders in früher Kindheit, mit jedem Gebiet in Berührung gekommen sind.
 Deshalb haben die meisten Menschen, die gute Musiker sind, die Fähigkeit, große Wissenschaftler zu sein.
 Wenn man sein ganzes Leben Kunst studieren würde, hätte man nicht viel Zeit Wissenschaft zu studieren, wie kann man also beides miteinander kombinieren?
 So wie ich es verstehe, sind Kunst und Wissenschaft komplementär; die Kunst hilft den Wissenschaftlern und umgekehrt.
 Künstler, die der Wissenschaft aus dem Weg gehen, schaden sich nur selbst, und Wissenschaftler, die der Kunst aus dem Weg gehen, neigen dazu, weniger erfolgreiche Wissenschaftler zu sein.
 Was mich in meiner Zeit am College am meisten beeindruckte, war die große Zahl von Wissenschaftsstudenten, die Musiker waren.
 
%  c33.html 
\hypertarget{c3_3}{}

\section{Was ist die Wissenschaftliche Methode?}\hypertarget{c3_3a}{}

\subsection{Einleitung}

Eine häufige falsche Vorstellung ist, daß Klavierspielen eine Kunst ist und deshalb der wissenschaftliche Ansatz nicht möglich und nicht anwendbar sei.
 Diese falsche Vorstellung ist auf ein falsches Verständnis dafür zurückzuführen, was Wissenschaft ist.
 Es mag viele Menschen überraschen, daß Wissenschaft in Wahrheit eine Kunst ist; Wissenschaft und Kunst können nicht voneinander getrennt werden, so wie Klaviertechnik und musikalisches Spielen nicht voneinander getrennt werden können.
 Wenn Sie es nicht glauben, gehen Sie einfach zu irgendeiner großen Universität.
 Sie wird immer eine herausragende Abteilung besitzen: die Abteilung für Kunst und Wissenschaft.
 Beide erfordern Vorstellungskraft, Originalität und die Fähigkeit zur Ausführung.
 Zu sagen, daß eine Person die Wissenschaft nicht kenne und deshalb einen wissenschaftlichen Ansatz nicht benutzen könne, ist so, als ob man sagen würde, daß man, wenn man weniger weiß, weniger lernen sollte.
 Das macht keinen Sinn, weil es genau die Person, die weniger weiß, ist, die mehr lernen muß.
 Offensichtlich müssen wir klar definieren, was Wissenschaft ist.
 \hypertarget{c3_3b}{}

\subsection{Definition}

\textbf{Die einfachste Definition der wissenschaftlichen Methode ist, daß sie jede Methode ist, die funktioniert}.
 Die wissenschaftliche Methode ist eine, die in völliger Harmonie mit der Realität oder Wahrheit ist.
 Wissenschaft ist Befähigung.
 Deshalb ist zu sagen, daß \enquote{Wissenschaft nur etwas für Wissenschaftler ist}, so, als ob man sagen würde, daß Jumbo Jets nur etwas für Luftfahrtingenieure sind.
 Es ist wahr, daß  Flugzeuge nur von Luftfahrtingenieuren gebaut werden können, aber das hindert nicht einen von uns daran, Flugzeuge für unsere Reisen zu benutzen - in Wahrheit sind diese Flugzeuge für uns gebaut worden.
 Genauso ist der Zweck der Wissenschaft, das Leben für alle leichter zu machen, nicht nur für Wissenschaftler.
 

Obwohl kluge Wissenschaftler benötigt werden, um die Wissenschaft voran zu bringen, kann jeder von der Wissenschaft profitieren.
 Deshalb \textbf{ist eine weitere Möglichkeit, Wissenschaft zu definieren, daß sie zuvor unmögliche Aufgaben ermöglicht und schwierige Aufgaben vereinfacht.}
 Von diesem Standpunkt aus nützt Wissenschaft den Unwissenderen unter uns mehr als den besser Informierten, die Dinge selbst herausfinden können.
 Dazu ein Beispiel: Wenn ein Analphabet gebeten würde, zwei sechsstellige Zahlen zu addieren, hätte er keine Möglichkeit, es von selbst zu tun.
 Jeder Drittklässler jedoch, der Rechnen gelernt hat, kann diese Aufgabe ausführen, wenn man ihm einen Stift und Papier gibt.
 Heute kann man dem Analphabeten innerhalb weniger Minuten beibringen, diese Zahlen auf einem Taschenrechner zu addieren.
 Nachweislich hat die Wissenschaft eine zuvor unmögliche Aufgabe für jeden leicht gemacht.
 

Die obigen Definitionen der wissenschaftlichen Methode liefern keine direkte Information darüber, wie man ein wissenschaftliches Projekt durchführt.
 \textbf{Eine praktische Definition des wissenschaftlichen Ansatzes ist, daß er eine Gruppe von eindeutig definierten Objekten und deren Beziehungen zueinander ist}.
 Eine der nützlichsten Beziehungen ist ein Klassifizierungsschema, das Objekte in Klassen und Unterklassen einteilt.
 Beachten Sie, daß das Wort \enquote{definieren} eine sehr spezielle Bedeutung bekommt.
 Objekte müssen in einer solchen Art definiert werden, daß sie nützlich sind und auf eine solche Art, daß die Beziehungen zwischen ihnen präzise beschrieben werden können.
 Und all diese Definitionen und Beziehungen müssen wissenschaftlich korrekt sein (hierbei bekommen Nichtwissenschaftler Probleme).
 

Lassen Sie uns ein paar Beispiele ansehen.
 Musiker haben grundlegende Objekte, wie z.B. \hyperlink{c1iii5a}{Tonleitern}, \hyperlink{c1iii7e}{Akkorde}, Harmonien, Verzierungen usw., definiert.
 In diesem Buch wurden wichtige Konzepte, wie z.B. \hyperlink{c1ii7}{Üben mit getrennten Händen}, \hyperlink{c1ii9}{Akkord-Anschlag}, \hyperlink{c1ii11}{parallele Sets}, \hyperlink{c1ii5}{abschnittsweises Üben}, \hyperlink{c1ii15}{automatische Verbesserung der Fähigkeiten nach dem Üben (PPI)} usw., präzise definiert.
 Damit diese wissenschaftliche Methode, dieses Buch zu schreiben, funktioniert (d.h. damit ein nützliches Lehrbuch herauskommt), ist es notwendig, alle nützlichen Beziehungen zwischen diesen Objekten zu kennen.
 Insbesondere ist es wichtig, vorauszusehen was der Leser \textit{benötigt}.
 Der Akkord-Anschlag wurde als Antwort auf eine Notwendigkeit zur Lösung eines Geschwindigkeitsproblems definiert.
 Man kann hier sehen, warum die Physik nicht so wichtig ist wie die menschliche Befähigung.
 Ich habe verschiedene Bücher gelesen, die das Staccato besprechen, ohne es jemals zu definieren.
 Die Wissenschaft spielt bereits auf den grundlegendsten Stufen der Definitionen, Erklärungen und Anwendungen eine Rolle.
 Der Autor muß bestens mit den besprochenen Themen vertraut sein, damit er keine fehlerhaften Aussagen macht.
 Das ist der Kern der Wissenschaft, nicht Mathematik oder Physik.
 

Eines der Probleme mit \hyperlink{Whiteside}{Whitesides Buch} ist der Mangel an präzisen Definitionen.
 Sie benutzt viele Worte und Konzepte, wie z.B. \hyperlink{c1iii1b}{Rhythmus} und \hyperlink{c1iii8}{Konturieren}, ohne sie zu definieren.
 Das macht es für den Leser schwierig, zu verstehen was sie sagt oder ihre Anweisungen anzuwenden.
 Natürlich mag es zunächst unmöglich erscheinen, diese komplexen Konzepte, auf die wir in der Musik oft treffen, zu definieren, besonders wenn man alle Komplexitäten und Nuancen im Umfeld eines schwierigen Konzeptes einschließen möchte.
 Es ist jedoch die normale wissenschaftliche Vorgehensweise, Bestimmungsgrößen zu benutzen, um die Definition zu begrenzen, wenn man bestimmte Beispiele benutzt und andere Bestimmungsgrößen, um die Definition auf andere Möglichkeiten auszudehnen.
 Es ist nur eine Frage sowohl des Verständnisses des Themas als auch der Bedürfnisse des Lesers.
 \hyperlink{Fink}{Finks} und \hyperlink{Sandor}{Sandors} Buch bieten Beispiele von ausgezeichneten Definitionen.
 Was ihnen fehlt, sind die Beziehungen: ein systematischer, strukturierter Ansatz, wie man diese Definitionen benutzt, um die Technik Schritt für Schritt zu erwerben.
 Sie haben auch ein paar der wichtigen Definitionen vergessen, die in diesem Buch enthalten sind.
 

Der Hauptbestandteil der wissenschaftlichen Methode ist Wissen, aber Wissen alleine ist nicht genug.
 Dieses Wissen muß in eine Struktur gebracht werden, so daß wir die Beziehungen zwischen den Objekten sehen, verstehen und ausnutzen können.
 Ohne diese Beziehungen weiß man nicht, ob man alle notwendigen Teile hat oder gar wie man sie benutzt.
 So sind z.B. \hyperlink{c1ii11}{parallele Sets} ziemlich nutzlos, solange man das HS-Üben nicht kennt.
 Die häufigste Methode, diesen Überbau herzustellen, ist ein Klassifizierungsschema.
 In diesem Buch werden die verschiedenen Verfahren in Anfängermethoden, mittlere Stufen des Lernens, Methoden zum Auswendiglernen, Methoden zur Steigerung der Geschwindigkeit, schlechte Angewohnheiten usw. eingeteilt.
 Hat man erst einmal die Definitionen und das Klassifizierungsschema, muß man anschließend die Details darüber, wie alles zusammengehört und ob es fehlende Elemente gibt, hinzufügen.
 Wir besprechen nun einige besondere Komponenten der wissenschaftlichen Methode.
 \hypertarget{c3_3c}{}

\subsection{Forschung}

Ein Handbuch über das Klavierspielen ist im Grunde eine Liste von Entdeckungen, wie man einige technische Probleme löst.
 Es ist ein Produkt der Forschung.
 In der wissenschaftlichen Forschung führt man Experimente durch, sammelt die Daten und schreibt die Resultate auf eine Art nieder, daß andere verstehen können, was man getan hat, und die Resultate reproduzieren können.
 Klavierspielen zu lehren ist nicht anders.
 Man muß zunächst verschiedene Übungsmethoden erforschen, die Resultate sammeln und sie aufschreiben, so daß andere davon profitieren können.
 Klingt trivial einfach.
 Aber wenn man sich umschaut, ist das \textit{nicht} das, was in bezug auf den Klavierunterricht geschehen ist.
 Liszt hat seine Übungsmethoden niemals schriftlich festgehalten.
 Die \enquote{intuitive Methode} (wie sie in diesem Buch beschrieben wird) erfordert keine Forschung; sie ist die am wenigsten informierte Art zu üben.
 Deshalb war \hyperlink{Whiteside}{Whitesides Buch} so erfolgreich - sie führte Forschungen durch und hielt ihre Ergebnisse fest.
 Leider hatte sie keine wissenschaftliche Ausbildung und versagte bei wichtigen Aspekten, wie z.B. einem klaren, kurzen Schreibstil (besonders bei den Definitionen) und der Ordnung (Klassifizierung und Beziehungen).
 Wenn es uns gelingt, diese Unzulänglichkeiten zu korrigieren, dann besteht natürlich einige Hoffnung, daß wir wissenschaftliche Methoden auf das Lehren des Klavierspielens anwenden können.
 Offensichtlich wurde von allen großen Pianisten ein enormes Maß an Forschung durchgeführt.
 Unglücklicherweise wurde sehr wenig davon dokumentiert; es fiel dem unwissenschaftlichen Ansatz der Klavierpädagogik zum Opfer.
 \hypertarget{c3_3d}{}

\subsection{Dokumentation und Kommunikation}

Das oberste Ziel der Dokumentation ist die Aufzeichnung allen Wissens auf einem Gebiet - es ist ein unschätzbarer Verlust, daß Bach, Chopin, Liszt usw. ihre Übungsmethoden nicht niedergeschrieben haben.
 Eine weitere Funktion der wissenschaftlichen Dokumentation ist das Eliminieren von Fehlern.
 Eine korrekte Idee, die von einem großen Meister formuliert und mündlich von den Lehrern an die Schüler weitergegeben wurde, ist fehleranfällig und völlig unwissenschaftlich.
 Wenn die Idee niedergeschrieben ist, kann man sie auf ihre Genauigkeit überprüfen und alle Fehler beseitigen sowie neue Erkenntnisse hinzufügen.
 D.h., Dokumentation erzeugt eine Einbahnstraße, bei der sich eine Idee im Laufe der Zeit in ihrer Genauigkeit nur verbessern kann.
 

Eine Erkenntnis, die sogar Wissenschaftler überrascht hat, ist, daß ungefähr die Hälfte aller neuen Entdeckungen nicht während der Durchführung der Forschungen gemacht werden, sondern wenn die Resultate niedergeschrieben werden.
 Aus diesem Grund hat sich das wissenschaftliche Schreiben zu einem Gebiet mit besonderen Erfordernissen entwickelt, die so beschaffen sind, daß nicht nur die Fehler minimiert werden, sondern auch der Entdeckungsprozeß maximiert wird.
 Während des Schreibens dieses Buchs entdeckte ich die Erklärung für die \hyperlink{c1iv2b}{Geschwindigkeitsbarrieren}.
 Ich war damit konfrontiert, etwas über Geschwindigkeitsbarrieren zu schreiben und begann mich natürlich zu fragen, was sie sind und was sie erzeugt.
 Es ist wohlbekannt, daß man, wenn man erst einmal die richtigen Fragen stellt, auf dem besten Weg ist, eine Antwort zu finden.
 Ähnlich wurde das Konzept der \hyperlink{c1iii7b}{parallelen Sets} mehr während des Schreibens entwickelt als während meiner Forschungen (Bücher lesen, mit Lehrern sprechen und das Internet benutzen) und persönlicher Experimente am Klavier.
 Das Konzept der parallelen Sets wurde jedesmal benötigt, wenn bestimmte Übungsverfahren zu Schwierigkeiten führten.
 Deshalb wurde es notwendig, dieses Konzept präzise zu definieren, damit man es wiederholt bei so vielen Gelegenheiten benutzen kann.
 

Es ist wichtig, mit allen anderen Wissenschaftlern, die ähnliche Arbeiten durchführen, zu kommunizieren und jegliche neuen Resultate der Forschung offen zu diskutieren.
 In dieser Hinsicht war die Klavierwelt beklagenswert unwissenschaftlich.
 Die meisten Bücher über das Klavierspielen haben nicht einmal ein Quellenverzeichnis (einschließlich der ersten Ausgabe meines Buchs, weil es innerhalb einer begrenzten Zeit geschrieben wurde - diese Unzulänglichkeit wurde in dieser zweiten Ausgabe korrigiert), und sie bauen selten auf den bisherigen Arbeiten von anderen auf.
 Lehrer an den bedeutenden Musikinstitutionen kommen der Aufgabe zu kommunizieren besser nach als private Lehrer, weil sie an einer Institution versammelt sind und zwangsläufig in Kontakt kommen.
 Als Folge davon ist die Klavierpädagogik an einer solchen Institution der der meisten privaten Lehrer überlegen.
 Zu viele Klavierlehrer sind in bezug auf das Annehmen oder Erforschen verbesserter Lehrmethoden inflexibel und stehen oftmals allem kritisch gegenüber, das von \textit{ihren} Methoden abweicht.
 Das ist eine sehr unwissenschaftliche Situation.
 

Beispiele der offenen Kommunikation in meinem Buch sind das miteinander Verflechten der Konzepte von: den \hyperlink{c1ii10}{Armgewichtsmethoden} und der \hyperlink{c1ii14}{Entspannung} (Ansatz nach der Art von Taubman), Ideen aus \hyperlink{Whiteside}{Whitesides Buch} (Kritik an den Übungen der Art von \hyperlink{c1iii7h}{Hanon} und der Methode des Daumenuntersatzes), Einschluß der verschiedenen von \hyperlink{Sandor}{Sandor} usw. beschriebenen Handbewegungen.
 Da das Internet die absolute Form der offenen Kommunikation ist, ist das Aufkommen des Internets eventuell das wichtigste Ereignis, das am Ende dazu führen wird, daß die Klavierpädagogik wissenschaftlicher durchgeführt wird.
 Dafür gibt es kein besseres Beispiel als dieses Buch.
 

Ein Mangel an Kommunikation ist offensichtlich die Hauptursache, warum so viele Klavierlehrer immer noch die intuitive Methode lehren, obwohl die meisten der in diesem Buch beschriebenen Methoden während der letzten zweihundert Jahre von dem einen oder anderen Lehrer gelehrt wurden.
 Wenn der wissenschaftliche Ansatz der völlig offenen Kommunikation und der richtigen Dokumentation von der Klavierlehrergemeinde früher angenommen worden wäre, dann wäre die jetzige Situation sicher eine ganz andere und eine große Zahl Klavierschüler würde mit Raten lernen, die im Vergleich zu den heutigen Standards unglaublich erscheinen.
 

Beim Schreiben der ersten Ausgabe meines Buchs wurde mir die Wichtigkeit der richtigen Dokumentation und des Ordnens der Ideen durch die Tatsache demonstriert, daß ich, obwohl ich die meisten Ideen in meinem Buch bereits ungefähr 10 Jahre kannte, nicht in vollem Umfang von ihnen profitieren konnte, bis ich dieses Buch fertiggestellt hatte.
 Mit anderen Worten: Nachdem ich das Buch fertiggestellt hatte, las ich es erneut und probierte es systematisch aus.
 Dann erst erkannte ich, wie effektiv die Methode war!
 Obwohl ich die meisten Bestandteile der Methode kannte, gab es offenbar einige Lücken, die erst gefüllt wurden, als ich damit konfrontiert wurde, alle Ideen in eine nützliche und organisierte Struktur zu bringen.
 Es ist so, als ob ich alle Einzelteile eines Autos hätte, sie aber solange nutzlos wären, bis ein Mechaniker sie zusammenbaut und das Auto einstellt.
 

So verstand ich z.B. nicht ganz, warum die Methode so schnell war (1000mal schneller als die intuitive Methode), bevor ich nicht die Berechnung der Lernrate durchgeführt hatte \hyperlink{c1iv5}{(s. Kapitel 1, Abschnitt IV.5)}.
 Ich führte die Berechnungen zunächst aus Neugierde aus, weil ich hoffte, ein Kapitel über die Lerntheorie zu schreiben.
 Tatsächlich dauert es fast ein Jahr, bis ich mich selbst überzeugen konnte, daß die Berechnung ungefähr richtig war - eine Lernrate von 1000mal schneller schien zunächst ein lachhaft absurdes Ergebnis zu sein, bis ich feststellte, daß Schüler, die die intuitive Methode benutzen, oftmals während ihres ganzen Lebens nicht über die Mittelstufe hinauskommen, während andere in weniger als zehn Jahren zu Konzertpianisten werden können.
 Die meisten Menschen neigten dazu, solche Unterschiede der Lernrate dem Talent zuzuschreiben, was nicht zu meinen Beobachtungen paßte.
 Ein Nebenprodukt dieser Berechnung war ein besseres Verständnis dafür, \textit{warum} die Methode schneller war, weil man keine Gleichung schreiben kann, ohne zu wissen, welche physikalischen Prozesse beteiligt sind.
 Als die mathematischen Formeln mir verrieten, welche Teile die Lernrate am meisten beschleunigten, konnte ich effektivere Übungsmethoden entwickeln.
 

Ein erstklassiges Beispiel einer neuen Entdeckung, die aus dem Schreiben dieses Buchs resultierte, ist das Konzept der \hyperlink{c1ii11}{parallelen Sets}.
 Ohne dieses Konzept fand ich es unmöglich, alle Ideen auf eine stimmige Weise zusammenzustellen.
 Als das Konzept der parallelen Sets eingeführt war, führte es natürlich zu den \hyperlink{c1iii7b}{Übungen für parallele Sets}.
 Nichts davon wäre geschehen, wenn ich das Buch nicht geschrieben hätte, obwohl ich Übungen für parallele Sets die ganze Zeit benutzt hatte, ohne es bewußt wahrzunehmen.
 Das kommt daher, daß der \hyperlink{c1ii9}{Akkord-Anschlag} eine primitive Form der Übungen für parallele Sets ist; sogar \hyperlink{Whiteside}{Whiteside} beschreibt Methoden für das Üben des \hyperlink{c1iii3}{Trillers}, die im Grunde Übungen für parallele Sets sind.
 \hypertarget{c3_3e}{}

\subsection{Konsistenzprüfungen}

Viele wissenschaftliche Entdeckungen werden als Resultat von Konsistenzprüfungen gemacht.
 Diese Prüfungen funktionieren folgendermaßen.
 Nehmen Sie an, Sie würden 10 Fakten über Ihr Experiment kennen, und Sie entdecken ein elftes.
 Sie haben nun die Möglichkeit, dieses neue Ergebnis gegen alle alten Resultate zu prüfen, und oftmals führt diese Prüfung zu einer weiteren Entdeckung.
 Eine einzige Entdeckung kann ohne jegliche weitere Experimente potentiell zu 10 weiteren Ergebnissen führen.
 Die neuen Methoden dieses Buchs brachten z.B. ein viel schnelleres Lernen hervor, was dann darauf schließen ließ, daß die intuitive Methode Übungsverfahren beinhalten muß, die in Wahrheit das Lernen behindern.
 Mit diesem Wissen wurde es eine einfache Sache, Gesichtspunkte der intuitiven Methode zu finden, die den Fortschritt verlangsamen.
 Diese Aufdeckung der Schwächen der intuitiven Methode wären fast unmöglich gewesen, wenn man nur die intuitive Methode gekannt hätte.
 Das ist eine Konsistenzprüfung, denn wenn beide Methoden korrekt wären, müßten Sie gleich effektiv sein.
 Solch ein geistiger Prozeß, automatisch von allem auf das man trifft die Konsistenz zu prüfen, mag vielen Menschen nicht selbstverständlich erscheinen.
 Als Wissenschaftler hatte ich das jedoch während meiner Laufbahn aus schierer Notwendigkeit bewußt getan.
 

Konsistenzprüfungen sind der ökonomischste und schnellste Weg, Fehler zu finden und neue Entdeckungen zu machen, weil man neue Ergebnisse erhält, ohne weitere Experimente durchzuführen.
 Es kostet wenig extra, außer Ihrer Zeit.
 Sie können nun sehen, warum der Prozeß des Dokumentierens so produktiv sein kann.
 Jedesmal, wenn ein neues Konzept eingeführt wird, kann es gegen alle anderen bekannten Konzepte des Klavierübens geprüft werden, um potentiell zu neuen Ergebnissen zu führen.
 Die Methode ist wegen der großen Zahl der Fakten, die bereits bekannt sind, mächtig.
 Lassen Sie uns annehmen, daß man diese bekannten Wahrheiten zählen könnte und es 1000 wären.
 Dann bedeutet eine neue Entdeckung, daß man nun 1000 weitere Möglichkeiten hat, um zu prüfen, ob sich neue Entdeckungen daraus ergeben!
 

Konsistenzprüfungen sind für das Eliminieren von Fehlern am wichtigsten und wurden benutzt, um Fehler in diesem Buch zu minimieren.
 Langsames Üben ist z.B. sowohl nützlich als auch schädlich.
 Diese Inkonsistenz muß irgendwie beseitigt werden; das geschieht durch sorgfältiges Definieren derjenigen Bedingungen, die langsames Üben erfordern (Auswendiglernen, HT-Üben), und der Bedingungen, unter denen langsames Üben abträglich ist (intuitive Methode ohne HS-Üben).
 Klar ist jedes Pauschalurteil, das sagt \enquote{Langsames Üben ist gut, weil immer schnell zu spielen zu Problemen führt.}, nicht mit allen bekannten Fakten konsistent.
 Wann immer ein Autor eine falsche Behauptung aufstellt, ist eine Konsistenzprüfung oft der leichteste Weg, diesen Fehler herauszufinden.
 \hypertarget{c3_3f}{}

\subsection{Grundlegende Theorie}

Wissenschaftliche Resultate müssen immer auf einer Theorie oder einem Prinzip basieren, das durch andere verifiziert werden kann.
 Sehr wenige Konzepte stehen allein, unabhängig von allem anderen.
 Mit anderen Worten: Für alles, von dem jemand behauptet, daß es funktioniert, muß es eine gute Erklärung geben, warum es funktioniert; anderenfalls ist es suspekt.
 Erklärungen wie \enquote{Es hat bei mir funktioniert." oder "Ich habe das 30 Jahre lang so unterrichtet." oder sogar "Das ist, wie Liszt es getan hat.} sind einfach nicht gut genug.
 Wenn ein Lehrer ein Verfahren 30 Jahre unterrichtet hat, sollte er genügend Zeit gehabt haben, herauszufinden, warum es funktioniert.
 Die \textit{Erklärungen} sind oft wichtiger als die Verfahren, die sie erklären.
 HS-Üben funktioniert z.B., weil es eine schwierige Aufgabe vereinfacht.
 Wenn dieses Prinzip der Vereinfachung eingeführt ist, kann man nach weiteren Dingen dieser Art Ausschau halten, wie z.B. schwierige Passagen zu kürzen oder das \hyperlink{c1iii8}{Konturieren}.
 Ein Beispiel für eine grundlegende Erklärung ist der Zusammenhang zwischen der Schwerkraft und der Armgewichtsmethode und ihrer Beziehung zum Tastengewicht.
 Im Beispiel der schweren Hand des Sumoringers und der leichten Hand des Kindes \hyperlink{c1ii10}{(Kapitel 1, II.10)} müssen beide bei einem korrekten Anschlag mit Freiem Fall einen Ton gleicher Intensität erzeugen, wenn ihre Hände aus der gleichen Höhe auf das Klavier herunterfallen.
 Das ist offensichtlich  für den Sumoringer wegen seiner Neigung, sich auf das Klavier zu stützen, um seine schwerere Hand anzuhalten, schwieriger.
 Deshalb ist der korrekte Freie Fall für den Sumoringer schwieriger auszuführen.
 Diese Feinheiten auf theoretischer Grundlage zu verstehen führt zur Ausführung eines wirklich korrekten Freien Falls.
 Mit anderen Worten: Bei einem korrekten Freien Fall darf man sich nicht auf dem Klavier abstützen, um die Hand anzuhalten, bis der Anschlag vollständig ist.
 Man braucht ein sehr geschmeidiges Handgelenk, um diese Meisterleistung zu vollbringen.
 

Selbstverständlich gibt es immer ein paar Konzepte, die sich der Erklärung widersetzen, und es ist extrem wichtig, sie klar als \enquote{gültige Prinzipien ohne Erklärungen} zu klassifizieren.
 Wie können wir in diesen Fällen wissen, daß sie gültig sind?
 Sie können nur als gültig angesehen werden, nachdem man eine unbestreitbare Aufzeichnung der experimentellen Überprüfung erstellt hat.
 Es ist wichtig, diese klar zu kennzeichnen, weil Verfahren ohne Erklärungen schwieriger anzuwenden sind und diese Verfahren sich während wir dazulernen und sie besser verstehen ändern.
 Das beste an den Methoden, für die es gute Erklärungen gibt, ist, daß man uns nicht jedes Detail, wie man das Verfahren durchführt, sagen muß - wir können die Details oft anhand unseres eigenen Verständnisses der Methode selbst einfügen.
 

Leider ist die Geschichte der Klavierpädagogik voller Verfahren für das Erwerben der Technik, die keine theoretische Grundlage haben, die aber trotzdem eine breite Akzeptanz erfahren haben.
 Die \hyperlink{c1iii7h}{Hanon-Übungen} sind das beste Beispiel dafür.
 Die meisten Anweisungen, wie man etwas tun soll, die ohne eine Erklärung dafür gegeben werden, warum sie funktionieren, haben in einem wissenschaftlichen Ansatz einen geringen Wert.
 Das nicht nur wegen der hohen Wahrscheinlichkeit, daß solche Verfahren falsch sind, sondern auch, weil es die Erklärung ist, die dabei hilft, das Verfahren korrekt anzuwenden.
 Weil es keine theoretische Grundlage für die Hanon-Übungen gibt, wenn er uns ermahnt, \enquote{die Finger stark anzuheben} und \enquote{eine Stunde täglich zu üben}, können wir in keinster Weise wissen, ob diese Verfahren tatsächlich hilfreich sind.
 In jedem Verfahren des täglichen Lebens ist es für jeden fast unmöglich, alle notwendigen Schritte eines Verfahrens für alle denkbaren Fälle zu beschreiben.
 Es ist ein Verständnis dafür, warum es funktioniert, das jedem gestattet, das Verfahren abzuändern, damit es den besonderen Bedürfnissen des einzelnen und der sich ändernden Umstände gerecht wird.
 

So empfehlen z.B. Lehrer, die die intuitive Methode benutzen, daß man das Spielen langsam und genau anfängt und die Geschwindigkeit schrittweise steigert.
 Andere Lehrer mögen das langsame Spielen so weit wie möglich zu unterbinden suchen, weil es eine solche Zeitverschwendung ist.
 Keines dieser Extreme ist das Beste.
 Das \hyperlink{c1ii16}{langsame Spielen des intuitiven Ansatzes} ist unerwünscht, weil man eventuell Bewegungen verfestigt, die das schnellere Spielen stören.
 Auf der anderen Seite ist langsames Spielen, wenn man erst einmal mit der endgültigen Geschwindigkeit spielen kann, sehr nützlich für das \hyperlink{c1iii6h}{Auswendiglernen} und für das Üben der \hyperlink{c1ii14}{Entspannung} und Genauigkeit.
 Deshalb ist die einzige Möglichkeit, die richtige Übungsgeschwindigkeit auszuwählen, im Detail zu verstehen, warum man diese Geschwindigkeit nehmen muß.
 In diesem Zeitalter der Informationstechnologie und des Internets sollte es fast keinen Platz mehr für blindes Vertrauen geben.
 

Das heißt nicht, daß es Regeln ohne Erklärungen nicht gibt.
 Schließlich gibt es immer noch viele Dinge in dieser Welt, die wir nicht verstehen.
 Beim Klavierspielen ist die Regel, \hyperlink{c1ii17}{vor dem Aufhören langsam zu spielen}, ein Beispiel dafür.
 Es muß eine gute Erklärung geben, aber ich habe noch keine gehört, die ich für zufriedenstellend halte.
 In der Wissenschaft sind Paulis Ausschließungsprinzip \textit{[oder kurz Pauli-Prinzip]} (zwei Fermionen können nicht die gleichen Quantenzahlen haben) und die Heisenbergsche Unschärferelation Beispiele von Regeln, die nicht von einem tieferen Prinzip abgeleitet werden können.
 Deshalb ist es genauso wichtig, etwas zu verstehen, wie zu wissen, was wir nicht verstehen.
 Die sachkundigsten Physikprofessoren sind diejenigen, die alle Dinge benennen können, die wir immer noch nicht verstehen.
 \hypertarget{c3_3g}{}

\subsection{Dogma und Lehre}

Wir wissen alle, daß man nicht jede Regel brechen kann, von der man glaubt, sie brechen zu können, und immer noch musikalisch spielen kann, es sei denn, man hat Initialen wie LvB.
 Die dogmatischen Lehrmethoden, die in der Klavierpädagogik so weit verbreitet sind, haben sich in diesem restriktiven Umfeld der Schwierigkeit, Schüler zum Erzeugen von Musik anzuleiten, entwickelt.
 Um es zynisch zu sagen: Der dogmatische Ansatz ist ein angenehmer Weg, die Unwissenheit des Lehrers dadurch zu verbergen, daß alles unter den Dogma-Teppich gekehrt wird.
 Alle großen Vorträge, die ich von berühmten Künstlern gehört habe, sind voller exzellenter wissenschaftlicher Erklärungen, warum man auf eine bestimmte Art vorspielen oder nicht vorspielen sollte.
 Es sind jedoch nicht alle großen Künstler auch gute Lehrer oder in der Lage, zu erklären was sie tun.
 Die Lektion daraus ist für die Schüler, daß sie im allgemeinen nichts akzeptieren sollten, das sie nicht verstehen können; das wird dazu führen, daß die Ausbildungsstufe, die sie erreichen, ansteigt.
 Ich bin überzeugt, daß sogar die Interpretation der Musik mit der Zeit ebenfalls wissenschaftlicher wird, genauso wie die Alchemie sich schließlich zur Chemie entwickelte.
 

Leider ist ein dogmatischen Herangehen an das Unterrichten nicht immer ein Zeichen für einen schlechteren Lehrer.
 In Wahrheit scheint es, vermutlich aus historischen Gründen, eher das Gegenteil zu sein.
 Zum Glück sind viele gute junge Lehrer, und besonders diejenigen an großen Institutionen, weniger dogmatisch - sie können erklären.
 Wenn die Lehrer besser ausgebildet sind, sollten sie in der Lage sein, Dogma vermehrt durch ein tieferes Verständnis für die zugrunde liegenden Prinzipien zu ersetzen.
 Das sollte die Effizienz und die Leichtigkeit des Lernens für den Schüler deutlich verbessern.
 

Den meisten Menschen ist bewußt, daß Wissenschaftler ihr ganzes Leben lang lernen müssen, nicht nur wenn sie an der Universität für ihre Abschlüsse arbeiten.
 Den meisten ist jedoch nicht bewußt, in welchem Ausmaß Wissenschaftler ihre Zeit der Ausbildung widmen, nicht nur um zu lernen, sondern auch um alle anderen zu unterrichten, insbesondere andere Wissenschaftler.
 Tatsächlich muß, um das Maß der Entdeckungen zu maximieren, die Ausbildung zu einer ganztägigen, alles verschlingenden Passion werden.
 Wissenschaftler entwickeln sich deshalb oftmals mehr zu Lehrern als z.B. Klavier- oder Schullehrer, sowohl wegen des breiteren Bereichs an \enquote{Schülern}, auf die sie treffen, als auch wegen der Breite der Themen, die sie abdecken müssen.
 Es ist wirklich erstaunlich, wie viel man wissen muß, um nur eine kleine neue Entdeckung zu machen.
 Deshalb muß ein notwendiger Teil der wissenschaftlichen Dokumentation die höchsten Fertigkeiten des Unterrichtens einschließen.
 Ein wissenschaftlicher Forschungsbericht ist nicht so sehr eine Dokumentation dessen, was getan wurde, als vielmehr ein Lehrbuch darüber, wie man das Experiment reproduziert und die zugrunde liegenden Prinzipien versteht.
 Deshalb ist die wissenschaftliche Methode für das Unterrichten ideal.
 Und es ist eine Lehrmethode, die zur dogmatischen Methode diametral verschieden ist.
 \hypertarget{c3_3h}{}

\subsection{Schlußfolgerungen}

Der wissenschaftliche Ansatz ist mehr als nur eine präzise Art, die Ergebnisse eines Experiments zu dokumentieren.
 Er ist ein Verfahren, das zur Beseitigung von Fehlern und zur Erzeugung von Entdeckungen entwickelt wurde.
 Vor allem ist er im Grunde ein Mittel zur Befähigung des Menschen.
 Wenn der wissenschaftliche Ansatz früher übernommen worden wäre, dann wäre die Klavierpädagogik heute aller Wahrscheinlichkeit nach völlig anders.
 Das Internet wird sicherlich die Übernahme von wissenschaftlicheren Vorgehensweisen in das Lernen des Klavierspielens beschleunigen.
 \hypertarget{c3_4}{}

\section{Theorie des Lernens}

\textit{[Abschnitt 4 ist im Original z.Zt. (30.5.2005) noch \enquote{preliminary draft} also ein \enquote{Rohentwurf}.]}

Ist es nicht seltsam, daß wir, wenn wir auf die Universität gehen, finden, daß \enquote{101 Lernen} kein erforderlicher Kurs ist (wenn er überhaupt existiert!)?
 Von Colleges und Universitäten erwartet man, daß sie Lernzentren sind.
 Psychologische Abteilungen haben oft einführende Kurse über Studiengewohnheiten usw., aber man sollte meinen, daß die Wissenschaft des Lernens der erste Punkt auf der Tagesordnung an jedem Lernzentrum wäre.
 Beim Schreiben dieses Buchs fand ich, daß es notwendig ist, über den Lernprozeß nachzudenken und eine - wie auch immer näherungsweise - Gleichung für die Lernrate abzuleiten.
 
%  c35.html 
\hypertarget{c3_5}{}

\section{Was Träume erzeugt und Methoden zu ihrer Kontrolle}\hypertarget{c3_5a}{}

\subsection{Einleitung}

Dieser Abschnitt hat nichts mit Klavieren zu tun.
 Er ist hier eingefügt, weil er ein wenig Klarheit darüber bringt, wie das Gehirn funktioniert.
 Ich kenne keine Untersuchungen über die Ursachen von Träumen und Methoden für ihre Kontrolle, wie ich sie unten beschreibe.
 Wenn Sie eine solche Quelle kennen, schicken Sie mir bitte eine Mail.
 

Haben Sie jemals wiederkehrende Träume gehabt und sich gefragt, was sie verursacht?
 Oder Alpträume, die Sie gerne losgeworden wären?
 Es scheint so, als hätte ich Antworten auf diese beiden Fragen gefunden und bei dem Prozeß einige Einsichten darüber gewonnen, wie das Gehirn während wir schlafen funktioniert.
 

Die meisten Traumdeuter sind heutzutage wie Handleser.
 Sie bemühen sich, Ihre Zukunft vorherzusagen und schreiben Träumen magische Kräfte oder Botschaften zu, die wundervoll wären, wenn sie wahr wären, aber leider so realistisch sind wie S\'eancen oder Kaffeesatzlesen.
 Ich habe gefunden, daß eine Interpretation der Träume, die auf körperlichen Anzeichen basiert, uns eine Menge darüber sagen kann, wie unser Gehirn funktioniert.
 Ich bespreche hier vier Arten von Träumen, die ich hatte und für die ich eine körperliche Erklärung gefunden habe.
 In Diskussionen mit Freunden habe ich entdeckt, daß viele ähnliche Träume haben und diese, fast mit Sicherheit, ähnliche Ursachen.
 Im letzten Abschnitt bespreche ich, was diese Träume uns darüber sagen, wie unser Gehirn funktioniert.
 Ich bin zu dem Schluß gekommen, daß dieses Herangehen an Träume viel lohnender ist als das der Wahrsager und ähnlicher Traumdeuter.
 Die vier Träume, die unten besprochen werden sind:
 
\begin{itemize}
	\item \hyperlink{c3_5b}{fallen},
	\item \hyperlink{c3_5c}{unfähig sein zu laufen},
	\item \hyperlink{c3_5d}{zu spät zu Besprechungen oder Prüfungen kommen oder unfähig sein, das Ziel zu finden}, und
	\item \hyperlink{c3_5e}{ein langer, komplexer Traum, der für mich spezifisch ist}.
\end{itemize}

Die ersten drei sind, so glaube ich, ziemlich verbreitete Träume, die viele Menschen haben.
 \hypertarget{c3_5b}{}

\subsection{Der Fall-Traum}

In diesem Traum falle ich, nicht von einem bestimmten Ort oder hinunter auf einen bestimmten Platz, aber ich falle definitiv und habe Angst.
 Und ich bin absolut unfähig, den Fall zu stoppen.
 Stets bin ich, wenn ich lande, unverletzt.
 Es gibt keine Schmerzen.
 Tatsächlich fühlt es sich, obwohl ich auf dem Boden aufgeschlagen bin, wie eine weiche Landung an, und der Traum hört immer auf, sobald ich lande.
 Die weiche Landung ist besonders seltsam, denn bei jedem Fall auf fast jede Fläche gibt es im allgemeinen am Ende irgendeine Art von Katastrophe.
 Was würde alle diese Details jenes Traums erklären?
 Ich habe eines Tages die körperliche Ursache dieses Traums entdeckt, als ich unmittelbar nach dem Traum aufwachte und feststellte, daß meine Knie heruntergefallen waren.
 Ich hatte auf dem Rücken geschlafen, hatte beide Knie hochgestellt und als ich meine Beine streckte, führte das Gewicht der Bettdecke dazu, daß meine Füße ausrutschten und die Knie herunterfielen.
 Diese fallenden Knie brachten mein Gehirn dazu, den Fall-Traum zu erzeugen!
 Zunächst war das nur eine hypothetische Erklärung und eine offensichtlich dumme noch dazu.
 Warum konnte mein Gehirn nicht erkennen, daß meine Knie gefallen waren?
 Nachdem die Hypothese aber erst einmal aufgestellt war, konnte ich sie jedesmal überprüfen, wenn ich diesen Traum hatte (über einen Zeitraum von mehreren Jahren), und es gelang mir mehrere Male, sie zu bestätigen.
 Beim Aufwachen konnte ich mich deutlich daran erinnern, daß meine Knie gerade eben heruntergefallen waren.
 Die Tatsache, daß die Knie auf das weiche Bett fallen, erklärt die weiche Landung, und da hinterher nichts passiert, endet der Traum.
 Warum bin ich unfähig, den Fall der Knie zu stoppen?
 Wie weiter unten wiederholt gezeigt wird, haben wir während wir schlafen manchmal eine sehr geringe Kontrolle über unsere Muskeln.
 Nicht nur das, das schlafende Gehirn kann nicht einmal die einfache Tatsache erkennen, daß das Knie fällt.
 Zusätzlich denkt es sich das aus, was normalerweise ein unglaubliches Szenario eines Falls sein sollte, und tatsächlich glaube ich es am Ende.
 Dieser letzte Teil ist der absurdeste, weil ich mich doch tatsächlich selbst hereinlege!
 \hypertarget{c3_5c}{}

\subsection{Der Unfähig-zu-laufen-Traum}

Das ist ein sehr frustrierender Traum.
 Ich möchte laufen, aber ich kann es nicht.
 Es macht keinen Unterschied, ob mich jemand verfolgt oder ob ich bloß schnell irgendwo hinlaufen will; ich kann nicht laufen.
 Wenn man läuft, muß man sich vorwärts beugen.
 Deshalb versuche ich im Traum, mich nach vorne zu beugen, aber ich kann es nicht.
 Irgend etwas schiebt mich fast zurück.
 Im Traum habe ich sogar überlegt, daß wenn ich nicht vorwärts laufen oder mich vorwärts beugen kann, warum dann nicht zurücklehnen oder rückwärts laufen?
 Auf diese Art kann ich mich zumindest bewegen.
 Ich kann mich auch nicht zurücklehnen, meine Füße sind wie gelähmt, und ich komme weder vorwärts noch rückwärts richtig voran.
 Wenn man läuft, muß man zunächst seine Knie nach vorne und oben bringen, so daß man nach hinten treten kann, aber ich kann auch das nicht.
 Was würde ein solches Gefühl auslösen während ich schlafe?
 Ich habe die Ursache dieses Traumes entdeckt, nachdem ich den \hyperlink{c3_5b}{Fall-Traum} gelöst hatte, so daß die Erklärung leichter zu finden war.
 Wieder kam ich auf die Erklärung, als ich unmittelbar nach dem Traum aufwachte und mich selber mit dem Gesicht nach unten auf dem Bauch liegend fand. Heureka!
 Wenn man auf dem Bauch liegt, kann man den Winkel des Körpers zum Bett nicht verändern; man kann sich nicht vorwärts lehnen.
 Man kann auch nicht die Knie nach oben ziehen, weil das Bett im Weg ist.
 Man kann sich nicht zurücklehnen, weil man von der Schwerkraft nach unten gedrückt wird.
 Man kann nicht rückwärts gehen, weil das Bettzeug im Weg ist.
 Das zeigt erneut, daß man während man schläft keine große Kontrolle über die Muskeln hat, denn wenn man wach wäre, dann wäre das Hochziehen der Knie nicht so schwierig, sogar wenn man mit dem Gesicht nach unten liegen würde.
 Nachdem ich die Erklärung gefunden hatte, konnte ich sie wieder mehrmals bestätigen; d.h., als ich wach wurde, lag ich mit dem Gesicht nach unten.
 An diesem Punkt fing ich an zu erkennen, daß vielleicht die meisten unserer Träume eine körperliche Erklärung haben.
 Das Ganze machte jedoch irgendwie keinen Sinn - warum sollte mein Gehirn nicht wissen, daß meine Knie herunterfallen, oder daß ich mit dem Gesicht nach unten schlafe?
 Wie kann mein Gehirn einen so komplexen Traum träumen und trotzdem nicht in der Lage sein, solch einfache Dinge zu erkennen?
 Und wieder hat sich mein Gehirn eine Geschichte ausgedacht und sie mich erfolgreich glauben lassen, während ich träumte.
 \hypertarget{c3_5d}{}

\subsection{Der Zu-spät-zur-Prüfung-kommen- oder Sich-verlaufen-Traum}

Dies ist ein weiterer frustrierender Traum.
 Können Sie sehen, wie ein Muster zum Vorschein kommt?
 Ich werde weiter unten spekulieren, warum Träume dazu neigen, negativ oder alptraumhaft zu sein.
 Dies ist kein bestimmter Traum, sondern eine ganze Klasse von Träumen, in denen ich versuche, zu einer Prüfung oder irgendwo anders hin zu kommen, aber spät dran bin und nicht hingelangen oder es nicht finden kann.
 Ich muß z.B. einen steilen Hang überwinden oder um Gebäude herumlaufen.
 Oder wenn ich in einem Gebäude bin, gehe ich durch einen Irrgarten aus Rampen, Treppen, Türen, Aufzügen usw., aber ich kann noch nicht einmal zum Ausgangspunkt zurück.
 Tatsächlich wird es immer schlimmer und komplexer.
 Nach einer Weile kann ich ziemlich erschöpft sein.
 Dieser Traum könnte auftreten, wenn ich in einer ungünstigen oder unbequemen Position schlafe, aus der ich nicht leicht herauskomme, wie z.B. auf meiner Hand schlafend oder im Laken oder Bettzeug eingewickelt.
 Jede Art von Schlafposition, die unbequem ist, aus der ich gerne herauskommen möchte, es aber nicht leicht tun kann während ich schlafe.
 Wenn ich in den Laken eingewickelt bin, kann ich mich nicht so leicht daraus befreien während ich schlafe, und je mehr ich damit kämpfe, desto mehr verwickle ich mich darin, und es kann sehr anstrengend werden.
 Ich war bisher nicht in der Lage, diese Traumfamilie oder eines seiner Mitglieder direkt mit einer bestimmten Ursache zu verbinden, wie bei den anderen drei Träumen.
 Ich habe jedoch eine leichte Schlafapnoe und das erste Auftreten dieser Art von Träumen fällt damit zusammen, was ich für das erste Auftreten der Schlafapnoe halte.
 Somit könnte der Traum durch meine Unfähigkeit zu atmen verursacht worden sein.
 

Was auch immer die genaue Ursache ist, ob eine unbequeme Position oder Schlafapnoe, so ist klar, daß ich, wenn ich wach gewesen wäre, leicht eine Lösung gefunden hätte.
 Somit ist das Muster, das zum Vorschein kommt, daß mein logisches Denkvermögen und meine Fähigkeit zur Lösung von Problemen stark beeinträchtigt sind; sehr einfache Probleme können mich in die Klemme bringen während ich schlafe.
 \hypertarget{c3_5e}{}

\subsection{Die Lösung für meinen langen und komplexen Traum}

Nachdem ich die Lösung für die drei oben genannten Träume gefunden hatte, war ich überzeugt, daß ein anderer wiederkehrender Traum ebenfalls eine körperliche Ursache hatte.
 Dieser Traum war lang und komplex aber immer derselbe.
 Er fängt angenehm an.
 Ich gehe für eine Klettertour nach draußen, und vor mir ist eine sanfter Hügel oder eine wogende Wiese, die in der Ferne zu einem Berg führen.
 Das erste Anzeichen, daß etwas nicht stimmt, kommt von diesem Berg.
 Er geht mit steilen Felswänden nach oben, und die Spitze ist so hoch, daß ich sie kaum sehen kann.
 Ich mache mich trotzdem auf den Weg, aber sofort tritt eine furchterregende Situation ein: Ich bin an der Kante einer vertikalen Felswand, und ich kann nicht einmal den Boden darunter sehen!
 Ich bekomme Angst, drehe mich sofort um und versuche zurück zu gehen, aber der Vorsprung, auf dem ich weitergehe, wird schmaler, als ob ich auf einem Schwebebalken gehen würde.
 Schließlich merke ich, daß ich fast am Ende bin aber eine letzte Hürde nehmen muß: einen Fluß!
 Bevor ich über Felsen springe, um über den Fluß zu kommen, prüfe ich ihn mit der Hand, und das Wasser ist kalt und tief.
 Ungefähr in diesem Stadium endet der Traum.
 Wie würde ich solch einen komplexen Traum erklären?
 Ich löste das Rätsel wieder, nachdem ich unmittelbar nach dem Traum erwachte.
 Ich hatte am Rand des Betts geschlafen, und eine Hand schaute unter der Bettdecke hervor und hing herunter.
 Nun konnte ich jedes Detail meines Traums erklären!
 Mein Traum beginnt offensichtlich damit, daß ich auf dem Bauch schlafe, mit meinem Kinn auf dem Bett, und ich schaue auf das Kissen vor mir (die wogende Wiese).
 Hinter dem Kissen ist das vertikale hölzerne Kopfende, aus kanadischer Walnuß hergestellt, das wie eine steile Felswand aussieht, welches der Berg ist.
 Mit meinem Kinn auf dem Bett kann ich kaum die Spitze des Kopfendes sehen.
 Bis hierher ist interessant, daß ich offensichtlich Sachen ansehe während ich schlafe.
 Da ich an der Bettkante schlafe, fällt eine Hand über die Kante, und das ist die Kante der Felswand, an der ich stehe.
 Ungefähr sieben Zoll \textit{[$<$ 20 cm]} von meinem Bett entfernt steht mein Nachttisch mit einer schmalen abgestuften Kante wie die Oberseite eines Schwebebalkens (schwer zu beschreiben).
 Meine Hand tastet also offensichtlich herum.
 Da meine Hand nun nicht mehr unter der Bettdecke liegt, fühlt sie sich kalt an (der kalte Fluß). Das ist es!
 Diese Erklärungen tragen jedem Detail meines Traums Rechnung!
 Diese Erklärungen haben mich überzeugt, daß Träume interpretiert werden KÖNNEN, und daß die meisten von ihnen körperliche Ursachen haben.
 Wenn das alles wahr ist, dann sollten wir in der Lage sein, die Ursachen und Erklärungen zu benutzen, um daraus abzuleiten, was das Gehirn während des Schlafens tut.
 Das ist eine aufregende Aussicht, von deren Verwirklichung nicht einmal die Wahrsager und Traumdeuter träumen konnten.
 \hypertarget{c3_5f}{}

\subsection{Die Kontrolle der Träume}

Das erstaunlichste an der Erklärung dieser Träume war, daß ich etwas Kontrolle über sie entwickelte.
 Nachdem ich völlig überzeugt war, daß jede Erklärung richtig ist, verschwanden diese Träume!
 Ich konnte mich nicht mehr selbst hereinlegen!
 Zu denken, daß fallende Knie dasselbe sind wie von einem Dach oder einer Klippe zu stürzen, heißt ganz klar, mich selbst hereinzulegen.
 Wenn der Mechanismus erst einmal verstanden wird, dann wird das Gehirn nicht mehr getäuscht.
 Obwohl das Gehirn hinreichend abgeschaltet ist, so daß es während des Schlafs leicht getäuscht werden kann, hat es demnach genügend Kapazität, um die Wahrheit zu erkennen, wenn der Mechanismus erst bekannt ist.
 

Trotzdem erschien es mir irgendwie weit hergeholt, daß ich mich selbst hereingelegt hatte.
 Um mich selbst davon zu überzeugen, daß diese Art von Täuschung möglich ist, mußte ich ein Beispiel aus dem richtigen Leben finden.
 Glücklicherweise habe ich eins gefunden.
 Es ist das, was Magier tun.
 Wenn man einen Zaubertrick beobachtet, weiß man, daß es keine Zauberei ist, man fällt aber jedesmal in dem Sinn darauf herein, daß es völlig verwirrend und sehr aufregend ist.
 Nun ändert sich die Geschichte gänzlich, wenn Ihnen jemand erklären sollte, wie der Trick funktioniert.
 Dann verschwindet plötzlich das Mysterium und die Spannung, und man konzentriert sich am Ende darauf, wie der Magier den Trick ausführt.
 Man kann nicht dazu verleitet werden zu denken, daß es Zauberei ist.
 Somit kann unser Gehirn in einem Traum so lange getäuscht werden, wie es nicht weiß, daß es getäuscht wird.
 Da die meisten Menschen die Erklärung für den Traum nicht kennen, sind sie sich der stattfindenden Täuschung offensichtlich nicht bewußt, und die Träume gehen weiter.
 Kennt man erst einmal die Ursache des Traums, weiß man auch, daß das Gehirn getäuscht wurde; deshalb ist es dann für das Gehirn viel einfacher, die Wahrheit herauszufinden, und der Traum verschwindet.
 Bevor man die Wahrheit herausgefunden hat, wußte das Gehirn nicht einmal, daß es getäuscht wurde, so daß es keinen Grund hatte, nach der Wahrheit zu suchen.
 Nun scheint alles Sinn zu machen.
 \hypertarget{c3_5g}{}

\subsection{Was uns diese Träume über das Gehirn lehren}

Diese vier Beispiele lassen darauf schließen, daß die meisten Träume einen konkreten körperlichen Ursprung haben.
 Ich habe diese Art von Erklärung niemals zuvor gesehen, trotzdem erscheint alles vernünftig.
 Soweit ich weiß, ist der \hyperlink{c3_5b}{Fall-Traum} weit verbreitet - viele haben diesen Traum.
 Bei mir war es das fallende Knie; bei jemand anderem mag es ein Arm oder ein Bein sein, daß über die Bettkante gleitet.
 

Die obigen Resultate bieten eine Unmenge an Möglichkeiten, über die Funktionsweise des Gehirns zu spekulieren.
 Dazu ein paar Ideen.
 Während des Schlafs ist der größte Teil des Gehirns abgeschaltet, so daß es nicht überraschend ist, wenn das Gehirn leicht getäuscht werden kann.
 Es scheint, daß die höheren Funktionen vollständiger abgeschaltet sind, so daß das logische Denken am stärksten beeinträchtigt ist.
 Es kann sein, daß Angst das Gefühl ist, das beim Einschlafen als letztes abgeschaltet und beim Aufwachen als erstes eingeschaltet wird - wahrscheinlich aus Gründen des Überlebens.
 Wenn ein Feind während des Schlafs angreift, ist Angst das erste Gefühl, das aufgeweckt werden muß.
 Das läßt darauf schließen, daß die meisten Träume tendenziell alptraumhaft sein könnten.
 Aber selbstverständlich kann das von Person zu Person unterschiedlich sein, und einige Menschen können hauptsächlich angenehme Träume haben, je nach der Veranlagung der Person.
 In meinem Fall legen die Anhaltspunkte nahe, daß die Träume, die ich entschlüsselt habe, unmittelbar bevor ich aufgewacht bin aufgetreten sind.
 Das läßt darauf schließen, daß die meisten Träume während der kurzen Zeit zwischen Schlaf und Erwachen auftreten.
 Obwohl es Schlafwandler gibt, die ihre starken Muskeln während des Schlafs kontrollieren können, zeigt das oben gesagte, daß die Bemühungen, während eines Traums die Muskeln zu bewegen, nicht gut in tatsächliche Bewegungen umgesetzt werden.
 Das \hyperlink{c3_5e}{vierte Beispiel} zeigt jedoch, daß man sich während des Schlafs viel bewegt - zusätzlich zu den normalen Bewegungen, die notwendig sind, um den Körper periodisch in eine andere Lage zu bringen, damit man einen ausgedehnten Durchblutungsmangel an den Stellen, an denen man aufliegt, vermeidet usw.
 Somit ist die Bewegung des Körpers während des Schlafs ein völlig normaler Prozeß als Antwort auf die Schmerzen, die entstehen, wenn man zu lange in einer Position bleibt.
 Eine Minderheit scheint in der Lage zu sein, im Grunde die ganze Nacht in einer Position zu schlafen; solche Menschen müssen eine Methode haben, die Auflagepunkte mit Sauerstoff usw. zu versorgen, so daß keine wunden Stellen entstehen (vielleicht bewegen sie sich unmerklich nach der einen oder anderen Seite, um den Druck zeitweilig zu verringern).
 

Ich glaube, daß ich hier ein paar überzeugende Beispiele dafür aufgeführt habe, wie Träume eher auf der Wirklichkeit basierend interpretiert werden können als auf den falschen übernatürlichen Kräften, die historisch bedingt mit der Traumdeutung verbunden werden.
 Dieser Ansatz scheint eine Einsicht in die Arbeitsweise des Gehirns während des Schlafs zu liefern.
 Eine mögliche Anwendung von Träumen, die mit der Realität verbunden werden können, ist, daß sie zu nützlichen Diagnosewerkzeugen für Störungen wie z.B. Schlafapnoe werden können.
 Sie können uns viel über unsere Bewegungen während des Schlafs sagen und darüber, wie man etwas ändern kann, damit man besser schlafen kann.
 
%  c36.html 
\hypertarget{c3_6}{}

\section{Das Unterbewußtsein}\hypertarget{c3_6a}{}

\subsection{Einleitung}

Das Gehirn hat einen bewußten und einen unterbewußten Teil.
 Die meisten Menschen haben es nicht gelernt, das Unterbewußtsein zu benutzen, aber das Unterbewußtsein ist wichtig, weil es
 
\begin{itemize}
	\item die Emotionen kontrolliert,
	\item 24 Stunden am Tag funktioniert, egal ob man wach ist oder schläft, und
	\item einige Dinge tun kann, die das Bewußtsein nicht kann, einfach weil es eine andere Art Gehirn ist.
\end{itemize}

Obwohl es schwierig ist, das bewußte Gehirn mit dem unterbewußten zu vergleichen, weil sie verschiedene Funktionen ausüben und verschiedene Fähigkeiten haben, könnten wir statistisch vermuten, daß das Unterbewußtsein bei der Hälfte der menschlichen Bevölkerung cleverer ist als das bewußte.
 Zusätzlich zur Tatsache, daß man eine zusätzliche Fähigkeit des Gehirns hat, macht es deshalb keinen Sinn, diesen Teil des Gehirns, der eventuell cleverer ist als der bewußte Teil, nicht zu benutzen.
 In diesem Abschnitt präsentiere ich meine Ideen dazu, wie das Unterbewußte eventuell funktioniert und zeige, wie wir mit Hilfe des Unterbewußtseins einige erstaunliche Leistungen vollbringen können.
 \hypertarget{c3_6b}{}

\subsection{Emotionen}

Das Unterbewußtsein kontrolliert Emotionen auf mindestens zwei Arten.
 Die erste ist eine schnelle Kampf- oder Fluchtreaktion - das Erzeugen von sofortiger Wut oder Furcht.
 Wenn solche Situationen aufkommen, muß man schneller reagieren können als man denken kann, so daß das bewußte Gehirn durch etwas umgangen werden muß, das für eine sofortige Reaktion fest verdrahtet und vorprogrammiert ist.
 Die zweite ist ein sehr langsames, schrittweises Erkennen einer tiefen oder grundlegenden Situation.
 Ob der erste und der zweite Teil des unterbewußten Gehirns Teile desselben Unterbewußtseins sind, ist eine akademische Frage, da wir fast mit Sicherheit viele Arten eines unterbewußten Verhaltens besitzen.
 Gefühle der Depression während einer Midlife-Krise könnten das Ergebnis von Vorgängen der zweiten Art von Unterbewußtsein sein: Das unterbewußte Gehirn hat während man älter wird Zeit gehabt, alle negativen Situationen herauszufinden, die sich entwickeln, und die Zukunft fängt an weniger hoffnungsvoll auszusehen.
 Solch ein Prozeß erfordert die Auswertung von Myriaden guter und schlechter Möglichkeiten, die die Zukunft bringen mag.
 Wenn man versuchen wollte, solch eine zukünftige Situation zu bewerten, müßte das bewußte Gehirn alle diese Möglichkeiten auflisten, jede bewerten und versuchen, sie zu behalten.
 Das Unterbewußtsein funktioniert anders.
 Es bewertet verschiedene Situationen auf eine unsystematische Weise; wie es eine bestimmte Situation für die Beurteilung auswählt, unterliegt nicht unserer Kontrolle; das wird mehr von alltäglichen Ereignissen kontrolliert.
 Das Unterbewußtsein speichert seine Schlußfolgerungen auch in etwas, was man \enquote{Emotionsfach} nennen könnte.
 Für jede Emotion gibt es ein Fach und jedesmal, wenn das Unterbewußtsein zu einem Entschluß kommt, sagen wir zu einem glücklichen, dann deponiert es den Entschluß in einem \enquote{Glücklichfach}.
 Der Füllgrad jedes Fachs bestimmt Ihren emotionalen Zustand.
 Das erklärt, warum Menschen oftmals spüren können, was richtig oder falsch ist oder ob eine Situation gut oder schlecht ist, ohne daß sie genau wissen, was die Gründe dafür sind.
 So beeinflußt das Unterbewußtsein unser Leben viel mehr als die meisten von uns merken.
 \hypertarget{c3_6c}{}

\subsection{Das Unterbewußtsein benutzen}

Üblicherweise geht das Unterbewußtsein seine eigenen Wege; man kontrolliert normalerweise nicht, welche Möglichkeiten es in Betracht zieht, weil die meisten von uns nicht gelernt haben, mit ihm zu kommunizieren.
 Die Ereignisse, denen man im täglichen Leben begegnet, machen es jedoch in der Regel ziemlich deutlich, welche Faktoren wichtig sind und welche unwichtig, und es zieht das Unterbewußtsein ganz natürlich zu den wichtigsten Ideen.
 Wenn diese wichtigen Ideen zu wichtigen Schlüssen führen, wird es interessierter.
 Wenn sich eine genügende Zahl solcher wichtiger Schlüsse aufstapeln, wird es sich mit Ihnen in Verbindung setzen.
 Das erklärt, warum manchmal plötzlich eine unerwartete Eingebung in unserem Bewußtsein aufblitzt.
 Darum ist hier die wichtige Frage, wie man am besten mit seinem Unterbewußtsein kommunizieren kann.
 

Jede Idee, von der man sich selber überzeugen kann, daß sie wichtig ist, oder jedes Rätsel oder Problem, das man mit großer Mühe zu lösen versucht hat, wird offensichtlich ein Kandidat zur Überprüfung durch das Unterbewußtsein sein.
 Das ist deshalb eine Art, wie man sein Problem dem Unterbewußtsein präsentieren kann.
 Außerdem muß das Unterbewußtsein, um in der Lage zu sein, ein Problem zu lösen, alle notwendigen Informationen besitzen.
 Deshalb ist es wichtig, alles zu untersuchen und so viele Informationen über das Problem zu sammeln wie man kann.
 Im College habe ich auf diese Art viele Probleme meiner Hausaufgaben gelöst, die meine klügeren Klassenkameraden nicht lösen konnten.
 Sie haben versucht, sich einfach hinzusetzen, ihre Aufgabe zu bearbeiten und hofften, das Problem zu lösen.
 Probleme in einer schulischen Umgebung sind solche, die immer mit den Informationen, die im Klassenzimmer oder Lehrbuch gegeben wurden, lösbar sind.
 Man muß nur die richtigen Teile zusammenfügen, um auf die Antwort zu kommen.
 Ich habe mir deshalb keine Gedanken darüber gemacht, ob ich in der Lage wäre, das Problem sofort zu lösen, sondern habe nur intensiv darüber nachgedacht, um sicherzustellen, daß ich das ganze Kursmaterial studiert hatte.
 Wenn ich ein Problem nicht sofort lösen konnte, wußte ich, daß mein Unterbewußtsein weiter daran arbeiten würde, so daß ich das Problem einfach vergessen und später dazu zurückkehren konnte.
 Somit war es nur erforderlich, daß ich nicht bis zur letzten Minute wartete, um zu versuchen solche Probleme zu lösen.
 Einige Zeit danach würde die Antwort plötzlich in meinem Kopf auftauchen, oftmals zu merkwürdigen und unerwarteten Gelegenheiten.
 Sie tauchten meistens am frühen Morgen auf, wenn mein Geist erholt und frisch war.
 Man kann also der Erfahrung nach sowohl lernen, dem Unterbewußtsein das Material zu präsentieren, als auch die Schlußfolgerungen daraus zu empfangen.
 Im allgemeinen kam die Antwort nicht, wenn ich mein Unterbewußtsein absichtlich darum bat, sondern sie kam, wenn ich etwas tat, das mit dem Problem nicht in Zusammenhang stand.
 Man kann das Unterbewußtsein auch benutzen, um sich an etwas zu erinnern, das man vergessen hat.
 Versuchen Sie zunächst, sich so gut Sie können daran zu erinnern, und bemühen Sie sich dann für eine Weile überhaupt nicht mehr.
 Nach einiger Zeit wird sich Ihr Gehirn oftmals für Sie daran erinnern.
 

Selbstverständlich kennen wir bis jetzt noch keinen direkten Weg, uns mit unserem Unterbewußtsein zu unterhalten.
 Und diese Kommunikationskanäle sind von Person zu Person sehr verschieden, so daß jede Person experimentieren muß, um herauszufinden was am besten funktioniert.
 Klar kann man die Kommunikation mit ihm sowohl verbessern, als auch die Kommunikationskanäle blockieren.
 Viele meiner clevereren Freunde im College wurden sehr frustriert, wenn sie herausfanden, daß ich die Antwort ohne Anstrengung gefunden hatte, während sie es nicht konnten; und sie wußten, daß sie cleverer waren.
 Diese Art der Frustration kann jegliche Kommunikation zwischen den verschiedenen Teilen des Gehirns blockieren.
 Es ist besser, eine entspannte, positive Einstellung aufrechtzuerhalten und das Gehirn seine Sache erledigen zu lassen.
 Das ist wahrscheinlich der Grund, warum Techniken wie Meditation und Qi Gong so gut funktionieren.
 Das sind effektive, lange Zeit getestete, Methoden der Kommunikation mit den verschiedenen Teilen des Gehirns und des Körpers.
 Beachten Sie, daß die verschiedenen Teile des Gehirns viele Körperfunktionen direkt kontrollieren, wie z.B. die Herzfrequenz, den Blutdruck, die Atmung, Verdauung, Speichelbildung, die Funktion der inneren Organe, sexuelle Reaktionen usw.
 Das sind mächtige Funktionen, die große Mengen von Energie erzeugen oder verschwenden können, so daß wie die Teile reibungslos zusammenarbeiten oder gegeneinander agieren einen wichtigen Effekt auf Ihre allgemeine Gesundheit und geistige Funktionen hat.
 Eine weitere wichtige Methode, einen maximalen Nutzen aus dem Unterbewußtsein zu ziehen, ist, es ohne Störung durch das bewußte Gehirn sich selbst zu überlassen, nachdem man ihm das Problem präsentiert hat.
 Mit anderen Worten: Sie sollten das Problem vergessen und sich sportlich betätigen, ins Kino gehen oder etwas anderes tun, das Ihnen Spaß macht, und das Unterbewußtsein wird seine Aufgabe besser erfüllen, weil es ein völlig anderer Teil Ihres Gehirns ist.
 Wenn Sie die ganze Zeit bewußt über das Problem nachdenken, dann beeinflussen Sie das Unterbewußtsein und erlauben ihm nicht, seine eigenen freien Forschungen zu betreiben.
 

Das Gehirn hat viele Teile, und es ist von Vorteil, jedes Teil zu kennen und zu lernen, wie man es benutzt.
 Das unterbewußte Gehirn ist wahrscheinlich eines der am meisten zu wenig genutzten Teile unseres Gehirns, weil zu vielen von uns seine Existenz nicht bewußt ist.
 Es muß bestimmt noch viele andere nützliche Teile unseres Gehirns geben.
 So gibt es z.B. zahlreiche automatische Gehirnprozesse, die unser tägliches Leben beeinflussen.
 Wenn wir ein Bild mit unseren Augen sehen, geschehen viele Dinge sofort und automatisch.
 Wenn man ein Bild empfängt, wird das Gehirn vorübergehend mit der Informationsverarbeitung überladen, so daß es andere Aufgaben nicht gut ausführen kann.
 Deshalb spürt man auch mit geöffneten Augen weniger Schmerzen als mit geschlossenen.
 Ein ähnlicher Effekt tritt bei Geräuschen auf.
 Deshalb vermindert Schreien bei Schmerzen tatsächlich die Schmerzen.
 Der angenehme Klang der Musik ist eine weitere automatische Reaktion, genauso ist es bei Reaktionen auf visuelle Eingaben wie schönen Blumen, beruhigenden Panoramablicken von Bergen oder Seen, oder der Wirkung von unangenehmen oder angenehmen Düften.
 Es ist eine dieser automatischen Reaktionen, die wir aufrufen, wenn wir Musik anhören;  trotzdem, gerade so wie wir nicht genau erklären können, warum eine schöne Blume schön aussieht, können wir nicht genau erklären, warum Musik so gut klingt.
 Vielleicht ist es eine von diesen festverdrahteten unterbewußten Reaktionen.
 

Die Identifizierung der verschiedenen Teile des Gehirn muß sicherlich eine der zukünftigen Revolutionen sein.
 Die medizinische Wissenschaft schreitet immer schneller voran, und das Gehirn zu verstehen wird einer der größten Durchbrüche sein, angefangen damit, wie es sich in der Kindheit entwickelt und wie wir diese Entwicklung erleichtern können.
 Deshalb ist es voll und ganz möglich, daß Mozart kein musikalisches Genie war, sondern ein Genie, das durch die Musik erzeugt wurde.

